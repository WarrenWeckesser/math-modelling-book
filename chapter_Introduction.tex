\chapter{Introduction}
%
Love and relationships.  Battles and wars.
Drugs and disease. Fun and games.
These are not the typical subjects that most
people think of when asked ``How is mathematics used
in the real world?''

...

This is a text on \emph{mathematical modeling}.
Actually, it is a bit narrower than that.
It is a text on modeling a selected class of
problems.  These are \emph{dynamic} problems;
problems in which \emph{time} (either continuous
or discrete) is the independent variable.

\bigskip
\noindent
\emph{``The merit of painting lies in the exactness of reproduction.
      Painting is a science and all sciences are based on mathematics. 
      No human inquiry can be a science unless it pursues its path
      through mathematical exposition and demonstration.''}

\hfill -- Leonardo Da Vinci

\bigskip
\noindent
\emph{``As far as the laws of mathematics refer to reality, they are not
certain, and as far as they are certain, they do not refer to reality.''}

\hfill  --Albert Einstein
\index{Einstein, Albert}

%
%%%%%%%%%%%%%%%%%%%%%%%%%%%%%%%%%%%%%%%%%%%%%%%%%%%%%%%%%%%
%
\newpage

\section{Modeling: An Introductory Example}

\subsection*{Classifying Mathematical Models -- An Example}
We consider the following scenario.
During a storm, a large tree with several
mice is blown into the ocean.  The storm carries the
tree many miles until it washes ashore on an
island that, until now, has had no mice.
This island has many seed-bearing plants that
mice love, and a nice climate, so the mice
have a good chance to survive and prosper.

How will the population of mice on this island
change over time?

For simplicity, we will consider just the population of female
mice.
We will also assume that a new generation is produced
each year.
We begin by making the following assumptions
\begin{enumerate}
\item In each generation, each female produces three offspring
(along with some number of males).
\item The offspring can reproduce after one year.
\item Mice live forever.
\end{enumerate}
Clearly these are not realistic assumptions.
We will accept them for now, in order to develop
a simple model.  Later we will look at some more
realistic variations.

To start, we will work with
\emph{discrete time}.
Let $p(n)$ be the population at the end of the
$n$th year, where $n$ is an integer.
With this notation, $p(0)$ is the initial population.
Let's suppose that $p(0)=1$; that is, the initial
population contained just one female mouse.

At the end of the first year, this mouse has produced
three female offspring,
so $p(1) = 4$.  At the end of the second year,
each of the four mice has produced three more offspring,
so $p(2) = 16$.
In general, we have
\begin{equation}
   p(n+1) = 4p(n)
\label{eqn:mousepop}
\end{equation}
Equation \eqref{eqn:mousepop} is the rule that describes
how the population changes over time.  Such a rule
involving discrete time is sometimes called 
a \emph{difference equation}.
\index{difference equation}
This terminology is a bit clearer if we rewrite the
equation as
\begin{equation}
    p(n+1)-p(n) = 3p(n)
\label{eqn:mousepopdiff}
\end{equation}
This gives the rule for computing the difference
between successive generations.

We can easily verify that \eqref{eqn:mousepop}
has the solution
\begin{equation}
   p(n) = p(0)4^n
\label{eqn:mousepopsol}
\end{equation}
This is a solution in the sense that the 
population at time $n$ is given directly
as a function of $n$ and the initial
population.

If we plot this solution, we will see
a ``stair-step'' plot, with the size of
the steps getting larger as $n$ increases.
If all the mice produce their offspring
at exactly the same time, then this stair-step
shape is reasonable. But we don't really expect
that to be the case.  Presumably mice will
be born throughout the year, and we expect the
actual graph of the population to have many smaller steps.
In fact, when the
population is large (and if we blur our vision
a bit), we might expect the graph to look like
a smooth curve.
Let $p(t)$ be the population at time $t$, where
now $t$ is a real number.
What mathematical
rule is obeyed by $p(t)$?


If we still believe our assumptions,
we still expect that in one year, the population
increases four-fold.
That is, we still have
\begin{equation}
    p(t+1) = 4p(t)
\label{eqn:mousepopcont}
\end{equation}
What is the corresponding rule for increments of
time less than one year? That is, what can we
say about $p(t+\frac{1}{2})$, $p(t+\frac{1}{3})$,
or in general, $p(t+h)$?
I claim that the correct rule is
\begin{equation}
   p(t+h) = 4^{h}p(t)
\label{eqn:mousepoph}
\end{equation}
If $h=0$, we obtain $p(t)=p(t)$, as we should, and if
$h=1$, we obtain \eqref{eqn:mousepopcont}.
If $h=1/k$, where $k$ is an integer, we have
\begin{equation}
\begin{split}
   p(t+1) & = p(t+\frac{k-1}{k} + \frac{1}{k}) \\
          & = 4^{1/k}p(t+\frac{k-1}{k}) \\
	  & = 4^{2/k}p(t+\frac{k-2}{k}) \\
	  & \quad \vdots \\
	  & = 4^{\frac{k-1}{k}}p(t+\frac{1}{k}) \\
	  & = 4p(t)
\end{split}
\end{equation}
so the repeated application of \eqref{eqn:mousepoph}
with $h=1/k$ also agrees with \eqref{eqn:mousepopcont}.

We now subtract $p(t)$ from both sides of
\eqref{eqn:mousepoph}, and divide by $h$:
\begin{equation}
   \frac{p(t+h) - p(t)}{h} =
     \frac{4^{h} p(t) - p(t)}{h} = \left(\frac{4^h-1}{h}\right)p(t).
\end{equation}
Take the limit $h\rightarrow 0$.
On the left we obtain $p'(t)$.
On the right, we apply L'Hopital's Rule
(and recall that $\frac{d}{dx}\left[a^x\right] = \ln(a)a^x$)
to obtain $\ln(4)p(t)$.
Thus we have
\begin{equation}
   p'(t) = \ln(4)p(t)
\label{eqn:mousepopdiffeq}
\end{equation}
This is a \emph{differential equation}.
\index{differential equation}
This equation says that the
instantaneous rate of change of the population
at time $t$ is proportional to the population at time $t$;
the proportionality constant is $\ln(4)$.

Equations \eqref{eqn:mousepop} and
\eqref{eqn:mousepopdiffeq} both give rules
for determining the population.
The first is a \emph{discrete time} model,
and the second is a \emph{continuous time}
model.
This distinction is one of the fundamental
categorizations of models.

We now consider a more complicated discrete model, 
in which we no longer assume that the mice live forever.
Suppose the mice only live three years.
Moreover, each female mouse produces two female
offspring during its second year and its third year.
At time $n$, we need \emph{three}
quantities to describe the state of the population.
We define
\begin{itemize}
\item $p_0(n)$ is the number of new female offspring
in year $n$;
\item $p_1(n)$ is the number of one-year-old females in year $n$; and
\item $p_2(n)$ is the number of two-year-old females in year $n$. 
\end{itemize}
Then we have
\begin{equation}
\begin{split}
   p_0(n+1) & = 2p_1(n) + 2p_2(n) \\
   p_1(n+1) & = p_0(n) \\
   p_2(n+1) & = p_1(n)
\end{split}
\label{eqn:mouseagemodel}
\end{equation}
Suppose that in the initial population,
$p_0(0)=0$, $p_1(0)=2$ and $p_2(0)=0$.
Let's compute the population for a few generations:

\centerline{%
\begin{tabular}{ccccc}
%   MM \= MMM \= MMM \= MMM \= MMMM \kill
   $n$ & $p_0(n)$ & $p_1(n)$ & $p_2(n)$ & Total \\
    0  &   0      &   2      &    0     &   2 \\
    1  &   4      &   0      &    2     &   6 \\
    2  &   4      &   4      &    0     &   8 \\
    3  &   8      &   4      &    4     &   16 \\
    4  &   16     &   8      &    4     &   28
\end{tabular}
}

We appear to have a growing population, but unlike
the simpler model, a formula for the solution
is not obvious.  (We will see how to solve a problem
like this later in the course.)

A key observation to make about the model
is that the ``state'' of the population
is three dimensional. In order to write down the rules
that determine how
the population changes, we needed to keep track
of three quantities.  We can put these in a vector:
\begin{equation}
   \BX(n) = \begin{bmatrix} p_0(n) \\ p_1(n) \\ p_2(n)
            \end{bmatrix}
\end{equation}
Then the rules given in
\eqref{eqn:mouseagemodel} can be written
more concisely as
\begin{equation}
   \BX(n+1) = \BF(\BX(n))
\label{eqn:general_map}
\end{equation}
where $\BF$ is the vector-valued function
(or \emph{map}, or \emph{mapping})
\index{map, mapping}
given by the right side of \eqref{eqn:mouseagemodel}.

Equation \eqref{eqn:general_map}
is a general form for multi-dimensional
discrete time models.
We will often call such an equation an
\emph{iterated map}.
\index{iterated map}
 (Just like the one-dimensional
case, these are also often called \emph{difference equations}.)
\index{difference equation}

We will also study multi-dimensional
\emph{systems of differential equations}:
\index{differential equation, system}
\begin{equation}
   \BX'(t) = \BF(\BX(t))
\label{eqn:general_diffeq}
\end{equation}
where
\begin{equation}
   \BX(t) = \begin{bmatrix}x_1(t) \\ x_2(t) \\ \vdots \\ x_m(t)\end{bmatrix}
   \quad
   \textrm{and}
   \quad
   \BX'(t) = \begin{bmatrix}x_1'(t) \\ x_2'(t) \\ \vdots \\ x_m'(t)\end{bmatrix}
\end{equation}

\subsection*{Deterministic vs. Stochastic Models.}
We have one more important categorization to discuss.
Both \eqref{eqn:general_map} and \eqref{eqn:general_diffeq}
are \emph{deterministic}.\index{deterministic}
That is, for a given starting state (e.g. $\BX(0)$),
the fate of the population is determined;
the equations produce only one possible solution.
There are no random events incorporated in the model.

Models that explicitly include random events are 
called \emph{stochastic}.\index{stochastic}
For example, suppose we model a population of mice
with the rule that in each year, there is a one in ten
chance that a mouse will die, and a one in two chance that
the mouse will produce one offspring.
(This means there is a four in ten chance that the mouse
will not die and will not produce an offspring.)
In such a model, if the initial population is 1, then
in the next year the population could be 0, 1, or 2.
The following year it could be 0, 1, 2, 3 or 4.
Figure \ref{fig:StochModel} shows five instances of
the population for this model. In all five cases,
the initial population is 2.

\begin{figure}
\centerline{\includegraphics[width=3.5in]{matlab/StochModel.eps}}
\caption{Five instances of the population in 
a stochastic model.}
\label{fig:StochModel}
\end{figure}

In such models, the question that we ask is not
``What is the population in year $n$?''
Rather, we usually ask ``What is the probability
distribution of the population in year $n$?''
If we have the probability distribution at year
$n$, we can then answer questions such as
``What is the probability that the population is zero?''
or ``What is the probability that the population is
greater than 500 in year 10?''

Figure \ref{fig:StochDist} shows a numerically
computed distribution of the population after
five years. This was computed by running 50000
simulations, and adding up the number of times
each possible final population occurred.
For example, in the figure we see that
approximately 3.6\% of the time, the population
is zero after five years.

\begin{figure}
\centerline{\includegraphics[width=3.5in]{matlab/StochDist.eps}}
\caption{Numerically computed population
probability distribution for the population
after five years.  This was computed by
tallying the results of 50000 individual
simulations.}
\label{fig:StochDist}
\end{figure}
%
\newpage
%
\begin{exercises}
\begin{exercise}
Except for a change of the coefficient, the equation for
growth of money in a bank account that earns interest
is the same as \eqref{eqn:mousepop}.
Suppose that a savings account has an
\emph{annual yield} of $5.5$ percent.
This means that, if $p(t)$ is the amount of money in the
account at some time $t$, the amount of money in the account
one year later is $p(t) + 0.055p(t) = 1.055p(t)$.
So
\begin{equation}
     p(t+1) = 1.055p(t)
\end{equation}
Find the corresponding
differential equation
for $p(t)$ if we allow $t$ to be a \emph{continuous}
variable.
\end{exercise}
\end{exercises}
