%
\chapter{Probability}
%
\section{Basic Definitions and Results}
%
yada yada...

Let $\mathcal{E}$ be the \emph{event space},
and let $\textrm{Pr}$ be a function that assigns
a real number to each subset of $\mathcal{E}$.
Let $A$ and $B$ be disjoint subsets of $\mathcal{E}$.
$\textrm{Pr}$ has the following properties.
\begin{equation}
\begin{split}
   \textrm{Pr}(\emptyset) & = 0 \\
   \textrm{Pr}(\mathcal{E}) & = 1 \\
   \textrm{Pr}(A\cup B) & = \textrm{Pr}(A) + \textrm{Pr}(B)
\end{split}
\end{equation}

\bigskip
%
To illustrate the concepts, we will often use a single six sided
die, or a pair of six sided dice.  The event space for a single
die is $\{1,2,3,4,5,6\}$.
The event space for a pair of die is given in the following table:

\noindent
\centerline{%
\begin{tabular}{cccccc}
1,1 & 2,1 & 3,1 & 4,1 & 5,1 & 6,1 \\
1,2 & 2,2 & 3,2 & 4,2 & 5,2 & 6,2 \\
1,3 & 2,3 & 3,3 & 4,3 & 5,3 & 6,3 \\
1,4 & 2,4 & 3,4 & 4,4 & 5,4 & 6,4 \\
1,5 & 2,5 & 3,5 & 4,5 & 5,5 & 6,5 \\
1,6 & 2,6 & 3,6 & 4,6 & 5,6 & 6,6 \\
\end{tabular}
}
There are $36$ possible events.
Our assumption that each die is ``fair'' implies that
each event is equally likely.

\begin{xexample}
Suppose two dice are rolled.

\medskip
\emph{What is the probability that
the sum is greater than 6?}

\medskip
By inspection of the above table, we count $21$ events where
the sum is greater than $6$.  Therefore the probability
that the sum is greater than $6$ is $21/36$, or $7/12$.

\medskip
\emph{What is the probability that one of the dice is a $1$?}

\medskip
Again we inspect the table above, and find $11$ events where
one of the dice is a $1$, so the probability is $11/36$.
Or we could reason as follows.  The probability that the
\emph{first} die is a $1$ is $1/6$, and the probability
that the second die is a $1$ is also $1/6$.  The sum
of these is not quite the probability that the first
or the second die is $1$, because by adding these
numbers, we have counted the event $(1,1)$ twice.
So we must subtract the probability that \emph{both}
the die are $1$. Thus the probability that one of the dice
is a $1$ is $(1/6)+(1/6)-(1/36) = 11/36$.  This is an
example of the rule
\[
   \textrm{Pr}(A \;\textrm{or}\; B) = \textrm{Pr}(A) + \textrm{Pr}(B) - \textrm{Pr}(A \;\textrm{and}\; B)
\]
In this case, the description of $A$ is ``the first die is a 1''.
By interpeting this as a subset of the event space, we have
$A = \{(1,1),(1,2),(1,3),(1,4),(1,5),(1,6)\}$.
Similarly, $B = \{(1,1),(2,1),(3,1),(4,1),(5,1),(6,1)\}$.
When we intepret $A$ and $B$ as subsets of the event space, the
above formula may be written
\[
  \textrm{Pr}(A \cup B) = \textrm{Pr}(A) + \textrm{Pr}(B) - \textrm{Pr}(A \cap B)
\]
\end{xexample}

\section{Conditional Probability}
%
\begin{xexample}
Two dice are rolled, and you are told that the sum of the dice
is greater than 6.  What is the probability that one of the die
is a 5?

\medskip
\noindent
\emph{Answer:} 3/7
\end{xexample}
%
\section{Probability Density Functions\index{probability density function}}
%
We now consider \emph{continuous} event spaces. Specifically,
we will consider the cases where $\mathcal{E}$ is the entire real line,
or $\mathcal{E}$ is the set of nonnegative real numbers.
The subsets of $\mathcal{E}$ for which we will compute probabilities
will be intervals of the real line, such as $1 < x < 3$, or
$100 \le x < \infty$.

\medskip
\noindent
\textbf{The Exponential Distribution}
\begin{equation}
 p(x) = \lambda e^{-\lambda x}, \quad (x \ge 0)
\end{equation}
\begin{equation}
  C(x) = \int_0^x p(s)\,ds =  1 - e^{-\lambda x}
\end{equation}
\begin{xexample}
A person enters a queue at time $t=0$ for which the
waiting time is exponentially distributed, with mean $20$ minutes.

\noindent
$\bullet$
 What is the probability that the person leaves the
queue in less than 10 minutes?

\noindent
\emph{Answer:} $\ds \textrm{Prob}(t < 10) =\int_0^{10} \frac{1}{20} \exp\left(\frac{-x}{20}\right)\,dx
     = C(10) = 1-e^{1/2} \approx 0.3935$

\noindent
$\bullet$
What is the probability that the person is in the queue for
more than an hour?

\noindent
\emph{Answer:} $\ds \textrm{Prob}(t > 60) = \int_{60}^{\infty} \frac{1}{20} \exp\left(\frac{-x}{20}\right)\,dx
     = 1-C(60) = e^{-3} \approx 0.04979$

\noindent
$\bullet$
If the person has not left the queue after 20 minutes, what is the
probability that the person will still be in the queue after an additional
20 minutes?
\label{ex:exp_example}
\end{xexample}
%
\section{Other Frequently Used Density Functions}
%
\noindent
\textbf{The Gamma Distribution}\index{gamma distribution}
\begin{equation}
  p(x) = \frac{\lambda^{k} x^{k-1} e^{-\lambda x}}{\Gamma(k)}
   \quad (x \ge 0)
\end{equation}
The parameter $k>0$ is called the \emph{shape parameter},
and $\lambda > 0$ is called the \emph{scale parameter}.
The mean is $k/\lambda$.

When $k$ is an integer, $\Gamma(k) = (k-1)!$; in this case,
the gamma distribution is known as the \emph{Erlang distribution}:
\begin{equation}
  p(x) = \frac{\lambda^{k} x^{k-1} e^{-\lambda x}}{(k-1)!}
   \quad (x \ge 0)
\end{equation}
When $k=1$, the distribution is simply the exponential distribution.
%
\begin{xexample}
A person enters a queue at time $t=0$ for which the
waiting time is a random variable with a gamma distribution
with mean $20$ minutes and shape parameter $k=2$.
(Compare this example to Example \ref{ex:exp_example}.)
\begin{enumerate}
\item What is the probability that the person leaves the
queue in less than 10 minutes?

\noindent
\emph{Answer:} $\ds \textrm{Prob}(t < 10) = \int_0^{10} \ldots \,dx
     = \ldots$
\item What is the probability that the person is in the queue for
more than an hour?

\noindent
\emph{Answer:}
 $\ds \textrm{Prob}(t > 60) = \int_{60}^{\infty} \ldots \,dx
     = \ldots$

\item If the person has not left the queue after 20 minutes, what is the
probability that the person will still be in the queue after an additional
20 minutes?
\end{enumerate}
\end{xexample}

\noindent
\textbf{The Lognormal Distribution}\index{lognormal distribution}...
