\documentclass{book}
\usepackage{makeidx}
\title{\textbf{Interdisciplinary Mathematical Modeling}}
\author{Warren Weckesser\\
Department of Mathematics\\
Colgate University}
\makeindex
\begin{document}
\maketitle
\tableofcontents
%
\chapter*{Foreword}
\paragraph{Rough notes for now...}
\begin{itemize}
\item  I have intentionally avoided examples from physics.
One goal of this book is
to promote the use of mathematical models in fields outside of the
physical sciences.
\item The emphasis of the modeling is dynamic processes.
The mathematical topics are differential equations, discrete maps,
and Markov Chains (which are, in fact, linear maps).
Optimization, linear programming and other related topics
are \emph{not} covered in this book.
\item Prerequisite for the book include calculus (single variable)
and basic linear algebra.  The essential facts from linear algebra that
are needed in this book are summarized in an appendix.
\end{itemize}
%
\chapter{Introduction}
%
\chapter{Differential Equations}
\section{What is a Differential Equation?}
\section{Modeling with Differential Equations}
\section{First Order Differential Equations}
...
\medskip
\noindent
If  $f'(x_{0})=0$, then $x(t)=x_{0}$ is a solution.  Such a constant
solution is called an \emph{equilibrium solution}.\index{equilibrium}
\subsection{Population Growth--Malthusian\index{Malthusian}}
\subsection{The Logistic Equation\index{logistic equation}}
... carrying capacity\index{carrying capacity}
\section{Systems of Differential Equations}
\subsection{Phase Plane Analysis}
\subsection{Equilibrium Points and Stability}
... The Jacobian\index{Jacobian} is ...
\subsection{Higher Dimensional Systems}
%
\chapter[Dimensional Analysis]{Dimensional Analysis and Nondimensionalization}
\section{Dimensions and Units}
\section{Nondimensional Equations and Parameters}
\section{Exercises}
%
\chapter{Applications of Differential Equations}
\section{Economics}
\subsection{Solow Growth Model}
\section{Population Models}
\subsection{Predator-Prey\index{predator-prey}}
\subsection{Competing Species\index{competing species}}
\section{War and Peace}
\subsection{Battle of Attrition}
\subsection{Arms Race}
\section{Love and Marriage}
\subsection{Romeo and Juliet\index{Romeo}\index{Juliet}}
\subsection{Laura and Petrarch\index{Laura}\index{Petrarch}}
\section{Epidemiology}
\subsection{The SI Model\index{SI model}}
\subsection{The SIS Model\index{SIS model}}
\subsection{The SIR Model\index{SIR model}}
\subsection{The SIQR Model\index{SIQR model}}
\section{Exercises}

%
\chapter{Discrete Maps}
\section{Modeling with Discrete Maps}
\section{One-Dimensional Linear Maps}
\section{One-Dimensional Nonlinear Maps}
\subsection{Cobwebbing}
\section{Higher Dimensional Maps}
\section{Applications}
\subsection{Marital Relationships}
The model discussed in this section is based on the
work of Gottman, Murray, \emph{et al}\cite{GM}.
\section{Exercises}
\chapter{Hybrid Models}
\section{Introduction}
Often, models of dynamical systems lead to equations that combine
both differential equations and discrete maps.
\subsection{Periodic Drug Doses}
\section{Exercises}
%
\chapter{Markov Chains}
\section{A Brief Introduction to Probability}
\section{Markov Chains}
\subsection{An Example: The Coin and Die Game}
\section{Monopoly (tm)}
\section{Baseball}
\section{Exercises}
\appendix
\chapter{A Brief Review of Linear Algebra}
\section{Matrices}
\section{Eigenvalues and Eigenvectors\index{eigenvalue}\index{eigenvector}}
\subsection{Shortcuts for $2\times 2$ Matrices}
\chapter{Computer Tools for Computation and Graphics}
%
% Do Octave, Scilab, Maxima, and Yacas require the trademark symbol?
%
In this appendix I give a brief introduction to some
readily available software packages that can be used to solve
differential equations and discrete maps.  MATLAB (tm)
is a widely used commercial package.  Octave (tm) and
Scilab (tm) are free open source software packages. 
\section{MATLAB (tm)}
\section{Octave (tm)}
\section{Scilab (tm)}
\chapter{Computer Tools for Symbolic Computing}
The computer can be a powerful tool for performing
algebraic computations.  In this appendix I give a brief introduction
to a few software packages that can do symbolic computations.
Maple (tm) and Mathematica (tm) are widely used commercial
packages.  Maxima (tm) and Yacas (tm) are free open-source
programs.
\section{Maple (tm)}
\section{Mathematica (tm)}
\section{Maxima (tm)}
\section{Yacas (tm)}
%
\begin{thebibliography}{99}
\bibitem{GM} Gottman, J. Murray, ...
\end{thebibliography}
\printindex
\end{document}