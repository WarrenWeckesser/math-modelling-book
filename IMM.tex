\documentclass{book}
%
\usepackage{amsmath}
\usepackage{amsfonts}
\usepackage{amssymb}
%
\usepackage{graphicx}
\usepackage{makeidx}
%
\title{\textbf{Interdisciplinary Mathematical Modeling}}
\author{Warren Weckesser\\
Department of Mathematics\\
Colgate University}
%
\newtheorem{question}{Question}
%
\makeindex
%
\begin{document}
\maketitle
\tableofcontents
%
\chapter*{Foreword}
\paragraph{Rough notes for now...}
\begin{itemize}
\item  I have intentionally avoided examples from physics.
One goal of this book is
to promote the use of mathematical models in fields outside of the
physical sciences.
\item The emphasis of the modeling is dynamic processes.
The mathematical topics are differential equations, discrete maps,
and Markov Chains (which are, in fact, linear maps).
Optimization, linear programming and other related topics
are \emph{not} covered in this book.
\item Prerequisite for the book include calculus (single variable)
and basic linear algebra.  The essential facts from linear algebra that
are needed in this book are summarized in an appendix.
\end{itemize}
%
\chapter{Introduction}
%
\chapter{Differential Equations}
\section{What is a Differential Equation?}
\section{Modeling with Differential Equations}
\section{First Order Differential Equations}
...
\medskip
\noindent
If  $f'(x_{0})=0$, then $x(t)=x_{0}$ is a solution.  Such a constant
solution is called an \emph{equilibrium solution}.\index{equilibrium}
\subsection{Population Growth--Malthusian\index{Malthusian}}


The simplest assumption for modeling the growth of a population
is:
\begin{quote}
\emph{The rate of change of the population is proportional to the current population level.}
\end{quote} 
Let $t$ be time, and let $p(t)$ be the population at time $t$.
For now we'll assume that some convenient system of units is used;
we'll have more to say about units later.
We also assume that $p$ is ``large'' in some sense, so that it is reasonable
to treat $p$ as a real number, not an integer.  If, for example, the population
was just three rabbits, the following models would probably not provide good
approximations to the actual growth of the population.  If the population
is measured in thousands of rabbits, then $p(0)=3.12$ makes sense,
and the following models are more reasonable.

We convert the above assumption into an equation involving $t$
and $p$.
The ``rate of change of the population'' is the derivative, $\frac{dp}{dt}$, and the
current population is $p(t)$, so the mathematical version of the above
assumption is
\begin{equation}
  \frac{dp}{dt} = rp
\end{equation}
where $r$ is the constant of proportionality.
(Note that I have suppressed the argument of $p$ for brevity.)
This is a \emph{first order differential equation}.
(The \emph{order} of a differential equation is the order of the
highest derivative in the equation.)

Frequently, the question we ask is ``What is the population at time $t$
if the population is $P_0$ at time $t=0$?''  That is, we impose the
condition that $p(0)=P_0$.  This is called an
\emph{initial condition}, and the problem of solving the differential
equation with a given initial condition is called an
\emph{initial value problem}.

To solve this equation, we must find a \emph{function} $p(t)$
that satisfies the equation.  In this case, the solution is
\begin{equation}
  p(t) = P_0 e^{rt}
\end{equation}
(Let's check:  $\frac{dp}{dt} = rP_0e^{rt} = r p(t)$, so  it solves the differential
equation, and $p(0) = P_0e^0 = P_0$, so it also satisfies the initial condition.)
Thus, if $r > 0$, the simple assumption given above results in
\emph{exponential growth}.
(If $r < 0$, we obtain \emph{exponential decay}.)

\subsection{The Logistic Equation\index{logistic equation}}

For a population with unlimited resources, exponential growth
is a pretty accurate description of what happens.
However, no environment can support exponential growth forever.
Eventually the food runs out, or there is simply not enough space
for a larger population.
Presumably there is a maximum population level that the environment
can sustain.  This level is called the \emph{carrying capacity}
of the environment. If the population is larger than the
carrying capacity, overcrowding or a lack of food results in a
\emph{decreasing} population.
We'll modify the equation to take this into account.

The right side of the equation gives the rate of change of $p$ as a function
of $p$.
If we divide the the right side by $p$, we obtain the
\emph{per capita growth rate}, or simply the \emph{growth rate}.  For the initial assumption,
the growth rate  is just the constant $r$.
To incorporate the assumption of a carrying capacity, we will assume that the
growth rate depends on $p$.  When $p$ is near zero, we assume that there
are plenty of resources for the population to grow, so the growth rate should be
near $r$.  As the population becomes bigger, the growth rate should decrease,
and it should be zero when the population is at the carrying capacity.
If the population exceeds the carrying capacity, the growth rate should
be negative.

The simplest formula for such a growth rate is a straight line that has
the value $r$ when $p=0$ and the value $0$ when $p=K$, as shown in
Figure~\ref{fig:growthrate}.
\begin{figure}
\centerline{\includegraphics[width=3.25in]{matlab/logistic_percapita_growthrate.eps}} 
\caption{Per capita growth rate for the logistic equation.}
\label{fig:growthrate}
\end{figure} 
The equation that we obtain is
\begin{equation}
  \frac{dp}{dt} = r\left(1-\frac{p}{K}\right)p
\label{eqn:logistic}
\end{equation}
This first order differential equation is commonly called the \emph{logistic equation}.
Later we'll see how to find the exact solutions to this equation.
For now, we'll learn as much as we can about its solutions without actually solving the
equation.  To do this, we plot the right side of \eqref{eqn:logistic} as a function of $p$.
The graph is shown in Figure~\ref{fig:logisticrhs}.
\begin{figure}
\centerline{\includegraphics[width=3.25in]{matlab/logistic_growthrate.eps}} 
\caption{A plot of $\frac{dp}{dt}$ as a function of $p$ for the logistic
equation \eqref{eqn:logistic}.}
\label{fig:logisticrhs}
\end{figure}
We can use the information in Figure~\ref{fig:logisticrhs} to determine the
behavior of solutions to \eqref{eqn:logistic}.
Suppose, for example, that the population is initially ``small''.
(In this case, small means ``a small fraction of $K$''.)
Then the graph in Figure~\ref{fig:logisticrhs} tells us that
$\frac{dp}{dt}$ is also small but positive.  In other words, the slope
of $p(t)$ is small and positive.  This means that $p(t)$ is an increasing
function of $t$, so in a little while, $p$ will be larger.  Looking back to
Figure~\ref{fig:logisticrhs}, we see that this means $\frac{dp}{dt}$ will be
larger than it was before, and therefore the slope of $p(t)$ has increased.
So initially, $p(t)$ will grow faster and faster.  This is not surprising, since
when $p$ is small, $p^2$ is very small, and if we ignore the $p^2$ term in the
logistic equation, we obtain $\frac{dp}{dt} \approx rp$.  So when the
population is small, the growth is almost exponential.

Eventually the population will reach $K/2$.  This is the population
level at which the population grows the fastest.
When the population reaches this level, further increases in the
population result in \emph{lower} growth rates.
For example, if $p(t) = 0.75K$ at some time $t$, we see in
Figure~\ref{fig:logisticrhs} that $\frac{dp}{dt}$ is positive, but it is
decreasing as $p$ increases.  In the next moment, $p$ will be larger,
but then the slope of $p(t)$ will be smaller.
As $p$ gets closer and closer to $K$, the slope gets smaller and smaller.
In fact, $p(t)$ will approach $K$ asymptotically, but never reach it.
A graph of a solution to the logistic equation~\eqref{eqn:logistic} is
shown in Figure~\ref{fig:logisticsol}.
\begin{figure}
\centerline{\includegraphics[width=3.5in]{matlab/logistic_solution.eps}} 
\caption{The graph of $p(t)$, a solution to the logistic
equation \eqref{eqn:logistic}.}
\label{fig:logisticsol}
\end{figure}

\begin{question}
What happens if the initial population is greater than $K$?
\end{question}

Note that if the inital population is exactly $K$, then
$\frac{dp}{dt} = 0$.  The constant function $p(t)=K$ is an
exact solution to the differential equation.
We call a constant solution an \emph{equilibrium solution}.

\begin{question}
Does the logistic equation \eqref{eqn:logistic} have any other
equilibrium solutions?
\end{question}

The method for analyzing solutions to the logistic equation can be
applied to any first order equation of the form
\begin{equation}
   \frac{dy}{dt} = f(y)
\end{equation}
where $f(y)$ is a given function.
In particular, we assume that $t$ does \emph{not} appear explicitly
in $f$.  Such an equation is called \emph{autonomous}.
When $t$ appears explicity in the right side of the equation,
we say the equation is \emph{nonautonomous}.
For example, the equation
\begin{equation}
  \frac{dp}{dt} = (r+a \sin(\omega t))p
\end{equation}
is nonautonomous, since $t$ appears explicitly in the right side of the
equation (in the argument of the sine function).
In this case, the explicit time dependence might model a growth
rate with seasonal dependence.

The basic procedure to analyze an autonomous
first order differential equation is to sketch the graph of $f(y)$, identify
the equilibrium solutions (where $f(y)=0$), and determine the
behavior of non-equilibrium solutions based on the graph
of $f(y)$.

\begin{question}
Consider the differential equation
\[
   \frac{dy}{dt} = y(y-1)(y-2)
\]
(a) Find the equilibrium solutions.\\
(b) Sketch $\frac{dy}{dt}$ as a function of $y$.\\
(c) In one set of axes, sketch several solutions $y(t)$. Include the equilibrium solutions,
and several more non-equilibrium solutions to show all the possible behaviors.
(Since this is not necessarily a population model, you should consider negative values
of $y$ as well as positive.)
\end{question}



\section{Systems of Differential Equations}
\subsection{Phase Plane Analysis}
\subsection{Equilibrium Points and Stability}
... The Jacobian\index{Jacobian} is ...
\subsection{Higher Dimensional Systems}
%
\chapter[Dimensional Analysis]{Dimensional Analysis and Nondimensionalization}
\section{Dimensions and Units}
\section{Nondimensional Equations and Parameters}
\section{Exercises}
%
\chapter{Applications of Differential Equations}
\section{Economics}
\subsection{Solow Growth Model}
\section{Population Models}
\subsection{Predator-Prey\index{predator-prey}}
\subsection{Competing Species\index{competing species}}
\section{War and Peace}
\subsection{Battle of Attrition}
\subsection{Arms Race}
\section{Love and Marriage}
\subsection{Romeo and Juliet\index{Romeo}\index{Juliet}}
\subsection{Laura and Petrarch\index{Laura}\index{Petrarch}}
\section{Epidemiology}
\subsection{The SI Model\index{SI model}}
\subsection{The SIS Model\index{SIS model}}
\subsection{The SIR Model\index{SIR model}}
\subsection{The SIQR Model\index{SIQR model}}
\section{Exercises}

%
\chapter{Discrete Maps}
\section{Modeling with Discrete Maps}
\section{One-Dimensional Linear Maps}
\section{One-Dimensional Nonlinear Maps}
\subsection{Cobwebbing}
\section{Higher Dimensional Maps}
\section{Applications}
\subsection{Marital Relationships}
The model discussed in this section is based on the
work of Gottman, Murray, \emph{et al}\cite{GM}.
\section{Exercises}
\chapter{Hybrid Models}
\section{Introduction}
Often, models of dynamical systems lead to equations that combine
both differential equations and discrete maps.
\subsection{Periodic Drug Doses}
\section{Exercises}
%
\chapter{Markov Chains}
\section{A Brief Introduction to Probability}
\section{Markov Chains}
\subsection{An Example: The Coin and Die Game}

In this game there are two players, \emph{Coin}
and \emph{Die}. \emph{Coin} has a coin, and \emph{Die} has a
single six-sided die.

\begin{itemize}
\item
When it is \emph{Coin}'s turn, he or she flips the coin.
If the coin turns up \textbf{heads}, \emph{Coin} wins the game.
If the coin turns up \textbf{tails}, it is \emph{Die}'s turn.

\item
When it is \emph{Die}'s turn, he or she rolls the die.
If the die turns up \textbf{1}, \emph{Die} wins.
If the die turns up \textbf{6}, it is \emph{Coin}'s turn.
Otherwise, \emph{Die} rolls again.
\end{itemize}

\paragraph{Class Exercise.}
With a partner, play the game 20 times, with \emph{Coin}
going first each time.
Keep track of who wins each game, and the total
number of coin flips and rolls of the die
that takes place in each game.

Then play 20 more games, but with \emph{Die} going first
each time, and keep track of the same information.

\section{Monopoly (tm)}
\section{Baseball}
\section{Exercises}
\appendix
\chapter{A Brief Review of Linear Algebra}
\section{Matrices}
\section{Eigenvalues and Eigenvectors\index{eigenvalue}\index{eigenvector}}
\subsection{Shortcuts for $2\times 2$ Matrices}
\chapter[Computer Tools]{Computer Tools for Computation and Graphics}
%
% Do Octave, Scilab, Maxima, and Yacas require the trademark symbol?
%
In this appendix I give a brief introduction to some
readily available software packages that can be used to solve
differential equations and discrete maps.  MATLAB (tm)
is a widely used commercial package.  Octave (tm) and
Scilab (tm) are free open source software packages. 
\section{MATLAB (tm)}
\section{Octave (tm)}
\section{Scilab (tm)}
\chapter{Symbolic Computing}
The computer can be a powerful tool for performing
algebraic computations.  In this appendix I give a brief introduction
to a few software packages that can do symbolic computations.
Maple (tm) and Mathematica (tm) are widely used commercial
packages.  Maxima (tm) and Yacas (tm) are free open-source
programs.
\section{Maple (tm)}
\section{Mathematica (tm)}
\section{Maxima (tm)}
\section{Yacas (tm)}
%
\begin{thebibliography}{99}
\bibitem{GM} Gottman, J. Murray, ...
\end{thebibliography}
\printindex
\end{document}