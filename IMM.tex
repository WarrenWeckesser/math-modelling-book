\documentclass{book}
%
\usepackage{amsmath}
\usepackage{amsfonts}
\usepackage{amssymb}
%
\usepackage{graphicx}
\usepackage{makeidx}
%
\newcommand{\BA}{\vec{\textbf{a}}}
\newcommand{\BB}{\vec{\textbf{b}}}
\newcommand{\BX}{\vec{\textbf{x}}}
\newcommand{\BF}{\vec{\textbf{f}}}
\newcommand{\BZ}{\vec{\textbf{0}}}
\newcommand{\BV}{\vec{\textbf{v}}}
%
\title{\textbf{Interdisciplinary Mathematical Modeling}}
\author{Warren Weckesser\\
Department of Mathematics\\
Colgate University}
%
\newtheorem{question}{Question}
%
\makeindex
%
\begin{document}
\maketitle
\tableofcontents
%
\chapter*{Foreword}
\paragraph{Rough notes for now...}
\begin{itemize}
\item  I have intentionally avoided examples from physics.
One goal of this book is
to promote the use of mathematical models in fields outside of the
physical sciences.
\item The emphasis of the modeling is dynamic processes.
The mathematical topics are differential equations, discrete maps,
and Markov Chains (which are, in fact, linear maps).
Optimization, linear programming and other related topics
are \emph{not} covered in this book.
\item Prerequisite for the book include calculus (single variable)
and basic linear algebra.  The essential facts from linear algebra that
are needed in this book are summarized in an appendix.
\end{itemize}
%
\chapter{Introduction}
%
\chapter{Differential Equations}
\section{What is a Differential Equation?}
\section{Modeling with Differential Equations}
\section{First Order Differential Equations}
...
\medskip
\noindent
If  $f'(x_{0})=0$, then $x(t)=x_{0}$ is a solution.  Such a constant
solution is called an \emph{equilibrium solution}.\index{equilibrium}
\subsection{Population Growth--Malthusian\index{Malthusian}}


The simplest assumption for modeling the growth of a population
is:
\begin{quote}
\emph{The rate of change of the population is proportional to the current population level.}
\end{quote} 
Let $t$ be time, and let $p(t)$ be the population at time $t$.
For now we'll assume that some convenient system of units is used;
we'll have more to say about units later.
We also assume that $p$ is ``large'' in some sense, so that it is reasonable
to treat $p$ as a real number, not an integer.  If, for example, the population
was just three rabbits, the following models would probably not provide good
approximations to the actual growth of the population.  If the population
is measured in thousands of rabbits, then $p(0)=3.12$ makes sense,
and the following models are more reasonable.

We convert the above assumption into an equation involving $t$
and $p$.
The ``rate of change of the population'' is the derivative, $\frac{dp}{dt}$, and the
current population is $p(t)$, so the mathematical version of the above
assumption is
\begin{equation}
  \frac{dp}{dt} = rp
\end{equation}
where $r$ is the constant of proportionality.
(Note that I have suppressed the argument of $p$ for brevity.)
This is a \emph{first order differential equation}.
(The \emph{order} of a differential equation is the order of the
highest derivative in the equation.)

Frequently, the question we ask is ``What is the population at time $t$
if the population is $P_0$ at time $t=0$?''  That is, we impose the
condition that $p(0)=P_0$.  This is called an
\emph{initial condition}, and the problem of solving the differential
equation with a given initial condition is called an
\emph{initial value problem}.

To solve this equation, we must find a \emph{function} $p(t)$
that satisfies the equation.  In this case, the solution is
\begin{equation}
  p(t) = P_0 e^{rt}
\end{equation}
(Let's check:  $\frac{dp}{dt} = rP_0e^{rt} = r p(t)$, so  it solves the differential
equation, and $p(0) = P_0e^0 = P_0$, so it also satisfies the initial condition.)
Thus, if $r > 0$, the simple assumption given above results in
\emph{exponential growth}.\index{exponential growth}
(If $r < 0$, we obtain \emph{exponential decay}.\index{exponential decay})

\subsection{The Logistic Equation\index{logistic equation}}

For a population with unlimited resources, exponential growth
is a pretty accurate description of what happens.
However, no environment can support exponential growth forever.
Eventually the food runs out, or there is simply not enough space
for a larger population.
Presumably there is a maximum population level that the environment
can sustain.  This level is called the
\emph{carrying capacity}\index{carrying capacity}
of the environment. If the population is larger than the
carrying capacity, overcrowding or a lack of food results in a
\emph{decreasing} population.
We'll modify the equation to take this into account.

The right side of the equation gives the rate of change of $p$ as a function
of $p$.
If we divide the the right side by $p$, we obtain the
\emph{per capita growth rate},
or simply the \emph{growth rate}.\index{growth rate}
For the initial assumption,
the growth rate  is just the constant $r$.
To incorporate the assumption of a carrying capacity, we will assume that the
growth rate depends on $p$.  When $p$ is near zero, we assume that there
are plenty of resources for the population to grow, so the growth rate should be
near $r$.  As the population becomes bigger, the growth rate should decrease,
and it should be zero when the population is at the carrying capacity.
If the population exceeds the carrying capacity, the growth rate should
be negative.

The simplest formula for such a growth rate is a straight line that has
the value $r$ when $p=0$ and the value $0$ when $p=K$, as shown in
Figure~\ref{fig:growthrate}.
\begin{figure}
\centerline{\includegraphics[width=3.25in]{matlab/logistic_percapita_growthrate.eps}} 
\caption{Per capita growth rate for the logistic equation.}
\label{fig:growthrate}
\end{figure} 
The equation that we obtain is
\begin{equation}
  \frac{dp}{dt} = r\left(1-\frac{p}{K}\right)p
\label{eqn:logistic}
\end{equation}
This first order differential equation is commonly called the \emph{logistic equation}.\index{logistic equation}
Later we'll see how to find the exact solutions to this equation.
For now, we'll learn as much as we can about its solutions without actually solving the
equation.  To do this, we plot the right side of \eqref{eqn:logistic} as a function of $p$.
The graph is shown in Figure~\ref{fig:logisticrhs}.
\begin{figure}
\centerline{\includegraphics[width=3.25in]{matlab/logistic_growthrate.eps}} 
\caption{A plot of $\frac{dp}{dt}$ as a function of $p$ for the logistic
equation \eqref{eqn:logistic}.}
\label{fig:logisticrhs}
\end{figure}
We can use the information in Figure~\ref{fig:logisticrhs} to determine the
behavior of solutions to \eqref{eqn:logistic}.
Suppose, for example, that the population is initially ``small''.
(In this case, small means ``a small fraction of $K$''.)
Then the graph in Figure~\ref{fig:logisticrhs} tells us that
$\frac{dp}{dt}$ is also small but positive.  In other words, the slope
of $p(t)$ is small and positive.  This means that $p(t)$ is an increasing
function of $t$, so in a little while, $p$ will be larger.  Looking back to
Figure~\ref{fig:logisticrhs}, we see that this means $\frac{dp}{dt}$ will be
larger than it was before, and therefore the slope of $p(t)$ has increased.
So initially, $p(t)$ will grow faster and faster.  This is not surprising, since
when $p$ is small, $p^2$ is very small, and if we ignore the $p^2$ term in the
logistic equation, we obtain $\frac{dp}{dt} \approx rp$.  So when the
population is small, the growth is almost exponential.

Eventually the population will reach $K/2$.  This is the population
level at which the population grows the fastest.
When the population reaches this level, further increases in the
population result in \emph{lower} growth rates.
For example, if $p(t) = 0.75K$ at some time $t$, we see in
Figure~\ref{fig:logisticrhs} that $\frac{dp}{dt}$ is positive, but it is
decreasing as $p$ increases.  In the next moment, $p$ will be larger,
but then the slope of $p(t)$ will be smaller.
As $p$ gets closer and closer to $K$, the slope gets smaller and smaller.
In fact, $p(t)$ will approach $K$ asymptotically, but never reach it.
A graph of a solution to the logistic equation~\eqref{eqn:logistic} is
shown in Figure~\ref{fig:logisticsol}.
\begin{figure}
\centerline{\includegraphics[width=3.5in]{matlab/logistic_solution.eps}} 
\caption{The graph of $p(t)$, a solution to the logistic
equation \eqref{eqn:logistic}.}
\label{fig:logisticsol}
\end{figure}

\begin{question}
Describe (and sketch) the solution that
results when the initial population is greater than $K$.
\end{question}

Note that if the inital population is exactly $K$, then
$\frac{dp}{dt} = 0$.  The constant function $p(t)=K$ is an
exact solution to the differential equation.
We call a constant solution an
\emph{equilibrium solution}.\index{equilibrium solution}

\begin{question}
Does the logistic equation \eqref{eqn:logistic} have any other
equilibrium solutions?
\end{question}

The method for analyzing solutions to the logistic equation can be
applied to any first order equation of the form
\begin{equation}
   \frac{dy}{dt} = f(y)
\end{equation}
where $f(y)$ is a given function.
In particular, we assume that $t$ does \emph{not} appear explicitly
in $f$.  Such an equation is called \emph{autonomous}.\index{autonomous}
When $t$ appears explicity in the right side of the equation,
we say the equation is \emph{nonautonomous}.\index{nonautonomous}
For example, the equation
\begin{equation}
  \frac{dp}{dt} = (r+a \sin(\omega t))p
\end{equation}
is nonautonomous, since $t$ appears explicitly in the right side of the
equation (in the argument of the sine function).
In this case, the explicit time dependence might model a growth
rate with seasonal dependence.

The basic procedure to analyze an autonomous
first order differential equation is to sketch the graph of $f(y)$, identify
the equilibrium solutions (where $f(y)=0$), and determine the
behavior of non-equilibrium solutions based on the graph
of $f(y)$.

\begin{question}
Consider the differential equation
\[
   \frac{dy}{dt} = y(y-1)(y-2)
\]
(a) Find the equilibrium solutions.\\
(b) Sketch $\frac{dy}{dt}$ as a function of $y$.\\
(c) In one set of axes, sketch several solutions $y(t)$. Include the equilibrium solutions,
and several more non-equilibrium solutions to show all the possible behaviors.
(Since this is not necessarily a population model, you should consider negative values
of $y$ as well as positive.)
\end{question}

\subsection{Solving First Order Differential Equations: Two Special Cases}

In this section, we discuss
two techniques for solving certain first order
differential equations.
The general form for a first order differential equation is
\begin{equation}
   y'(t) = f(t,y)
\end{equation}
\paragraph{1. Separable Equations} ~

\vspace{0.2cm}
\noindent
If $f$ can be ``separated'' into a quotient of a function of $t$ and a function
of $y$ as
\begin{equation}
   f(t,y) = \frac{h(t)}{g(y)},
\end{equation}
there is a chance that the solutions can
be found analytically.
The differential equation may be written
\begin{equation}
  g(y(t))y'(t) = h(t)
\label{eqn:separated}
\end{equation}
We now integrate both sides:
\begin{equation}
    \int_0^t g(y(s))y'(s) \, ds = \int_0^t h(s) \, ds
\label{eqn:integrated}
\end{equation}
The right side is an integral of a known function.
To deal with the left side, we first let $p(y) = \int_{y_0}^y g(z) dz$.
Then  we use the chain rule:
\begin{equation}
   \frac{d}{dt} p(y(t)) = p'(y)y'(t) = g(y)y'(t).
\end{equation}
Thus, on the left side of \eqref{eqn:integrated}, we have
\begin{equation}
  \int_0^t g(y)y'(s)\,ds = \int_0^t \left(\frac{d}{ds} p(y(s))\right)\, ds = p(y(t)) = 
    \int_{y_0}^y g(z) \, dz
\end{equation}
The net result can be summarized with the following formal procedure:
\begin{enumerate}
\item Treat the derivative $\frac{dy}{dt}$ as a fraction and rewrite the differential equation
as
\begin{equation}
   g(y)dy = h(t)dt.
\end{equation}
\item Integrate with respect to $y$ on the left and with respect to $t$ on the right.
The constants of integration can be combined into one constant on the right after
integrating.
\item Solve for $y$.
\end{enumerate}
The \emph{real} short summary is:
\begin{quote}
  \emph{Separate}, \emph{Integrate}, \emph{Isolate} (i.e. solve for $y$).
\end{quote}
Note that even if $f$ can be separated into $h(t)/g(y)$, there are two potential obstacles
to this method. First, it might not be possible to evaluate the integrals.
Second, it might not be possible to solve for $y$ after integrating.
In this case, we have an implicit equation relating $y$ and $t$, which can still
be useful in some cases.

Note that all autonomous first order differential equations are separable.

\paragraph{Example 1.}
We'll apply the method to
\begin{equation}
   \frac{dp}{dt} = r p
\end{equation}
In this case, separating gives
\begin{equation}
   \frac{dp}{p} = r dt,
\end{equation}
but note that we have assumed that $p\ne 0$.
Integrating gives
\begin{equation}
   \ln | p | = rt+C_0
\end{equation}
Exponentiate both sides to obtain
\begin{equation}
  |p| = e^{rt+C_0} = C_1e^{rt}, \quad \textrm{where} \quad C_1 = e^{C_0}.
\end{equation}
Note that $C_1 > 0$.
For $p > 0$, we have $|p|=p$, so $p=C_1e^{rt}$.
If $p < 0$, $|p| = -p$, so $p = -C_1e^{rt}$.
This is the same as saying the constant in front of $e^{rt}$ is negative.
Also note that $p(t)=0$ is an equilibrium solution.
We can combine the three cases $p>0$, $p=0$ and $p<0$ into one
solution
\begin{equation}
   p(t) = Ce^{rt},
\end{equation}
where $C$ is an arbitrary constant.
To satisfy an initial condition $p(0)=P_0$, we let $C=P_0$.

\paragraph{Example 2.}
Now consider a population model in which the per capita growth
rate varies periodically.  This might be because of seasonal variation.
The differential equation is
\begin{equation}
   \frac{dp}{dt} = (r + a\cos(\omega t))p
\end{equation}
Separating gives
\begin{equation}
  \frac{dp}{p} = \left( r+a\cos(\omega t) \right) dt
\end{equation}
and integrating gives
\begin{equation}
  \ln | p | = rt + \frac{a}{\omega} \sin(\omega t) + C_0
\end{equation}
From here, the work is similar to the previous example.
The final concise formula for the solution is
\begin{equation}
   p(t) = Ce^{rt + (a/\omega)\sin(\omega t)}
\end{equation}
where $C$ is an arbitrary constant.
\paragraph{2. Linear Equations} ~

\vspace{0.2cm}
\noindent
If the first order differential equation has the form
\begin{equation}
    y'(t) = p(t) y + g(t),
\label{eqn:linear}
\end{equation}
it is called a \emph{linear} equation.
We can always express the solution to such an equation
in terms of integrals.  The only obstacle will be
evaluating the integrals.

To derive the solution, we first simply move $p(t)y$ to the left:
\begin{equation}
    y'(t) - p(t) y =  g(t),
\label{eqn:intermediateA}
\end{equation}
The left side looks vaguely like the result of applying the
product rule of differentiation to a product of $y$ and something else.
If $\mu(t)$ is some function, then $(\mu y)' = \mu y' + \mu' y$,
which suggests that we multiply \eqref{eqn:intermediateA}
by $\mu$ (whatever it is)
to obtain
\begin{equation}
    \mu y'(t) - \mu p(t) y =  \mu g(t),
\label{eqn:intermediateB}
\end{equation}
and then determine what $\mu(t)$ should
be by solving $\mu' = -\mu p(t)$.
This is a separable equation; a solution is
\begin{equation}
   \mu(t) = e^{-\int p(t)\, dt}.
\end{equation}
We could include an arbitrary multiple of this function
by multiplying it by an arbitraryconstant $C$, but we
don't need it.  Equivalently, when we find the integral
$\int p(t)\,dt$, we may choose the constant of integration
to be zero.

With this choice of $\mu$, 
\eqref{eqn:intermediateB} becomes
\begin{equation}
    (\mu y)' =  \mu g(t).
\label{eqn:intermediateC}
\end{equation}
Integrating both sides gives%
\footnote{Note that I have explicitly included the integration
constant $C$ on the right.  Normally, when we write an indefinite
integral $\int q(t)\, dt$, the constant of integration is implicitly
assumed to be part of the result. For example, $\int 2x\,dx = x^2+C$.
However, it is a common mistake to \emph{forget} the constant, so
I choose to include it explicitly.}
\begin{equation}
    \mu y =  \int \mu g(t) \, dt + C
\label{eqn:intermediateD}
\end{equation}
and so we have the following solution
to \eqref{eqn:linear}:
\begin{equation}
    y =  \frac{1}{\mu}\left(\int \mu g(t) \, dt + C\right)
    \quad \textrm{where} \quad
    \mu(t) = e^{-\int p(t)\, dt}.
\label{eqn:linearsolution}
\end{equation}

\paragraph{Example 1.}
Consider the initial value problem
\begin{equation}
   y' = 2y + 3, \quad y(0) = 5.
\label{eqn:linearexample1}
\end{equation}
This problem is linear, with $p(t)=2$ and $g(t) = 3$.
(It is also separable, so there is more than one way
to solve this problem.)
The integrating factor is
\begin{equation}
   \mu(t) = e^{-\int p(t)\,dt} = e^{-\int 2\, dt}
      = e^{-2t},
\end{equation}
and the solution is
\begin{equation}
\begin{split}
   y(t) & = e^{2t} \left( \int \left(e^{-2t}\right)\left(3\right)\,dt + C\right) \\
        & = e^{2t} \left( -\frac{3}{2} e^{-2t} +C \right) \\
	& = -\frac{3}{2} + Ce^{2t}.
\end{split}
\end{equation}
To satisfy the initial condition $y(0)=5$, we have
\begin{equation}
   y(0) = -\frac{3}{2} + C = 5 \implies C = \frac{13}{2} .
\end{equation}
So the solution to the initial value problem is
\begin{equation}
   y(t) = -\frac{3}{2} + \frac{13}{2}e^{2t}.
\end{equation}
You should check the answer by substituting it back
into \eqref{eqn:linearexample1}.

\paragraph{Example 2.}
\begin{equation}
   y' = -\frac{y}{t} + t, \quad y(1) = 2.
\end{equation}
In this case, $p(t) = -1/t$ and $g(t) = t$.
Note that the initial condition is given at
$t=1$ rather than the usual $t=0$.  Nothing
is wrong or difficult about that. We just have
to remember to evaluate the solution at $t=1$
when we solve for the constant to make
the solution satisfy the initial condition.
Also, since we can not divide by zero, we will
assume that we are only interested in the solution
where $t > 0$.

The integrating factor is
\begin{equation}
   \mu(t) = e^{-\int p(t)\,dt} = e^{\int \frac{1}{t}\, dt}
      = e^{\ln |t|} = e^{\ln t} = t,
\end{equation}
where, because we assume $t>0$,  we used $|t|=t$.
The solution is
\begin{equation}
\begin{split}
   y(t) & = \frac{1}{\mu} \left( \int \mu g(t)\,dt + C\right) \\
        & = \frac{1}{t} \left( \int t^2 \,dt+C\right) \\
	& = \frac{1}{t} \left( \frac{t^3}{3} + C \right)\\
	& = \frac{t^2}{3} + \frac{C}{t}
\end{split}
\end{equation}
To satisfy the initial condition $y(1)=2$, we have
\begin{equation}
   y(1) = \frac{1}{3} + C = 2 \implies C = \frac{5}{3}.
\end{equation}
Thus the solution to the initial value problem is
\begin{equation}
   y(t) = \frac{t^2}{3} + \frac{5}{3t}.
\end{equation}


\paragraph{Example 3.}
Consider the differential equation
\begin{equation}
    y' = 2ty + 3.
\end{equation}
The differential equation is linear, with p(t) = 2t and g(t)=3.
The integrating factor is
\begin{equation}
   \mu(t) = e^{-\int p(t)\,dt} = e^{-\int 2t\, dt}
      = e^{-t^2},
\end{equation}
and the solution is
\begin{equation}
\begin{split}
   y(t) & = e^{t^2} \left( \int \left(e^{-t^2}\right)\left(3\right)\,dt + C\right) \\
        & = e^{t^2} \left( 3\int e^{-t^2}\,dt+C\right)
\end{split}
\end{equation}
There is no analytical expression for the integral $\int e^{-t^2}\, dt$,
so this is the best we can do.%
\footnote{%
Because the integral of $e^{-t^2}$ appears frequently in mathematics,
it has been given a name.
You may have seen the \emph{error function}
$\textrm{erf}(x) = \frac{2}{\sqrt{\pi}}\int_0^x e^{-s^2}\,ds$.}




\section{Systems of Differential Equations}
\subsection{Phase Plane Analysis}
\subsection{Equilibrium Points and Stability}
... The Jacobian\index{Jacobian} is ...
\subsection{Higher Dimensional Systems}
%
\chapter[Dimensional Analysis]{Dimensional Analysis and Nondimensionalization}
\section{Dimensions and Units}
\section{Nondimensional Equations and Parameters}
\section{Exercises}
%
\chapter{Applications of Differential Equations}
\section{Economics}
\subsection{Solow Growth Model}
\section{Population Models}
\subsection{Predator-Prey\index{predator-prey}}
\subsection{Competing Species\index{competing species}}
\section{War and Peace}
\subsection{Battle of Attrition}
\subsection{Arms Race}
\section{Love and Marriage}
\subsection{Romeo and Juliet\index{Romeo}\index{Juliet}}
\subsection{Laura and Petrarch\index{Laura}\index{Petrarch}}
\section{Epidemiology}
\subsection{The SI Model\index{SI model}}
\subsection{The SIS Model\index{SIS model}}
\subsection{The SIR Model\index{SIR model}}
\subsection{The SIQR Model\index{SIQR model}}
\section{Exercises}

%
\chapter{Discrete Maps}
\section{Modeling with Discrete Maps}
\section{One-Dimensional Linear Maps}
\section{One-Dimensional Nonlinear Maps}
\subsection{Cobwebbing}
\section{Higher Dimensional Maps}
\section{Applications}
\subsection{Marital Relationships}
The model discussed in this section is based on the
work of Gottman, Murray, \emph{et al}\cite{GM}.
\section{Exercises}
\chapter{Hybrid Models}
\section{Introduction}
Often, models of dynamical systems lead to equations that combine
both differential equations and discrete maps.
\subsection{Periodic Drug Doses}
\section{Exercises}
%
\chapter{Markov Chains}
\section{A Brief Introduction to Probability}
\section{Markov Chains}
\subsection{An Example: The Coin and Die Game}

In this game there are two players, \emph{Coin}
and \emph{Die}. \emph{Coin} has a coin, and \emph{Die} has a
single six-sided die.

\begin{itemize}
\item
When it is \emph{Coin}'s turn, he or she flips the coin.
If the coin turns up \textbf{heads}, \emph{Coin} wins the game.
If the coin turns up \textbf{tails}, it is \emph{Die}'s turn.

\item
When it is \emph{Die}'s turn, he or she rolls the die.
If the die turns up \textbf{1}, \emph{Die} wins.
If the die turns up \textbf{6}, it is \emph{Coin}'s turn.
Otherwise, \emph{Die} rolls again.
\end{itemize}

\paragraph{Class Exercise.}
With a partner, play the game 20 times, with \emph{Coin}
going first each time.
Keep track of who wins each game, and the total
number of coin flips and rolls of the die
that takes place in each game.

Then play 20 more games, but with \emph{Die} going first
each time, and keep track of the same information.

\section{Monopoly (tm)}
\section{Baseball}
\section{Exercises}
%
\appendix
%
\chapter{Preliminary Mathematical Topics}
\section{Complex Numbers}
%
\chapter{A Brief Review of Linear Algebra}
\section{Matrices}
\section{Eigenvalues and Eigenvectors\index{eigenvalue}\index{eigenvector}}
\subsection{Shortcuts for $2\times 2$ Matrices}
In Linear Algebra, you learned how to find the eigenvalues
and eigenvectors of matrices.  In this section, we give
some shortcuts for $2\times 2$ matrices.

Let
\[
   A = \begin{bmatrix}
              a & b \\ c & d
       \end{bmatrix}.
\]
To find the eigenvalues of $A$, we must solve
$\det(A-\lambda I)=0$ for $\lambda$.
(The expression $\det(A-\lambda I)$ is called
the \emph{characteristic polynomial}.)  We have
\[
\begin{split}
   \det(A-\lambda I) & = (a-\lambda)(d-\lambda)-bc \\
                     & = \lambda^2-(a+d)\lambda + (ad-bc) \\
		     & = \lambda^2 - \textrm{Tr(A)}\lambda + \det(A)
\end{split}
\]
where $\textrm{Tr(A)} = a+d$ is the \emph{trace} of $A$.
(The trace of a square matrix is the sum of the diagonal elements.)
Then the eigenvalues are found by using the quadratic
formula, as usual.

Now consider the problem of finding the eigenvectors
for the eigenvalues $\lambda_1$ and $\lambda_2$.
An eigenvector associated with $\lambda_1$ is a nontrivial
solution $\BV_1$ to
\begin{equation}
    (A-\lambda_1 I)\BV = \BZ.
\label{eqn:eigvec}
\end{equation}
Now
\[
   A - \lambda_1 I = \begin{bmatrix}
                           a-\lambda_1 & b \\
			   c & d-\lambda_1
                     \end{bmatrix}
\]
The matrix $A-\lambda_1 I$ \emph{must} be singular.
That is precisely what makes $\lambda_1$ an eigenvalue.
If a $2\times 2$ matrix is singular, the second
row \emph{must} be a multiple of the first row (unless
the first row is zero).  Therefore, we know that putting
$A-\lambda_1 I$ into row echelon form must result in
a row of zeros.  Since we know this must be the case,
there is no need to actually do it!  All we need to
find an eigenvector is the first row.
In particular, if $\BV = [v_1,v_2]^{\textsf{T}}$,
then \eqref{eqn:eigvec} implies
\begin{equation}
  (a-\lambda_1)v_1 + b v_2 = 0.
\label{eqn:eigveceqn}
\end{equation}
We could solve this for, say, $v_2$ in terms of $v_1$,
and give all the possible eigenvectors in terms of
the arbitrary parameter $v_1$. (This is the
\emph{eigenspace} associated with the eigenvalue $\lambda_1$.)
However, for the
purpose of solving a system of differential equations,
all we need is \emph{one} eigenvector.
An easy solution to \eqref{eqn:eigveceqn}
is $v_1=-b$ and $v_2 = (a-\lambda_1)$.
Thus (unless $(a-\lambda_1)$ and $b$ both happen to be
zero), once we write down the matrix $A-\lambda_1 I$,
we can immediately get the eigenvector
\[
   \BV_1 = \begin{bmatrix} -b \\ a-\lambda_1 \end{bmatrix}
\]
If both $(a-\lambda_1)$ and $b$ are zero, we can use the
second row to find an eigenvector:
\[
   \BV_1 = \begin{bmatrix} d-\lambda_1 \\ -c \end{bmatrix}.
\]
So, once we have an eigenvalue
of a $2\times 2$ matrix, it is very easy to find
a corresponding eigenvector.
This works even when the eigenvalue is complex.
It will give a correct complex eigenvector.

\paragraph{Example 1.}
\[
   A = \begin{bmatrix} 1 & 2 \\ 3 & -4 \end{bmatrix}
\]
The characteristic polynomial is
\[
   \lambda^2 - (1+(-4))\lambda + ((1)(-4)-(2)(3)) = \lambda^2 + 3\lambda - 10,
\]
so we find
\[
  \lambda = \frac{-3\pm\sqrt{9-4(-10)}}{2} = -5, 2.
\]
Let $\lambda_1 = -5$ and $\lambda_2 = 2$.
Now we'll find an eigenvector for each eigenvalue.

\medskip
\noindent
\underline{$\lambda_1 = -5$}
\[
   A-\lambda_1 I = \begin{bmatrix}
                   6 & 2 \\ 3 & 1
                   \end{bmatrix}
\]
As expected, we see that the second row
is a multiple of the first. Using the shortcut discussed
above, we can immediately find one eigenvector to be
\[
   \BV_1 = \begin{bmatrix} -2 \\ 6 \end{bmatrix}
\]
Of course, since any nonzero multiple of an eigenvector
is also an eigenvector, we could also choose
\[
   \BV_1 = \begin{bmatrix} -1 \\ 3 \end{bmatrix}
\]

\medskip
\noindent
\underline{$\lambda_2 = 2$}
\[
  A - \lambda_2 I = \begin{bmatrix}
                      -1 & 2 \\ 3 & -6
                    \end{bmatrix}
\]
In this case, a possible eigenvector is
\[
  \BV_2 = \begin{bmatrix} -2 \\ -1 \end{bmatrix}
\]
or, if we want to minimize the number of minus signs,
\[
  \BV_2 = \begin{bmatrix} 2 \\ 1 \end{bmatrix}
\]

\paragraph{Example 2.}
\[
   A = \begin{bmatrix} -1 & -3 \\ 4 & 3 \end{bmatrix}
\]
The charateristic polynomial is
\[
  \lambda^2 - 2\lambda + 9,
\]
and the eigenvalues are
\[
  \lambda = \frac{2\pm \sqrt{4-36}}{2} = 1\pm 2\sqrt{-2}
    = 1 \pm 2 \sqrt{2} \, i
\]
Let $\lambda_1 = 1 + 2\sqrt{2}\, i$, and $\lambda_2 = \lambda_1^{*}$.
We'll find an eigenvector associated with
the eigenvalue $\lambda_1$.

We have
\[
   A - \lambda_1 I = \begin{bmatrix}
                        -1-(1+2\sqrt{2}\,i) & -3 \\
			4 & 3-(1+2\sqrt{2}\,i)
                     \end{bmatrix}
		   = \begin{bmatrix}
		        -2-2\sqrt{2}\, i & -3 \\
			4 & 2-2\sqrt{2}\,i
		     \end{bmatrix}
\]
By using the shortcut discussed above, we can
immediately write down the eigenvector
\[
  \BV_1 = \begin{bmatrix} 3 \\ -2-2\sqrt{2}\, i \end{bmatrix}
\]
(If we were solving a system of differential equations, we would
then want to express $\BV_1$ as
\[
   \BV_1 = \begin{bmatrix} 3 \\ -2 \end{bmatrix}
           + i \begin{bmatrix} 0 \\ -2\sqrt{2} \end{bmatrix}
\]
so $\BA = \begin{bmatrix} 3 \\ -2\end{bmatrix}$
and $\BB = \begin{bmatrix} 0 \\ -2\sqrt{2} \end{bmatrix}$.)
\chapter[Computer Tools]{Computer Tools for Computation and Graphics}
%
% Do Octave, Scilab, Maxima, and Yacas require the trademark symbol?
%
In this appendix I give a brief introduction to some
readily available software packages that can be used to solve
differential equations and discrete maps.  MATLAB (tm)
is a widely used commercial package.  Octave (tm) and
Scilab (tm) are free open source software packages. 
\section{MATLAB (tm)}
\section{Octave (tm)}
\section{Scilab (tm)}
\chapter{Symbolic Computing}
The computer can be a powerful tool for performing
algebraic computations.  In this appendix I give a brief introduction
to a few software packages that can do symbolic computations.
Maple (tm) and Mathematica (tm) are widely used commercial
packages.  Maxima (tm) and Yacas (tm) are free open-source
programs.
\section{Maple (tm)}
\section{Mathematica (tm)}
\section{Maxima (tm)}
\section{Yacas (tm)}
%
\begin{thebibliography}{99}
\bibitem{GM} Gottman, J. Murray, ...
\end{thebibliography}
\printindex
\end{document}