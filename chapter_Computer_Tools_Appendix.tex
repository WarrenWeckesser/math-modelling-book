\chapter{Computer Tools}
%
% Do Octave, Scilab, Maxima, and Yacas require the trademark symbol?
%
In this appendix I give a brief introduction to some
readily available software packages that can be used to solve
differential equations and discrete maps.  MATLAB (tm)
is a widely used commercial package.
Scilab (tm)\cite{SCILAB} is a similar program that is free.
(Another program to consider is Octave.)
Python is a program language... etc...

The notes provide instructions for the following tasks:
\begin{enumerate}
\item Solve a system of algebraic equations.
\item Find the eigenvalues and eigenvectors of a matrix.
\item Solve a system of differential equations.
\item Solve a $n$-dimensional map.
\item Analyze a large Markov chain numerically; note that
the software packages (well, at least Scilab) have
some predefined functions for Markov chains.
\end{enumerate}
\section{MATLAB (tm)}

\subsection*{Example: Find the eigenvalues of a matrix.}
The \texttt{eig} function is used to find the eigenvalues
of a matrix in MATLAB;
Figure \ref{fig:matlabeigenvaluesoutput} shows an example.
The matrix in this example is
\[
   A = \begin{bmatrix}
          0.6 & 0.2 & 0.2 \\
	  0.1 & 0.2 & 0.7 \\
	  0.4 & 0.0 & 0.6 \\
       \end{bmatrix} .
\]
The computation shows that the eigenvalues of the 
matrix are $1$ and the complex conjugate pair
$0.2 \pm  0.24494897i$
(approximately).

\begin{figure}
\fbox{%
\input{matlab/eigenvalues.matlab.listing}
}
\caption{An example of the use of
the \texttt{eig} function in MATLAB.
(By default, MATLAB prints numbers with four
decimal places.  The command \texttt{\textbf{format long}}
increases the number of decimal places
printed to 14.)}
\label{fig:matlabeigenvaluesoutput}
\end{figure}

\subsection*{Example: Solving the SIR Model with MATLAB.}
Figures \ref{fig:matlabSIRvflisting} and
\ref{fig:matlabSIRscriptlisting} show MATLAB code that solves
the SIR model \eqref{eqn:SIR}.
To execute this example, put the files in your MATLAB working
directory, and in MATLAB give the command
\begin{verbatim}
>> SIRscript
\end{verbatim}
The plot created by this code is shown
in Figure \ref{fig:matlabSIRscriptoutput}.

%
% Note: The listings were created with mylister.
%
\begin{figure}
\fbox{%
\input{matlab/SIRvectorfield.m.listing}
}
\caption{Listing of \texttt{SIRvectorfield.m}.}
\label{fig:matlabSIRvflisting}
\end{figure}

\begin{figure}
\fbox{%
\input{matlab/SIRscript.m.listing}
}
\caption{Listing of \texttt{SIRscript.m}.}
\label{fig:matlabSIRscriptlisting}
\end{figure}

\begin{figure}
\centerline{\includegraphics{matlab/SIRscript.plot.eps}}
\caption{Output generated by the MATLAB program
\texttt{SIRscript.m}.}
\label{fig:matlabSIRscriptoutput}
\end{figure}
%
%
\newpage
\section{Scilab (tm)}

\subsection*{Example: Find the eigenvalues of a matrix.}
The \texttt{spec} function is used to find the eigenvalues
of a matrix in Scilab. (The set of all eigenvalues of a matrix
is often called the \emph{spectrum}\index{spectrum} of a matrix,
which is where the name of this function comes from.)
Figure \ref{fig:scilabeigenvaluesoutput} shows an example
of the use of the \texttt{spec} function.
The matrix in this example is
\[
   A = \begin{bmatrix}
          0.6 & 0.2 & 0.2 \\
	  0.1 & 0.2 & 0.7 \\
	  0.4 & 0.0 & 0.6 \\
       \end{bmatrix} .
\]
The computation shows that the eigenvalues of the 
matrix are $1$ and the complex conjugate pair
$0.2 \pm  0.2449490i$
(approximately).

\begin{figure}[hbp]
\fbox{%
\input{scilab/eigenvalues.scilab.listing}
}
\caption{An example of the use of
the \texttt{spec} function in Scilab.}
\label{fig:scilabeigenvaluesoutput}
\end{figure}


\subsection*{Example: Solving the SIR Model with Scilab.}
Figures \ref{fig:scilabSIRvflisting} and
\ref{fig:scilabSIRscriptlisting} show Scilab code that solves
the SIR model \eqref{eqn:SIR}.
To execute this example, put the files in your Scilab working
directory, and in Scilab give the command
\begin{verbatim}
-->exec('SIRscript.sce');
\end{verbatim}
The plot created by this code is shown
in Figure \ref{fig:scilabSIRscriptoutput}.

%
% Note: The listings were created with
% $ ./mylister  scilab/SIRvectorfield.sce > scilab/SIRvectorfield.sce.listing.tex
%
\begin{figure}
\fbox{%
\input{scilab/SIRvectorfield.sci.listing}
}
\caption{Listing of \texttt{SIRvectorfield.sci}.}
\label{fig:scilabSIRvflisting}
\end{figure}

\begin{figure}
\fbox{%
\input{scilab/SIRscript.sce.listing}
}
\caption{Listing of \texttt{SIRscript.sce}.}
\label{fig:scilabSIRscriptlisting}
\end{figure}

\begin{figure}
\centerline{\includegraphics[width=5in]{scilab/SIR.eps}}
\caption{Output generated by the Scilab program
\texttt{SIRscript.sce}.}
\label{fig:scilabSIRscriptoutput}
\end{figure}
%
%
\section{Python}
\subsection*{Example: Find the eigenvalues of a matrix.}
The \texttt{eig} function from the scipy.linalg library 
is used to find the eigenvalues
of a matrix A.
Figure \ref{fig:pythoneigenvaluesoutput} shows an example.
The matrix in this example is
\[
   A = \begin{bmatrix}
          0.6 & 0.2 & 0.2 \\
	  0.1 & 0.2 & 0.7 \\
	  0.4 & 0.0 & 0.6 \\
       \end{bmatrix} .
\]
The computation shows that the eigenvalues of the 
matrix are $1$ and the complex conjugate pair
$0.2 \pm  0.24494897i$
(approximately).

\begin{figure}[hbp]
\fbox{%
\input{python/python_example.eigenvalues.listing}
}
\caption{An example of the use of
the \texttt{eig} function in Python.}
\label{fig:pythoneigenvaluesoutput}
\end{figure}

\subsection*{Example: Solving the SIR Model with Python.}
Figures \ref{fig:pythonSIRvflisting} and
\ref{fig:pythonSIRscriptlisting} show Scilab code that solves
the SIR model \eqref{eqn:SIR}.
To execute this example, put the files in your working
directory, and give the command
\begin{verbatim}
  python SIRscript.py
\end{verbatim}
The plot created by this code is shown
in Figure \ref{fig:pythonSIRscriptoutput}.

%
%
\begin{figure}
\fbox{%
\input{python/SIRvectorfield.py.listing}
}
\caption{Listing of \texttt{SIRvectorfield.py}.}
\label{fig:pythonSIRvflisting}
\end{figure}

\begin{figure}
\fbox{%
\input{python/SIRscript.py.listing}
}
\caption{Listing of \texttt{SIRscript.py}.}
\label{fig:pythonSIRscriptlisting}
\end{figure}

\begin{figure}
\centerline{\includegraphics{python/SIRscript_output.eps}}
\caption{Output generated by the Python program
\texttt{SIRscript.py}.}
\label{fig:pythonSIRscriptoutput}
\end{figure}
