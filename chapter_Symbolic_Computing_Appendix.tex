\chapter{Symbolic Computing}
The computer can be a powerful tool for performing
algebraic computations.  In this appendix I give a brief introduction
to a few software packages that can do symbolic computations.
Maple (tm) is a widely used commercial
package, and Maxima is a free open-source
program.
(Other widely used programs that might also be useful
are Mathematica (tm), Axiom or Yacas.)
\section{Maple (tm)}
\textbf{Solving first order differential equations.}
\begin{xexample}
We'll solve
\[
   y' = -y + 3
\]
\end{xexample}
\section{Maxima (tm)}
\textbf{Solving first order differential equations.}
\begin{xexample}
We'll solve
\[
   y' = -y + 3
\]
and then we'll require that the solution satisfy the
initial condition $y(0)=1$.
The Maxima code in Figure~\ref{fig:maxima_example_firstorderde} shows how this is done.
In Maxima output, the number $e$ is written \texttt{\%E},
and the arbitrary constant in the general solution of
the first order differential equation is written \texttt{\%C}.
Lines 6 and 7 of Figure~\ref{fig:maxima_example_firstorderde} show that the
solution to the equation is
\[
    y(t) = e^{-t}\left(2 e^{t} + C\right),
\]
which we can rewrite as
\[
    y(t) = 2 + C e^{-t}.
\]
\begin{figure}[hbp]
\fbox{%
\input{maxima/maxima_example.firstorderde}
}
\caption{Solving a first order differential
equation in Maxima.}
\label{fig:maxima_example_firstorderde}
\end{figure}
\end{xexample}
