
\chapter{Applications of Differential Equations}
%
In this chapter, we developed models using differential
equations for a wide variety of systems.
We include problems from a wide variety of fields:
\begin{itemize}
\item \emph{Economics}: Solow growth model
\item \emph{Population Dynamics}: predator-prey, competing species
\item \emph{War and Peace}: Lanchester battle model, Richardson's arms
race model
\item \emph{Love Affairs}: a variation of Romeo and Juliet, Rinaldi's
model of Petrarch's love for Laura
\item \emph{Epidemiology}: the SI, SIR, and SIQR models
\end{itemize}
%
\newpage
%
\section{Solow's Economic Growth Model}
\index{Solow growth model}
% \emph{(Draft version\footnote{\copyright ~2006 Warren Weckesser}.)}

%\medskip
We consider a model from macroeconomics.\index{economics}
Let $K$ be the capital,%
\footnote{\emph{Capital} includes things that are owned to be used
in production, such as buildings and manufacturing equipment.}%
~$L$ the labor, and $Q$ the production output of an economy.
We are interested in a \emph{dynamic} problem, so $K(t)$, $L(t)$ and
$Q(t)$ are all functions of time, but we will suppress the $t$ argument.
In elementary economics, one learns that a common assumption is that
$Q$ can be expressed as function of $K$ and $L$:
\begin{equation}
   Q = f(K,L).
   \label{EQN:PROD}
\end{equation}
%We make the reasonable assumptions that $f_K > 0$ and $f_L > 0$.
%(The subscript denotes a partial derivative: $f_K = \partial f/\partial K$.)
%These assumptions mean that $Q$ increases if either $K$ or $L$ increases.
%That is, with more capital or more labor, we can produce more.
%We also assume that $f_{KK} < 0$ and $f_{LL} < 0$.
%These assumptions say that $f$ has \emph{diminishing returns} to the
%inputs $K$ and $L$.  In other words, the larger $K$ is, the less is the
%effect of increasing $K$, and the same holds for $L$.
We assume that $f$ has, using economics terminology,
\emph{constant returns to scale}.\index{constant returns to scale}  Mathematically, this means
that multiplying $K$ and $L$ by the same amount results in $Q$ being
multiplied by the same amount.  That is, for any constant $b$,
\begin{equation}
   f(bK,bL) = bf(K,L).
\end{equation}
For example, the Cobb-Douglas function\index{Cobb-Douglas function}
$f(K,L) = K^{1/3}L^{2/3}$
satisfies this assumption.

We make two more assumptions.
We assume that a constant proportion of $Q$ is invested in capital.
This means that the \emph{rate of change} of $K$ is proportional to
$Q$:
\begin{equation}
    \frac{dK}{dt} = s Q,
\label{EQN:DKDT}
\end{equation}
where $s > 0$ is the proportionality constant.
We also assume that the labor force is growing according
to the equation
\begin{equation}
   \frac{dL}{dt} = \lambda L,
   \label{EQN:DLDT}
\end{equation}
where $\lambda > 0$ is the per capita growth rate.  This is a
familiar first order equation for $L$ which we can solve to find 
\begin{equation}
    L = L_0 e^{\lambda t}.
\label{eqn:solowLsol}
\end{equation}

If possible, we would like to combine \eqref{EQN:PROD},
\eqref{EQN:DKDT}, and \eqref{EQN:DLDT} (or \eqref{eqn:solowLsol})
into a single
equation that we may easily analyze. A natural first
attempt is to substitute \eqref{EQN:PROD} into
\eqref{EQN:DKDT} and use \eqref{eqn:solowLsol} to obtain
\begin{equation}
  \frac{dK}{dt} = sf(K,L_0e^{\lambda t})
\label{eqn:solowfirstattempt}
\end{equation}
This is a first order differential equation for $K(t)$.
It is, however, nonautonomous; the right side depends on $t$ explicitly.
We could try to analyze this equation, but it would
be nice if we could find an \emph{autonomous} first order
differential equation.  It turns out we can derive an
autonomous equation for 
the \emph{ratio} $\frac{K}{L}$ instead of $K$.

First, because $f$ has constant returns to scale,
we may write
\begin{equation}
  f(K,L) = f\left(L\frac{K}{L},L\right) = Lf\left(\frac{K}{L},1\right).
\end{equation}
Then, after dividing by $L$, \eqref{eqn:solowfirstattempt} becomes
\begin{equation}
  \frac{1}{L}\frac{dK}{dt} = sf\left(\frac{K}{L},1\right)
\label{eqn:solowintermediate}
\end{equation}
Next, we consider the
derivative of $\frac{K}{L}$ given by the quotient rule,
and we use \eqref{EQN:DLDT}:
\begin{equation}
  \frac{d}{dt}\left(\frac{K}{L}\right)
    = \frac{1}{L}\frac{dK}{dt} - \frac{K}{L^2}\frac{dL}{dt}
    = \frac{1}{L}\frac{dK}{dt} - \lambda\frac{K}{L}.
\end{equation}
If we subtract $\lambda\frac{K}{L}$ from both sides of
\eqref{eqn:solowintermediate}, the left side becomes
$\frac{d}{dt}\left(\frac{K}{L}\right)$, and we obtain
\begin{equation}
  \frac{d}{dt}\left(\frac{K}{L}\right) =
    sf\left(\frac{K}{L},1\right) - \lambda\frac{K}{L}
\label{eqn:solowKoverL}
\end{equation}
We now have an equation in which the unknown function
is $\frac{K}{L}$.  Let us define
\begin{equation}
   k = \frac{K}{L}
\label{eqn:solowdefk}
\end{equation}
and
\begin{equation}
    g(k) = f(k,1).
\label{eqn:solowdefg}
\end{equation}
Then \eqref{eqn:solowKoverL} becomes
\begin{equation}
  \frac{dk}{dt} = s g(k) - \lambda k
\end{equation}
This is the \emph{Solow Growth Model} \cite{Solow}\index{Solow growth model}
which models the growth of the ratio of capital to labor
under the assumptions given earlier.

\medskip
\fbox{%
\begin{minipage}{4.7in}
\medskip
\centerline{\textbf{Summary}}
\medskip
\emph{Assumptions}
\\[3pt]
\hspace*{0.15cm}(1) $Q=f(K,L)$ where $f(K,L)$ is a function
with constant returns to scale.
\\[4pt]
\hspace*{0.15cm}(2) $\ds \frac{dK}{dt} = sQ$; a fraction of the production output
is invested in capital.
\\[4pt]
\hspace*{0.15cm}(3) $\ds \frac{dL}{dt} = \lambda L$; labor grows according to this
equation.
\\[6pt]
\emph{Definitions}
\\[3pt]
\hspace*{0.25cm}$\bullet$ $\ds k = \frac{K}{L}$; we analyze the \emph{ratio} of capital to labor.
\\[3pt]
\hspace*{0.25cm}$\bullet$ $g(k) = f(k,1)$.
\\[6pt]
\emph{Result}\\
\centerline{$\ds \frac{dk}{dt} = s g(k) - \lambda k$}
\medskip
\end{minipage}
}

\medskip
\begin{xexample}
\label{exm:solow}
As an example, let's take the production function to be
\begin{equation}
   f(K,L) = K^{1/3}L^{2/3}.
\label{eqn:cobbexample}
\end{equation}
Then
\begin{equation}
g(k) = f(k,1) = k^{1/3},
\end{equation}
 and the differential equation for
$k$ is
\begin{equation}
   \frac{dk}{dt} = sk^{1/3} - \lambda k .
\label{eqn:solowcobb}
\end{equation}
Figure~\ref{fig:solow_rhsplot} shows the graph of
$\frac{dk}{dt}$ versus $k$.
\begin{figure}
\centerline{%
\includegraphics[width=2.75in]{python/solow_rhsplot.eps}
}
\caption{The graph of the right side of equation \eqref{eqn:solowcobb}.}
\label{fig:solow_rhsplot}
\end{figure}

By solving
\[
   sk^{1/3} - \lambda k = 0,
\]
we find the equilibrium solutions to be $k=0$ or $k=(s/\lambda)^{3/2}$.


Changing $\lambda$ or $s$ will change the scale
(and the numerical value of the non-zero equilibrium),
but the graph of $dk/dt$ versus $k$ will always have the
same qualitative shape as the graph shown above.

We see that if $k>0$ is small,
$\frac{dk}{dt} > 0$, so $k$
will increase; the equilibrium $k=0$ is \emph{unstable}.
The graph of $k(t)$ will have an inflection point when $k$ reaches
$\left(\frac{s}{3\lambda}\right)^{3/2}$
(where right side of \eqref{eqn:solowcobb} has its maximum).
$k$ will then converge asymptotically to the non-zero equilibrium.

The equilibrium $k=(s/\lambda)^{3/2}$ is \emph{asymptotically stable}:
any solution that starts near the equilibrium will converge to the equilibrium
as $t\rightarrow \infty$.
In fact, \emph{all} solutions with $k(0)>0$ will converge asymptotically to this
equilibrium.

What does this mean in terms of the capital $K$ and the labor $L$?
Since $k(t) = K(t)/L(t)$, and $L(t) = L_0e^{\lambda t}$, if $k(t)$ converges to an
asymptotically stable equilibrium $k_1$, then $K(t)$ must behave asymptotically like
$k_1  L(t)$.  This means that, in the long term, $K(t)$ must 
grow exponentially, with the
same exponent as $L(t)$.
This model predicts that in the long term, capital will grow
exponentially along with the labor.
If, for example, the capital is too low, it will rapidly increase until it becomes
approximately proportional to the labor, and then it will settle into a long term behavior in
which capital remains proportional to the labor.
\end{xexample}
%
\newpage
%
\begin{exercises}
\begin{exercise}
Verify that the maximum value of the right side
of \eqref{eqn:solowcobb} occurs
at $k=\left(\frac{s}{3\lambda}\right)^{3/2}$.
\end{exercise}
\begin{exercise}
Find an explicit solution for \eqref{eqn:solowcobb},
assuming that the initial condition is $k(0)=k_0 > 0$.
Use your solution to verify analytically that
$\lim_{t\rightarrow\infty} k(t) = \left(s/\lambda\right)^{3/2}$.
\end{exercise}
\begin{exercise}
Suppose that labor grows according to a logistic equation
\begin{equation}
  \frac{dL}{dt} = \lambda L \left(1-\frac{L}{M}\right)
\end{equation}
where $M$ is the carrying capacity for the labor population.
Derive a new differential equation for $k(t)$.
Is your new equation autonomous?
If not, can you still determine the asymptotic
behavior of the solutions as $t\rightarrow\infty$?
\end{exercise}
\begin{exercise}
Suppose we include the fact that capital deteriorates
over time.  We replace the assumption given
in equation \eqref{EQN:DKDT} with
\begin{equation}
   \frac{dK}{dt} = -rK + sQ
\end{equation}
where $r > 0$.
(This says that, if $Q=0$, then $K$ will decay and
approach zero asymptotically.)
\begin{enumerate}
\item[(a)]
Show that this leads to the differential equation
\begin{equation}
   \frac{dk}{dt} = s g(k) - (r+\lambda) k,
\end{equation}
where $k$ and $g(k)$ are defined
in \eqref{eqn:solowdefk} and \eqref{eqn:solowdefg}, 
respectively.
\item[(b)]
Suppose the production function is the Cobb-Douglas function
given in equation \eqref{eqn:cobbexample}.
Find the equilibrium solutions of this model, 
and describe the behavior of all possible solutions
(assuming $k(0)>0$).
Compare to the results given in Example \ref{exm:solow}.
\end{enumerate}
\end{exercise}
\begin{exercise}
Find an explicit solution to equation
\eqref{eqn:solowfirstattempt} when $f$ is given by
\eqref{eqn:cobbexample}.  (The equation will be separable.)
Then divide your solution by $L$ to obtain $k(t)$,
and verify that $\lim_{t\rightarrow\infty} k(t) =
\left(s/\lambda\right)^{3/2}$.
\end{exercise}
\begin{exercise}
Consider a general Cobb-Douglas function with
constant returns to scale
\begin{equation}
   f(K,L) = K^{\alpha}L^{1-\alpha}
\end{equation}
where $0 < \alpha < 1$.
Repeat the analysis of Example \ref{exm:solow},
but use this function in place of the
specific function given in \eqref{eqn:cobbexample}.
\end{exercise}
\end{exercises}
\newpage
%
\section{Predator-Prey\index{predator-prey}}
%
\emph{Explain assumptions...}

\medskip

One version of these equations is
\begin{equation}
\begin{split}
  \frac{dx}{dt} & = -\gamma (1-\alpha y) x \\
  \frac{dy}{dt} & = \delta (1-\beta x) y
\end{split}
\label{eqn:predatorprey}
\end{equation}
where $\alpha$, $\beta$, $\gamma$ and $\delta$ are positive constants.

\medskip

The system \eqref{eqn:predatorprey} has a conserved quantity, which
we now derive.
If we consider a solution in the $(x,y)$ plane, the slope at a point
on the curve is
\begin{equation}
  \frac{dy}{dx} = \frac{\frac{dy}{dt}}{\frac{dx}{dt}}
    = \frac{\delta(1-\beta x)y}{-\gamma(1-\alpha y)x}
\end{equation}
This is a separable first order equation.  By separating, we find
\begin{equation}
  \frac{\gamma(1-\alpha y)}{y} dy = -\frac{\delta(1-\beta x)}{x} dx
\end{equation}
By integrating with respect to $y$ on the left and $x$ on the right,
we obtain
\begin{equation}
  \gamma \ln(y) - \gamma\alpha y = -\delta\ln(x) + \delta\beta x + C
\end{equation}
In other words, if $(x(t),y(t))$ is a solutions to \eqref{eqn:predatorprey},
the function
\begin{equation}
  H(x,y) = \delta\ln(x) - \delta\beta x + \gamma\ln(y) - \gamma\alpha y
\end{equation}
remains constant.
\medskip

\newpage
%
\section{Competing Species\index{competing species}}
%
We consider an example that models the populations
of two species that are competing for a common resource.
In the absence of the other species, each species
grows according to a logistic equation.
However, the presence of one species lowers
the \emph{per capita} growth rate of the other species.
One way to write the equations for this system
is
\begin{equation}
\begin{split}
  \frac{dx}{dt} & = r_1\left(1-\frac{x}{K_1}-\beta_1 y\right)x \\
  \frac{dy}{dt} & = r_2\left(1-\frac{y}{K_2}-\beta_2 x\right)y
\end{split}
\end{equation}
Note that if $y(0)=0$, then $y(t)$ remains $0$, and
the equation for $x(t)$ is
\begin{equation}
    \frac{dx}{dt} = r_1 \left(1-\frac{x}{K_1}\right),
\end{equation}
which is the familiar logistic equation.
Similarly, if $x(0)=0$, then $x(t)$ remains $0$ and
$y(t)$ is governed by a logistic equation.

Let's do a careful analysis of a specific example,
in which $r_1 = 1$, $K_1 = 1$, $\beta_1 = 1$, 
$r_2 = 3/4$, $K_2 = 3/4$, and $\beta_2 = 2/3$.
The differential equations are
\begin{equation}
\begin{split}
  \frac{dx}{dt} & = (1-x-y)x \\
  \frac{dy}{dt} & = \frac{3}{4}\left(1 -\frac{4}{3}y - \frac{2}{3}x\right)y .
\end{split}
\label{eqn:compspecexample}
\end{equation}
We'll find the equilibria, find the linearization at each
equilibrium to determine the behavior near each one, and then
use the nullclines of~\eqref{eqn:compspecexample} to understand
what happens in the phase plane ``far away'' from the equilibria.

\noindent
\textbf{Equilibria.}
To find the equilibria, we must solve
\begin{equation}
\begin{split}
(1-x-y)x & = 0 \\
\frac{3}{4}\left(1 -\frac{4}{3}y - \frac{2}{3}x\right)y & = 0.
\end{split}
\label{eqn:exampleequil}
\end{equation}
The first equation holds if $x=0$ or $y = 1-x$.
We consider each case separately in the second equation.
\begin{itemize}
\item
If $x=0$, then the second equation of~\eqref{eqn:exampleequil} implies
$y=0$ or $y=\frac{3}{4}$.

So two equilibria are $(0,0)$ and $(0,3/4)$.
\item
If $y=1-x$, then the second equation
of~\eqref{eqn:exampleequil} implies
\begin{equation}
  \left(1-\frac{4}{3}(1-x) - \frac{2}{3}x\right)(1-x) = 0
  \implies x=1 \quad \textrm{or} \quad x=\frac{1}{2}.
\end{equation}
So two equilibria are $(1,0)$ and $(1/2,1/2)$.
\end{itemize}

\noindent
\textbf{Linearization at each equilibrium.}
We have found the following equilibria: $(0,0)$, $(0,3/4)$, $(1,0)$, $(1/2,1/2)$.
We now determine the behavior of~\eqref{eqn:compspecexample}
near each equilibrium by finding the linearization at each
equilibrium.
We will need the Jacobian matrix:
\begin{equation}
  J = \begin{bmatrix}
          1-2x-y & -x \\
	  -\frac{1}{2}y & \frac{3}{4}-2y-\frac{1}{2}x
      \end{bmatrix}
\end{equation}
For each equilibrium, we will find the Jacobian matrix
and plot the phase portrait of the linearization.
We make two remarks about the phase portraits of the linearized
systems:
\begin{enumerate}
\item
Recall from Section~\ref{sec:DELinearization} that we
used the local coordinates $(u,v)$ for the linearization.
In a linear system, the scale of the coordinates is not
important: if you zoom in on the origin of a linear system,
the phase portrait will look exactly the same.
So, in the following phase portraits of the linearizations,
the ranges on the axis are  from $-1$ to $1$.  These
are \emph{not} the actual $x$ and $y$ ranges.
\item
In a population model such as this, $x<0$ and $y<0$
are not relevant.  However, we will still plot negative
values in the linearization.  It seems easier to simply
plot the linear phase portrait, ignoring the actual
meaning until later.  Also, recall that the linearized
system is expressed in coordinates measured relative
to the equilibrium.  If a coordinate of the equilibrium
is positive, then a negative value of the corresponding
local coordinates means that the population is less than
the equilibrium value. It does not necessarily mean that
the actual population is negative.
\end{enumerate}

\medskip

At $(0,0)$, the Jacobian matrix is
\begin{equation}
  J = \begin{bmatrix}
          1 & 0 \\
	  0 & \frac{3}{4} \\
      \end{bmatrix} .
\end{equation}
The eigenvalues are $\lambda_1 = 3/4$ and $\lambda_2 = 1$, with
corresponding eigenvectors
\[
  \BV_1 = \begin{bmatrix} 0 \\ 1 \end{bmatrix}
  \quad \textrm{and}\quad 
  \BV_2 = \begin{bmatrix} 1 \\ 0 \end{bmatrix}.
\]
Since $\lambda_1 > 0$ and $\lambda_2 > 0$,
the equilibrium $(0,0)$ is a
\emph{source}.  The trajectories come out of $(0,0)$
tangent to the eigenvector $\BV_1$.
The phase portrait of the linearization at $(0,0)$ is
shown in Figure~\ref{fig:CompSpecLinPlots}(a).
%\centerline{%
%\includegraphics{matlab/LinCS1.eps}
%}
%\noindent
If we were to ``zoom in'' on the point $(0,0)$
in~\eqref{eqn:compspecexample}, this is what the 
phase portrait would look like.

%\medskip

At $(1,0)$, the Jacobian matrix is
\begin{equation}
  J = \begin{bmatrix}
          -1 & -1 \\
	  0 & \frac{1}{2} \\
      \end{bmatrix}.
\end{equation}
The eigenvalues are $\lambda_1 = -1$ and $\lambda_2 = 1/2$,
with corresponding eigenvectors
\[
  \BV_1 = \begin{bmatrix} 1 \\ 0 \end{bmatrix}
  \quad \textrm{and} \quad
  \BV_2 = \begin{bmatrix} 1 \\ -3/2 \end{bmatrix}.
\]
Since $\lambda_1 < 0$ and $\lambda_2 > 0$, the equilibrium
$(1,0)$ is a \emph{saddle point}.
The phase portrait of the linearization at $(1,0)$ is
shown in Figure~\ref{fig:CompSpecLinPlots}(b).

%\centerline{%
%\includegraphics{matlab/LinCS2.eps}
%}

%\medskip

At $(0,3/4)$, the Jacobian matrix is
\begin{equation}
  J = \begin{bmatrix}
          \frac{1}{4} & 0 \\
	  -\frac{3}{8} & -\frac{3}{4} \\
      \end{bmatrix}.
\end{equation}
The eigenvalues are $\lambda_1 = -\frac{3}{4}$ and
$\lambda_2 = \frac{1}{4}$,
with corresponding eigenvectors
\[
  \BV_1 = \begin{bmatrix} 1 \\ -3/8 \end{bmatrix}
  \quad \textrm{and}\quad 
  \BV_2 = \begin{bmatrix} 0 \\ 1 \end{bmatrix}.
\]
Since $\lambda_1 < 0$ and $\lambda_2 > 0$,
the equilibrium $(0,3/4)$
is a \emph{saddle point}.
The phase portrait of the linearization at $(0,3/4)$ is
shown in Figure~\ref{fig:CompSpecLinPlots}(c).

%\centerline{%
%\includegraphics{matlab/LinCS3.eps}
%}

%\medskip

At $(1/2,1/2)$, the Jacobian matrix is
\begin{equation}
  J = \begin{bmatrix}
          -\frac{1}{2} & -\frac{1}{2} \\
	  -\frac{1}{4} & -\frac{1}{2} \\
      \end{bmatrix}.
\end{equation}
The eigenvalues are
$\lambda_1 = \frac{-2-\sqrt{2}}{4} \approx -0.853 < 0$
and $\lambda_2 = \frac{-2+\sqrt{2}}{4} \approx -0.147 < 0$,
with corresponding eigenvectors
\[
  \BV_1 = \begin{bmatrix} \sqrt{2} \\ 1 \end{bmatrix}
  \quad \textrm{and}\quad
  \BV_2 = \begin{bmatrix} \sqrt{2} \\ -1 \end{bmatrix}.
\]
Since both eigenvalues are negative,
the equilibrium at $(1/2,1/2)$ is a \emph{sink}.
The phase portrait of the linearization at $(1/2,1/2)$ is
shown in Figure~\ref{fig:CompSpecLinPlots}(d)

%\centerline{%
%\includegraphics{matlab/LinCS4.eps}
%}

%\medskip

\begin{figure}
\centerline{%
PLACEHOLDER
%\includegraphics[width=2.5in]{matlab/LinCS1.eps}
%\includegraphics[width=2.5in]{matlab/LinCS2.eps}
}
\vspace{-0.2in}
\centerline{\hspace{0.2in}\textbf{(a)}\hspace{2.1in}\textbf{(b)}}
\centerline{%
PLACEHOLDER
%\includegraphics[width=2.5in]{matlab/LinCS3.eps}
%\includegraphics[width=2.5in]{matlab/LinCS4.eps}
}
\vspace{-0.2in}
\centerline{\hspace{0.2in}\textbf{(c)}\hspace{2.1in}\textbf{(d)}}
\caption{Phase portraits of the linearizations at each equilibrium point
of the competing species example:
(a) at $(0,0)$; (b) at $(1,0)$; (c) at $(0,3/4)$; (d) at $(1/2,1/2)$.}
\label{fig:CompSpecLinPlots}
\end{figure}


\noindent
\textbf{Nullclines.}
Since the eigenvalues at each linearization
all had nonzero real parts, each linearization
provides a good approximation to the behavior
of~\eqref{eqn:compspecexample} near the corresponding
equilibrium point.
However, the local linearizations do not tell us what is
happening in the phase plane far from the equilibria.
In a planar system such as this, the nullclines
can provide useful information about the phase portrait.

The $x$ nullcline is given by
\begin{equation}
   (1-x-y)x = 0 \implies x=0 \quad\textrm{or}\quad y = 1-x.
\end{equation}
So $\frac{dx}{dt}=0$ on the lines\footnote{In this example, and in all
the planar linear systems that we have studied,
the nullclines are straight lines. This is not true in general;
nullclines can be curves.}
$x=0$ and $y=1-x$.

The $y$ nullcline is given by
\begin{equation}
  \frac{3}{4}\left(1-\frac{4}{3}y - \frac{2}{3} x\right)y = 0
\end{equation}
which gives the lines
\begin{equation}
  y = 0 \quad \textrm{or} \quad y = \frac{3}{4} - \frac{1}{2}x.
\end{equation}
On these lines, $\frac{dy}{dt}=0$.

The nullclines give the points in the plane where 
$\frac{dx}{dt}=0$ or $\frac{dy}{dt}=0$.
They form the boundaries of regions in the plane
where $\frac{dx}{dt}$ and $\frac{dy}{dt}$ do not change sign.
This can be very useful information.
For example, if we know that in a certain region,
$\frac{dx}{dt} > 0$ and $\frac{dy}{dt}>0$, we know that all
vectors of the vector field point ``up and to the right'';
this means all trajectories move up and to the right.
If this region is in the first quadrant, it implies that
all trajectories move away from the origin.

In the following plot, the nullclines (that are not coordinate
axes) are plotted with dashed lines, and trajectories
are plotted as solid lines.  Since our equations give 
a model for two species, we only include the first quadrant.
Negatives values would not be meaningful, so we do not include them.

\smallskip
\centerline{%
\fbox{\includegraphics[width=4.5in]{matlab/CompSpec.eps}}
}

\medskip
\noindent
In the region labeled \textsf{\textbf{A}},
$\frac{dx}{dt}<0$ and $\frac{dy}{dt}<0$.
All trajectories in this region (which extends out to infinity)
must move down and to the left.  This implies that
all these trajectories must either cross the $x$ or $y$ axes,
or cross the nullclines that form the boundaries between
\textsf{\textbf{A}} and \textsf{\textbf{B}}
and between \textsf{\textbf{A}} and \textsf{\textbf{D}}.
But the coordinate axes are themselves solutions,
so solutions that do not start on the coordinate axes
can not cross the axes.  Thus any trajectory that begins in
\textsf{\textbf{A}} must eventually cross into
\textsf{\textbf{B}} or \textsf{\textbf{D}}
(or, in one special case, converge to the equilibrium
at $(1/2,1/2)$).

Now consider trajectories in \textsf{\textbf{B}}.
In \textsf{\textbf{B}}, we have
$\frac{dx}{dt} > 0$ and $\frac{dy}{dt} < 0$.
All trajectories in this region move down and to the right.
They can not cross either of the nullclines that
form the upper and lower boundaries of the region,
because the vector field on these nullclines points
\emph{into} \textsf{\textbf{B}}.  Therefore,
\emph{all trajectories in} \textsf{\textbf{B}}
\emph{must converge to} $(1/2,1/2)$.

A similar argument applies to \textsf{\textbf{D}},
where $\frac{dx}{dt} < 0$ and $\frac{dy}{dt} >0$.
In this region, all trajectories move up and to the left,
but they can't cross the nullclines, so
\emph{all trajectories in} \textsf{\textbf{D}}
\emph{must converge to} $(1/2,1/2)$.

Finally, in \textsf{\textbf{C}} we have
$\frac{dx}{dt} >0$ and $\frac{dy}{dt} > 0$.
All trajectories in this region move up and to the right.
Therefore, they must all eventually cross into
\textsf{\textbf{B}} or \textsf{\textbf{D}}, except
for the special trajectory in \textsf{\textbf{C}}
that converges to $(1/2,1/2)$.

Our conclusion is that any trajectory that begins in the
first quadrant, with $x(0)>0$ and $y(0)>0$, must
converge to $(1/2,1/2)$. Note that we have concluded this
without actually solving the differential equations
given in~\eqref{eqn:compspecexample}.
Thus, this population model
predicts that the two species will co-exist in 
a stable equilibrium.
(Different parameters values may lead to different
conclusions.)

\begin{exercises}
\begin{exercise}
For each of the following predator-prey systems, identify which dependent
variable $x$ or $y$ is the prey population and which is the predator
population.  Is the growth of the prey limited by any factors other 
than the number of predators?  Do the predators have sources of food
other than prey? (Assume that the parameters $\alpha$, $\beta$,
$\gamma$, $\delta$, and $N$ are all positive.)

\parbox{2.5in}{
\begin{enumerate}
\item $$ {dx\over dt} = -\alpha x +\beta x y$$
 $$ {dy\over dt} = \gamma y - \delta x y$$ 
\end{enumerate} }
\hfill
\parbox{3.0in}{
\begin{enumerate}
\item[b)] $$ {dx\over dt} = \alpha x -\alpha{x^2\over N}-\beta x y$$
 $$ {dy\over dt} = \gamma y + \delta x y$$ 
\end{enumerate}}
\end{exercise}

\begin{exercise}
  Perform a scaling analysis for the competitive species
model presented in class.  Transform the equations:
$$ {dR\over dt} = a R  - b R^2 - c RS$$
$$ {dS\over dt} = d S  - e S^2 - f RS$$
by scaling the variables and leaving parameters in front
of the cross terms and a single parameter in front of the
entire right hand side of the second equation:
$$ {du\over d\tau} = u - u^2 - \alpha uv $$
$$ {dv\over d\tau} = r\left(v - v^2 - \beta uv\right) $$
\end{exercise}


\begin{exercise}
Consider the following special case of the above model 
$$ {dx\over dt} = x - x^2 - 3.2 xy $$
$$ {dy\over dt} = 2(y - y^2 - 2.5 xy) $$
Use a Phase Plane Plotter (for example 
http://www.math.psu.edu/melvin/phase/newphase.html 
)
to display the phase plane portrait for this system.
Choose an appropriate window (scale of axes) to display the 
dynamics of the system.  The window should include all steady
state points and show their interaction using nearby trajectories.
Draw enough trajectories to make the long term behavior clear.  
Print the graph and label the trajectories with arrows indicating 
the direction they are traversed in time.

Explain in a paragraph the long term behavior of this system.
What does the model predict as the long term effect of 
competition for food?
\end{exercise}
\begin{exercise}
The following questions involve the generalized Lotka-Volterra model:
$$ \dot{x} = r_1 x + a_{11}x^2 + a_{21} xy $$
$$ \dot{y} = r_1 y + a_{22}y^2 + a_{12} xy $$

\begin{enumerate}
\item Examine the model and identify the meaning
of the parameters $r_1$, $r_2$, $a_{11}$, $a_{12}$, $a_{21}$, $a_{22}$
in terms of species interaction and growth for a model of two species.

\item Reinterpret the parameters for the same equations as a model
of the arms race.  What do $x$ and $y$ represent?

\item Reinterpret these parameters for the same equations as a model
of warfare.  What do $x$ and $y$ represent?

\item Reinterpret these parameters for the same equations as a model 
of competition between marketing departments for two consumer goods
manufacturers.
\end{enumerate}
\end{exercise}
\end{exercises}

\newpage
%
\section{Battle of Attrition--The Lanchester Model}
\index{Lanchester model}
%
We consider a model of a battle in which the opponents
simply blast away at each other until one side is wiped
out.
This model was proposed by F. W. Lanchester\cite{Lanchester}.
The simple version presented here
is \emph{not} a realistic model of modern warfare!
%(It has been suggested as a model for the battles that
%occur between colonies of ants...\emph{find this reference!})
In this model $x(t)$ and $y(t)$ are the sizes of the
opposing armies at time $t$.
As the battle rages, the losses incurred by army $x$
are simply proportional to the size of army $y$,
and the losses incurred by army $y$ are proportional
to the size of army $x$ (but not necessarily with
the same proportionality constant).
That is,
\begin{equation}
\begin{split}
   \frac{dx}{dt} & = - \alpha y \\
   \frac{dy}{dt} & = - \beta x
\end{split}
\end{equation}
where $\alpha$ and $\beta$ are positive constants.
This is a planar linear system.
The coefficient matrix is
\begin{equation}
  A = \begin{bmatrix} 0 & -\alpha \\ -\beta & 0 \end{bmatrix}.
\end{equation}
The eigenvalues are $\lambda_1 = -\sqrt{\alpha\beta}$
and $\lambda_2 = \sqrt{\alpha\beta}$,
so we know that the origin $(0,0)$ is a saddle point.
A corresponding set of eigenvectors are
\[
  \BV_1 = \begin{bmatrix} 1 \\ -\sqrt{\frac{\beta}{\alpha}}\end{bmatrix}
  \quad \textrm{and}\quad
  \BV_2 = \begin{bmatrix} 1 \\ \sqrt{\frac{\beta}{\alpha}}\end{bmatrix}.
\]
The $x$ nullcline is $y=0$, which means that trajectories
cross the $x$ axis vertically. Similarly, the $y$ nullcline is
$x=0$, so trajectories cross the $y$ axis horizontally.
Since the nullclines are the axes, we know that in the first
quadrant, $dx/dt < 0$ and $dy/dt < 0$, so for any initial
condition in the first quadrant, the trajectory will eventually
cross either the $x$ or $y$ axis. If the initial condition
happens to be in the eigenspace associated with the negative
eigenvalue $\lambda_1$, the trajectory will converge to
$(0,0)$. (In this exceptional case, the armies wipe each
other out!)
This eigenspace is given by the line
\begin{equation}
  y = \sqrt{\frac{\beta}{\alpha}}\, x.
\label{eqn:lanchester_eigenspace}
\end{equation}
If the initial condition is above the $\lambda_1$ eigenspace
(so $y_0 > \sqrt{\frac{\beta}{\alpha}}\, x_0$), then
$y$ wins, and if the initial condition is below this line,
then $x$ wins.

There is a different approach we can take to analyze
this system.
By multiplying the first equation by $2\beta x$ and the
second by $-2\alpha y$, we obtain
\begin{equation}
\begin{split}
 2\beta x\frac{dx}{dt} & = -2 \alpha \beta x y \\
 -2\alpha y\frac{dy}{dt} & = 2 \alpha \beta x y.
\end{split}
\end{equation}
Then adding these equations results in
\begin{equation}
   2\beta x\frac{dx}{dt} -2\alpha y\frac{dy}{dt} = 0.
\end{equation}
We note that the left side of this equation is
the $t$ derivative of $\beta x^2 - \alpha y^2$.
So we'll integrate from $0$ to $t$:
\begin{equation}
   \int_0^t \left(2\beta x\frac{dx}{dt} -2\alpha y\frac{dy}{dt}\right)\, ds = \int_0^t 0 \, ds = 0.
\end{equation}
which gives
\begin{equation}
\begin{split}
  \left.\beta x^2(s) - \alpha y^2(s)\right|_{s=0}^{s=t} & = 0 \\
   \beta x^2(t) - \alpha y^2(t) - (\beta x^2(0) - \alpha y^2(0)) & = 0 \\
      \beta x^2(t) - \alpha y^2(t) & = \beta x^2(0) - \alpha y^2(0) =
         \beta x_0^2 - \alpha y_0^2 \\
\end{split}
\end{equation}
By dropping the explicit $t$ dependence of $x(t)$
and $y(t)$, we obtain
\begin{equation}
   \beta x^2 - \alpha y^2 = \beta x_0^2 - \alpha y_0^2.
   \label{eqn:lanchester_hyperbola}
\end{equation}
This is the equation of a hyperbola in the $(x,y)$ plane.
Whether the hyperbola crosses the $x$ or $y$ axis is
determined by the sign of
$\beta x_0^2 - \alpha y_0^2$.
If this value is positive, then the hyperbola
crosses the positive $x$ axis
at $x=\sqrt{x_0^2 - \frac{\alpha}{\beta} y_0^2}$.
If $\beta x_0^2 - \alpha y_0^2$
is negative, then the hyperbola
crosses the $y$ axis at
$y = \sqrt{y_0^2 - \frac{\beta}{\alpha} x_0^2}$.
Thus, if we know the initial sizes of the armies
and the constants $\alpha$ and $\beta$, we can
easily determine which army will win, and what the
size of the winning army will be at the end of the
battle.

\begin{xexample}
\label{exm:Lanchester}
Suppose we have two armies, with $\alpha=1.0$,
$\beta = 1.5$, and $x_0=10,000$.
What is the outcome of the battle if
$y_0=10,000$?  What if $y_0=14,000$?
How large must $y_0$ be if both armies
wipe each other out?

We can answer these questions with
equation \eqref{eqn:lanchester_hyperbola}.
We have
\begin{equation}
  1.5x^2-y^2 = 1.5(10000)^2-y_0^2
\end{equation}
If $y_0=10,000$, we have
\begin{equation}
  1.5x^2-y^2 = 1.5(10000)^2-(10000)^2 = 5\times 10^7.
\end{equation}
The right side is positive, so the hyperbola must cross
the $x$ axis.  That is, the $x$ army wins, and the
size of the surviving army is
\begin{equation}
   x = \sqrt{\frac{5\times 10^7}{1.5}} \approx 5800.
\end{equation}
If $y_0=14,000$, we have
\begin{equation}
  1.5x^2-y^2 = 1.5(10000)^2-(14000)^2 = -4.6\times 10^7.
\end{equation}
The right side is negative, so the hyperbola must
cross the $y$ axis.  The $y$ army wins, and its remaining size
is
\begin{equation}
   y = \sqrt{4.6\times 10^7} \approx 6780.
\end{equation}
The armies will wipe each other out when
\begin{equation}
  y_0 = \sqrt{\frac{\beta}{\alpha}}x_0 
      = \sqrt{\frac{1.5}{1.0}}10000
      \approx 12247.
\end{equation}
Plots of the three curves in the $xy$ plane for each of these
cases are shown in Figure~\ref{fig:LanchesterExamplePlots}.
\begin{figure}
\centerline{
\includegraphics[width=2in]{matlab/LanchesterExamplePlots.eps}
}
\caption{Plots of the three curves discussed in
Example~\ref{exm:Lanchester}.}
\label{fig:LanchesterExamplePlots}
\end{figure}
\end{xexample}

Equation \eqref{eqn:lanchester_eigenspace} gives an asymptote
of the family of hyperbolas.
If the initial sizes of the armies result in a point
on this line, the two armies will eliminate each other.
The equation lets us make an interesting observation.
Suppose $x_0$ is, say, three times larger than $y_0$.
How much more efficient must the $y$ army be in order to
defeat the larger opponent?
In other words, by what factor must $\alpha$
exceed $\beta$
to ensure that $y_0 > \sqrt{\frac{\beta}{\alpha}}\,x_0$?
According to
this equation, the $y$ army must
be more than \emph{nine times} more effective than the $x$ army
if it hopes to win the battle.
In this simple model, raw numbers are very important.

\newpage
\begin{exercises}
\begin{exercise}
\label{ex:lanchester_ants}
A colony of $50,000$ red ants is in a battle
with a colony of $35,000$ larger and stronger black ants.
The sizes and strengths of the two types of ants
are such that $\alpha=3\beta$.
\begin{enumerate}
\item[(a)] Based on this description, which type of ant
(black or red) is represented by $x$
and which is represented by $y$ in the
Lanchester model?
\item[(b)] If the battle proceeds according to the Lanchester
model, which colony will win, and how many ants will
remain in the winning colony?
\end{enumerate}
\end{exercise}
\begin{exercise}
\label{ex:lanchester_general}
In the year 17**, a general is facing an impending battle.
Across the valley from his encampment, the opposing
army has $10,000$ soldiers, but the general has only $6,500$.
His army and the opposing army have similar abilities
and armaments, so, in a Lanchester model of the battle,
the proportionality constant $\alpha$ and $\beta$
are equal.
If, as might be the case in these times, the battle
is accurately modeled by the Lanchester equations,
the general's army will lose the battle.
Fortunately, the general has $5,000$ reinforcement
on the way.  Unfortunately, the opposing army
also expects reinforcements, but only $2,000$.
This is enough to give the advantage to the opposing
army.

After consulting his maps, the general realizes that
he can divert his reinforcements to engage the
opposing army's reinforcements before they
reach the valley. According to the Lanchester model,
the general's reinforcements would defeat
the opposing army's reinforcements.  The surviving
reinforcements could then join the general's army
before the general engages in the battle with the
opposing army across the valley.

\begin{enumerate}
\item[(a)] Show that, if the surviving reinforcements reach the valley
before the main battle begins, the general will win
the main battle.
What will be the size of the surviving army?
\item[(b)] The general realizes that the opposing army
might attack before his surviving reinforcements
reach him. If the reinforcements arrive too late,
his losses might have been too great, and the
reinforcements will not be enough to win the battle.
How large can the pre-reinforcement losses be before
the reinforcement will not be enough to win the main battle?
\end{enumerate}
\end{exercise}
\begin{exercise}
We can use a Lanchester style model to analyze guerilla warfare.
Let $x$ be the number of guerilla troops and $y$ be the number of
conventional troops in a combat region.  We'll neglect "operational
losses" (self-crowding) and reinforcements.  The proposed model is
$$ \dot{x} = -axy  \hspace{3cm}  \dot{y}= -bx $$
%
\begin{enumerate}
\item Discuss the assumptions and relationships necessary to justify
the model.  Does the model seem reasonable? Could it describe what 
is happening in Iraq?  Afghanistan?
\item Find the steady states of the model.  What do these points represent
for the combat situation?  Is there any way the combat could end without
tending toward a steady state?
\item This model is simple enough that we can
find the trajectories without linearizing about the steady states.
Find an equation for the slope $dy/dx$ of the trajectory at each point.
\item Integrate this relationship to find the equation for each trajectory.
What determines the constant of integration?
\item Determine the separatrix (curve in the first quadrant) that separates 
trajectories for which the guerillas win from those where the conventional forces win.
\end{enumerate}
\end{exercise}
\end{exercises}
%
\newpage
%
\section{Arms Race}
%
\section{Romeo and Juliet\index{Romeo}\index{Juliet}}
\begin{exercises}
  \begin{exercise}
    Examine the Love Model where Romeo and Juliet are romatic clones:
$$
\begin{array}{c}
   {dR\over dt} = aR+bJ
   \\
   {dJ\over dt} = bR+aJ
\end{array}
$$
for some constants $a$ and $b$.  Should they expect 
boredom (neither cares for the other) 
sorrow (one loves but the other doesn't)
or bliss (either both hate or both love)?
Considering $b>0$ and any value for $a$ and the initial conditions,
examine all possible outcomes.
  \end{exercise}
\end{exercises}

\section{Laura and Petrarch\index{Laura}\index{Petrarch}}

\begin{exercises}
\begin{exercise}
Adjust parameters, see how the behavior changes...
\end{exercise}
\begin{exercise}
Change the shape of a function... see if the results change
significantly...
\end{exercise}
\end{exercises}

\newpage

% \section{Epidemiology and the SI Model\index{SI model}}

\section{The SIR Model\index{SIR model} of the Spread of a Disease}
Recall the SI model of the spread of a disease, discussed
in Section \ref{sec:SIModel}.
For many diseases, people recover after being infected,
and from then on they are immune to the disease.
To incorporate this into our model of the spread of
a disease, we consider a third class of \emph{recovered}
individuals.
% (Calling them the recovered people is the
%optimistic point of view.  It turns out that the
%model is the same if we interpret this group as
%\emph{removed}, meaning deceased.)

We assume that the rate at which the infected
group changes to recovered is proportional to
the number of infected people.

\emph{Why? ...}

So, if we have a group of infectives, and no
one else is becoming infected, the equation
for $I(t)$ is
\begin{equation}
  \frac{dI}{dt} = -\gamma I,
\end{equation}
where $\gamma > 0$ is the constant of proportionality.
The parameter $\gamma$ depends on how fast people
typically recover from the disease.
The decrease in $I$ because of recovery results
in a corresponding increase in $R(t)$.
Putting this all together, we obtain the
$SIR$ model:\index{SIR model}
\begin{equation}
\begin{split}
   \frac{dS}{dt} & = -\alpha S I \\
   \frac{dI}{dt} & = \alpha S I - \gamma I \\
   \frac{dR}{dt} & = \gamma I
\end{split}
\label{eqn:SIR}
\end{equation}
(Instead of $r$, we have used $\alpha$ as the coefficient of the $SI$ terms.
This is to avoid confusion with $R$ when talking about the model.)
As in the $SI$ model, we have assumed that the total
population remains constant.  This is apparent in the equations:
add them all together to find
\begin{equation}
   \frac{dS}{dt} + \frac{dI}{dt} + \frac{dR}{dt} = 0
   \quad \textrm{which implies} \quad
   S(t) + I(t) + R(t) = N.
\end{equation}
Also note that $R$ does not appear in the first two equations
of \eqref{eqn:SIR}.  This means we can analyze the equations
for $S$ and $I$, and then infer the behavior of $R$ afterwards.

\medskip

\emph{To do: Discuss $(S,I)$ phase plane..}

\medskip

From calculus, we know
\begin{equation}
   \frac{ \frac{dI}{dt} }{ \frac{dS}{dt} } = \frac{dI}{dS}
\end{equation}
This says that, if we consider a solution in the $(S,I)$ plane,
and interpret the curve as the graph of a function of $S$
(i.e. $I=f(S)$), we can find the slope of the curve
at a point by computing $\frac{dI}{dt}/\frac{dS}{dt}$.
Thus we can use the first two equations in
\eqref{eqn:SIR} to obtain
\begin{equation}
   \frac{dI}{dS} = -1 +\frac{\alpha}{\gamma S}
\end{equation}
We can integrate to find
\begin{equation}
   I = -S + \frac{\alpha}{\gamma}\ln(S) + C
\end{equation}
In other words, the function
\begin{equation}
   H(S,I) = I+S-\frac{\alpha}{\gamma}\ln(S)
\end{equation}
is \emph{constant} along solutions.


\medskip
\noindent
\emph{To be continued...}

\newpage
\begin{exercises}
\begin{exercise}
\label{ex:sismodel}
In this exercise, we consider a different variation of the
SI model.
For some diseases, surviving an infection
does not guarantee immunity from subsequent infections.
Infectives can recover,
but once they have recovered, they return to the
susceptible pool.
This model is often called the SIS model.\index{SIS model}
\begin{enumerate}
\item[(a)]
By using the same assumptions concerning the infectious period
as used in the development of the SIR model, give the differential
equations for the SIS model, and show that the equation for
$I(t)$ may be written
\begin{equation}
  \frac{dI}{dt} = \alpha(N-I)I-\gamma I,
\label{eqn:SISequation}
\end{equation}
where $N$ is the total size of the population.
\item[(b)]
Without explicitly solving
\eqref{eqn:SISequation},
determine the long term behavior of the solutions.
How does the long term behavior depend on the parameters
$N$, $\alpha$ and $\gamma$, and the initial condition
$I(0)=I_0$?
\item[(c)]
Equation \eqref{eqn:SISequation} is a a separable
first order differential equation.
Find the solution to the equation.
\end{enumerate}
\end{exercise}

\begin{exercise}
From your own experience, knowledge, and creativity, describe
how disease (say caused by a virus) is spread through a population?  
What mechanisms of transfer are possible?  What pattern of infection 
would you expect in the US? (which states early, which later...  
urban vs rural) How do airplanes affect the pattern of the spread?
\end{exercise}

\begin{exercise}
From your own experience, knowledge, and creativity, describe
how revolutionary ideas spread through a population?  What mechanisms 
of transfer are possible?  What pattern of support would you expect in
the US? (which states early, which later...  urban vs rural)
How did the rise of the modern mass media (TV, radio) affect the spread?
How do telephones, FAXes and the Internet affect the pattern of the spread?
What is different between these two methods of communication and how
might that affect revolutionary behavior?  Use examples of recent 
revolutions to critique your reasoning?
\end{exercise}
%\begin{exercise}
%We talked about a model of drug use which is similar to the SIR model
%of disease spread.  In that model, users and pushers(sellers) are assumed to be the
%same and so are lumped into a single group called $I$.  Rewrite the model to
%separate these groups so that $I$ is replaced by $U$--users and $P$--pushers.
%The behavior difference between the two is that users do not influence nonusers
%as much as pushers do and police tend to arrest (remove) pushers at a higher 
%rate than they do the users.  Users and nonusers can become pushers.  
%But once someone becomes a pusher they don't revert (or their gang removes them!).
%
%Write your newly formed equations neatly together and then briefly describe each term
%(whether new, changed or the same) in each equation to justify the form you chose 
%(or why it didn't change).
%\end{exercise}
\end{exercises}

\section{The SIQR Model\index{SIQR model}}
\section{The SEIR Model\index{SEIR model}}
%
