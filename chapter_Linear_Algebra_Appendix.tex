%
\chapter{Linear Algebra}
%
\section{Matrices}
A \emph{matrix} is an arrangement of numbers into
a set of rows and columns.  We call a matrix with
$m$ rows and $n$ columns an $m\times n$ matrix.
For example,
\[
   A = \begin{bmatrix}
           1 & 5 & -7 \\ 3 & 0 & 4
       \end{bmatrix}
\]
is a $2\times 3$ matrix.
%

\medskip
\noindent
\emph{To do:} Matrix multiplication, basic facts, identity matrix,
determinants, etc.
%
\section{Eigenvalues and Eigenvectors\index{eigenvalue}\index{eigenvector}}
%
The \emph{eigenvalue problem}\index{eigenvalue problem}
for an $n \times n$ matrix $A$
is the problem of finding a number $\lambda$ and a nonzero vector $\BV$
such that
\begin{equation}
    A\BV = \lambda \BV.
    \label{eqn:EIGVALPROB}
\end{equation}
The number $\lambda$ is called an \emph{eigenvalue}, 
and the vector $\BV$ is called an \emph{eigenvector}
associated with the eigenvalue $\lambda$.
Note that if $\BV$ is an eigenvector associated with
$\lambda$, then so is $c\BV$ for any nonzero constant
$c$.
The set of all such vectors, augmented with the zero
vector, is called the \emph{eigenspace}\index{eigenspace}
associated with the eigenvalue $\lambda$.

We can rewrite \eqref{eqn:EIGVALPROB} as
\begin{equation}
   \left(A - \lambda I\right) \BV = \BZero.
\end{equation}
Because we insist that eigenvectors be nonzero,
this equation can give us eigenvectors only if
the matrix $A-\lambda I$ is \emph{singular}.
We know that a matrix is singular if and only if
its determinant is zero, so the equation that determines
the eigenvalues is
\begin{equation}
   \det\left(A-\lambda I\right) = 0.
   \label{eqn:CHAREQN}
\end{equation}
The polynomial $\det\left(A-\lambda I\right)$ is called
the \emph{characteristic polynomial}\index{characteristic polynomial}
of $A$, and
\eqref{eqn:CHAREQN} is 
called the \emph{characteristic equation}\index{characteristic equation}.
%
\newpage
%
\section{Shortcuts for $2\times 2$ Matrices}
In this section, we give
some shortcuts for finding the inverse of and the eigenvectors of $2\times 2$ matrices.

Let
\[
   A = \begin{bmatrix}
              a & b \\ c & d
       \end{bmatrix}.
\]

\noindent
\textbf{Inverse.}
You can easily check that the inverse is
\[
   A^{-1} = \frac{1}{\det A}\begin{bmatrix}
                               d & -b \\ -c & a
                            \end{bmatrix}.
\]
So to find the inverse of a $2x2$ matrix,
\emph{interchange the diagonal elements}, \emph{change the sign of the off-diagonal elements}, and
\emph{divide by the determinant}.

\begin{xexample}
\[
  A = \begin{bmatrix}
          1 & 7 \\ -3 & 4
      \end{bmatrix}
  \quad\quad\quad
  A^{-1} = \frac{1}{25}\begin{bmatrix}
                          4 & -7 \\ 3 & 1
                       \end{bmatrix}
\]
\end{xexample}

\noindent
\textbf{Eigenvalues and eigenvectors.}
To find the eigenvalues of $A$, we must solve
$\det(A-\lambda I)=0$ for $\lambda$.
% (The expression $\det(A-\lambda I)$ is called
% the \emph{characteristic polynomial}.)
We have
\[
\begin{split}
   \det(A-\lambda I) & = (a-\lambda)(d-\lambda)-bc \\
                     & = \lambda^2-(a+d)\lambda + (ad-bc) \\
		     & = \lambda^2 - \textrm{Tr(A)}\lambda + \det(A)
\end{split}
\]
where $\textrm{Tr(A)} = a+d$ is the \emph{trace} of $A$.
(The trace of a square matrix is the sum of the diagonal elements.)
Then the eigenvalues are found by using the quadratic
formula, as usual.

Now consider the problem of finding the eigenvectors
for the eigenvalues $\lambda_1$ and $\lambda_2$.
An eigenvector associated with $\lambda_1$ is a nontrivial
solution $\BV_1$ to
\begin{equation}
    (A-\lambda_1 I)\BV = \BZero.
\label{eqn:eigvec}
\end{equation}
Now
\[
   A - \lambda_1 I = \begin{bmatrix}
                           a-\lambda_1 & b \\
			   c & d-\lambda_1
                     \end{bmatrix}
\]
The matrix $A-\lambda_1 I$ \emph{must} be singular.
That is precisely what makes $\lambda_1$ an eigenvalue.
If a $2\times 2$ matrix is singular, the second
row \emph{must} be a multiple of the first row (unless
the first row is zero).  Therefore, we know that putting
$A-\lambda_1 I$ into row echelon form must result in
a row of zeros.  Since we know this must be the case,
there is no need to actually do it!  All we need to
find an eigenvector is the first row.
In particular, if $\BV = [v_1,v_2]^{\textsf{T}}$,
then \eqref{eqn:eigvec} implies
\begin{equation}
  (a-\lambda_1)v_1 + b v_2 = 0.
\label{eqn:eigveceqn}
\end{equation}
We could solve this for, say, $v_2$ in terms of $v_1$,
and give all the possible eigenvectors in terms of
the arbitrary parameter $v_1$. (This is the
\emph{eigenspace} associated with the eigenvalue $\lambda_1$.)
However,
% for the
% purpose of solving a system of differential equations,
often
all we need is \emph{one} eigenvector from this space.
(More precisely, we want a \emph{basis} for the eigenspace.)
An easy solution to \eqref{eqn:eigveceqn}
is $v_1=-b$ and $v_2 = (a-\lambda_1)$.
Thus (unless $(a-\lambda_1)$ and $b$ both happen to be
zero), once we write down the matrix $A-\lambda_1 I$,
we can immediately obtain the eigenvector
\[
   \BV_1 = \begin{bmatrix} -b \\ a-\lambda_1 \end{bmatrix}
\]
If both $(a-\lambda_1)$ and $b$ are zero, we can use the
second row to find an eigenvector:
\[
   \BV_1 = \begin{bmatrix} d-\lambda_1 \\ -c \end{bmatrix}.
\]
So, once we have an eigenvalue
of a $2\times 2$ matrix, it is very easy to find
a corresponding eigenvector.
This works even when the eigenvalue is complex.
It will give a correct complex eigenvector.

\begin{xexample}
\[
   A = \begin{bmatrix} 1 & 2 \\ 3 & -4 \end{bmatrix}
\]
The characteristic polynomial is
\[
   \lambda^2 - (1+(-4))\lambda + ((1)(-4)-(2)(3)) = \lambda^2 + 3\lambda - 10,
\]
so we find
\[
  \lambda = \frac{-3\pm\sqrt{9-4(-10)}}{2} = -5, 2.
\]
Let $\lambda_1 = -5$ and $\lambda_2 = 2$.
Now we'll find an eigenvector for each eigenvalue.

\medskip
\noindent
\underline{$\lambda_1 = -5$}
\[
   A-\lambda_1 I = \begin{bmatrix}
                   6 & 2 \\ 3 & 1
                   \end{bmatrix}
\]
As expected, we see that the second row
is a multiple of the first. Using the shortcut discussed
above, we can immediately find one eigenvector to be
\[
   \BV_1 = \begin{bmatrix} -2 \\ 6 \end{bmatrix}
\]
Of course, since any nonzero multiple of an eigenvector
is also an eigenvector, we could also choose
\[
   \BV_1 = \begin{bmatrix} -1 \\ 3 \end{bmatrix}
\]

\medskip
\noindent
\underline{$\lambda_2 = 2$}
\[
  A - \lambda_2 I = \begin{bmatrix}
                      -1 & 2 \\ 3 & -6
                    \end{bmatrix}
\]
In this case, a possible eigenvector is
\[
  \BV_2 = \begin{bmatrix} -2 \\ -1 \end{bmatrix}
\]
or, if we want to minimize the number of minus signs,
\[
  \BV_2 = \begin{bmatrix} 2 \\ 1 \end{bmatrix}
\]
\end{xexample}

\begin{xexample}
\[
   A = \begin{bmatrix} -1 & -3 \\ 4 & 3 \end{bmatrix}
\]
The characteristic polynomial is
\[
  \lambda^2 - 2\lambda + 9,
\]
and the eigenvalues are
\[
  \lambda = \frac{2\pm \sqrt{4-36}}{2} = 1\pm 2\sqrt{-2}
    = 1 \pm 2 \sqrt{2} \, i
\]
Let $\lambda_1 = 1 + 2\sqrt{2}\, i$, and $\lambda_2 = \lambda_1^{*}$.
We'll find an eigenvector associated with
the eigenvalue $\lambda_1$.

We have
\[
   A - \lambda_1 I = \begin{bmatrix}
                        -1-(1+2\sqrt{2}\,i) & -3 \\
			4 & 3-(1+2\sqrt{2}\,i)
                     \end{bmatrix}
		   = \begin{bmatrix}
		        -2-2\sqrt{2}\, i & -3 \\
			4 & 2-2\sqrt{2}\,i
		     \end{bmatrix}
\]
By using the shortcut discussed above, we can
immediately write down the eigenvector
\[
  \BV_1 = \begin{bmatrix} 3 \\ -2-2\sqrt{2}\, i \end{bmatrix}
\]
(If we were solving a system of differential equations, we would
then want to express $\BV_1$ as
\[
   \BV_1 = \begin{bmatrix} 3 \\ -2 \end{bmatrix}
           + i \begin{bmatrix} 0 \\ -2\sqrt{2} \end{bmatrix}
\]
so $\BA = \begin{bmatrix} 3 \\ -2\end{bmatrix}$
and $\BB = \begin{bmatrix} 0 \\ -2\sqrt{2} \end{bmatrix}$.)
\end{xexample}