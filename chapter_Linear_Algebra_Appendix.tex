%
\chapter{Linear Algebra}
%
\section{Matrices -- a shorthand notation}
A \emph{matrix} is an arrangement of numbers into
a set of rows and columns.  We call a matrix with
$m$ rows and $n$ columns an $m\times n$ matrix.
For example,
\[
   \MA = \begin{bmatrix}
           1 & 5 & -7 \\ 3 & 0 & 4
       \end{bmatrix}
\]
is a $2\times 3$ matrix.  The entries are referred to using double
subscripts representing the row and then column for the entry.  Thus
$\MA_{23}=4$ above.
%
Matrices are useful notation when dealing with systems of linear
equations or with collections of vector quantities.  Each row of
the matrix can be thought of as a vector with each number representing
the coordinate for that column's dimension(direction).
Similarly, each column can be thought of as a vector.  Note that
the usefulness of these two views depends on the situation and while
we sometimes switch from the row view to the column view, we rarely
use both at the same time.

{\bf Multiplication:} Matrix multiplication is a generalization of the dot product operation 
on vectors.  Each row of the first matrix is combined with each column
of the second matrix via the dot product to give one number in the resulting
product matrix.
For example, (bold face variables represent matrices arrows represent vectors)
\[
   \begin{array}{c}
     2x+3y=5 \\ 4x-6y=3
   \end{array}
   \Leftrightarrow
   \begin{bmatrix}
      2 & 3 \\ 4 & -6
   \end{bmatrix}
   \cdot
   \begin{bmatrix}
      x \\ y
   \end{bmatrix}
    =
   \begin{bmatrix}
      5 \\ 3
   \end{bmatrix}
   \Leftrightarrow
   \MA\BX = \BZero
\]
Note that in general, order matters for matrix multiplication:  
$\MA\MB \neq \MB\MA$ for many $\MA$ and $\MB$.
For this reason, we distinguish premultiplication and postmultiplication.
Also for this reason, division cannot use the standard fraction notation.
Instead we use negative powers (called the inverse) to represent division.

In the special case where a matrix $\MA$ multiplies a vector,
the result is a linear combination of the columns of the matrix $\MA$.
\[
   \begin{bmatrix}
      2 & 3 \\ 4 & -6
   \end{bmatrix}
   \cdot
   \begin{bmatrix}
      x \\ y
   \end{bmatrix}
    =
   x
   \begin{bmatrix}
      2 \\ 4
   \end{bmatrix}
   +y
   \begin{bmatrix}
      3 \\ -6
   \end{bmatrix}
\]
One can think of this as addition of vectors.  The resulting vector must have the
same dimension ($m=2$ in this case) as the vectors we are taking the linear combination of.  
Since the coefficients in the linear combination are arbitrary, the result is arbitrary.
But for some matrices, the dimension of the resulting linear combination is less 
than $m$.  This occurs when one column of the matrix can be written as a linear combination
of the others.  The collection of all possible vectors resulting from multiplication
of an arbitrary column vector $\BX$ by a matrix $\MA$ is the \emph{ Row Space} of $\MA$.
Similarly, multiplying $\MA$ by a row vector $\BX$ results in a set of vectors
called the \emph{ Column Space} of $\MA$.  Matrix theory ensures that the dimension
of the row space and column space for any matrix are the same and that number is called
the \emph{ rank} of the matrix.  The rank $r$ of $\MA$ is the effective number of 
independent equations represented by the matrix $\MA$.  Note that $r\leq m$.

As an example, the rank of the following matrices are 1, 2 and 2 respectively.
\[
   \begin{bmatrix}
      2 & 3 \\ 4 & 6
   \end{bmatrix}
   \quad
   \begin{bmatrix}
      2 & 3 \\ 4 & -6
   \end{bmatrix}
   \quad
   \begin{bmatrix}
      2 & 3 & 4 \\ 4 & 6 & 7 \\ 6 & 9 & 11
   \end{bmatrix}
\]
This can be seen by noting that in the first matrix, the second row is a multiple of
the first.  Thus the second equation in $\MA \BX$ would not be independent of the first.
The second matrix generates two independent equations. 
The third matrix has its third row being the sum of the first two.

\medskip
\noindent
{\bf Special Matrices:} 
Addition of matrices involves the sum of elements componentwise.
Some special matrices involve the identity under addition and multiplication.
Thus we define $\MZero$ to be the matrix of zeros and $\MA+\MZero=\MA$ for all $\MA$.
Using our notion of multiplication, we note that a matrix ($\MI$ of zeros with ones down
the main (left to right, up to down) diagonal ensures $\MA\MI=\MI\MA=\MA$ for all $\MA$.
The zero matrix $\MZero$ and the identity matrix $\MI$ are particularly important matrices.

\medskip
\noindent
{\bf Inverse:}
The {\emph inverse} of a square matrix $\MA$ is the matrix by which you can multiply to
obtain the identity $\MI$.  Denoting the inverse $\MA^{-1}$, we have $\MA^{-1}\MA=\MI$.
It is important to note that just as the number $0$ has no multiplicative inverse, there
are matrices with no inverse.  Many matrix properties are equivalent to having an inverse.
For example, if the rank $r$ is the size of $\MA$, ($r=n$) then $\MA$ has an inverse (is invertible).
A matrix that has no inverse is called a {\emph singular} matrix.

\medskip
\noindent
{\bf Determinant:}
A common matrix property that is used to determine whether a matrix is invertible
is called the determinant.  The determinant is a function of all the elements of $\MA$
which produces a single number.  That number is $0$ precisely when $\MA$ has no inverse.
This is because, generally, you must divide by the determinant to obtain the inverse.
Unfortunately the determinant is defined recursively, breaking up the matrix into smaller
and smaller pieces.  For n=1,2, or 3 there are geometric patterns involving the diagonals 
of $\MA$ to help in this process but for larger matrices no such simple patterns exist.

Let $|\MA|_{ij}$ be the determinant of the martrix obtained by removing 
the $i^{th}$ row and $j^{th}$ column from $\MA$.  Define the determinant of a $1\times 1$
matrix be the value of its single entry.  Then the determinant $\det\p{\MA}$ is defined
to be 
\[
  \det\p{\MA} = \sum_{j=1}^{n} (-1)^{j+1} \MA_{1j}|\MA|_{1j}
\]

For the $2\times 2$ case, we have $\det\p{\MA} = \MA_{11} \MA_{22} - \MA_{12}\MA_{21}$.
You will notice that this is the product along the main diagonal minus the product
of the off diagonal (upper left to lower right).

The $3\times 3$ case is only slightly more complicated.  Here, we envision diagonals
wrapping around the right and left edges of the matrix and reappearing on the other side.
Thus there are three right moving diagonals and three left moving diagonals.  The determinant
is the sum of products along the right moving diagonals minus the sum of products along the 
left moving diagonals.  For example,
\[
  \det\p{\begin{bmatrix} a & b & c \\ d & e & f \\ g & h & i \end{bmatrix}}=
  (aeh+bfg+cdh)-(ceg+bdi+afh).
\]

Remember that no such simple method works for larger matrices.  Also that
the determinant is zero if and only if the matrix has an inverse.

\medskip
\noindent
{\bf Trace:}
A second property of matrices which does not predict whether a matrix is invertible, but
is useful nevertheless is called the {\emph trace}.  The trace $Tr\p{\MA}$ is the sum of
the main diagonal entries.  It is useful because its value does not change if you change
the coordinate system in which you specify your vectors (and thus coliummns and rows of $\MA$).

%
\section{Eigenvalues and Eigenvectors\index{eigenvalue}\index{eigenvector}}
%
Matrix multiplication transforms vectors, producing new vectors in the 
same dimensional space.  
One important way to describe how this process occurs is to find special
vectors for which the transformation is very simple.  The transformation of
other vectors can then be stated in terms of the effect on these special
vectors.  Eigenvectors are vectors for which the transformation is a simple
stretching.  That is, they transform to a multiple of themselves.  This multiple
measures the stretching $\MA$ provides in that direction and is called the
eigenvalue associated with that eigenvector.

The \emph{eigenvalue problem}\index{eigenvalue problem}
for an $n \times n$ matrix $A$
is the problem of finding the \emph{eigenvalue} $\lambda$ and the associated 
\emph{eigenvector} $\BV$ such that
\begin{equation}
    A\BV = \lambda \BV.
    \label{eqn:EIGVALPROB}
\end{equation}
Note that if $\BV$ is an eigenvector associated with
$\lambda$, then so is $c\BV$ for any nonzero constant $c$.
So there is a family of such eigenvectors all pointing along
the same straight line.  This set of vectors, augmented with the zero
vector, is called the \emph{eigenspace}\index{eigenspace}
associated with the eigenvalue $\lambda$.  Generally one
representative of the family (often with unit length) is
chosen as the eigenvector of interest.

To solve \eqref{eqn:EIGVALPROB} we rewrite it as
\begin{equation}
   \left(A - \lambda I\right) \BV = \BZero,
\end{equation}
and note that if the matrix $A-\lambda I$ has an inverse, 
then multiplying both sides by it yields $\BV=\BZero$.
Because we insist that eigenvectors be nonzero,
the matrix $A-\lambda I$ must be \emph{singular} and thus
\begin{equation}
   \det\left(A-\lambda I\right) = 0.
   \label{eqn:CHAREQN}
\end{equation}
The $n^{th}$ order polynomial in $\lambda$ on the left hand side is called
the \emph{characteristic polynomial}\index{characteristic polynomial}
of $A$, and
\eqref{eqn:CHAREQN} is 
called the \emph{characteristic equation}\index{characteristic equation}.
The roots of this polynomial are the eigenvalues of $\MA$.
Once we find the eigenvalues of $\MA$ we can plug each one into \eqref{eqn:CHAREQN}
and solve for the associated eigenvector.  Usually there are $n$ eigenvalues and
$n$ eigenvectors, though some trouble may arise if \eqref{eqn:CHAREQN} has roots
of multiplicity greater than one.  Those eigenvalues are also said to have multiplicity
greater than one and there may be fewer eigenvectors in that case.  Generalized
eigenvectors can be defined and constructed in such cases to bring the total
number of eigenvectors to $n$ and the result is a \emph{basis} (the unit vectors
of a coordinate system) for $n$ dimensional space.

Switching coordinate systems from the original standard basis to the eigenvector
basis is extremely powerful because the transformation of multiplying by $\MA$ 
is then written as a diagonal matrix (or almost diagonal in the case of generalized 
eigenvectors).  This diagonal matrix is called the Jordan Form for $\MA$.  The 
values on the diagonal are precisely the eigenvalues of $\MA$.  The coordinate
switch is acheived by forming a matrix whose columns are the eigenvectors of $\MA$.
Let $\MS$ be such a matrix.  Then in the new coordinate system the action of multiplying
by $\MA$ is given by $\MD=\MS^{-1}\MA\MS$ which is a diagonal matrix (unless generalized
eigenvectors had to be used to form $\MS$).

Two helpful properties of $\MA$ which do not change under a coordinate transformation
are the determinant and the trace.  This is sometimes helpful in finding eigenvalues because
the determinant is the product of the eigenvalues and the trace is the sum:
\[
  \det\p{\MA} = \Pi_{i=1}^n \lambda_i  \quad \mbox{ and } \quad Tr\p{\MA}=\sum_{i=1}^n \lambda_i.
\]

%
%\newpage
%
\section{Shortcuts for $2\times 2$ Matrices}
In this section, we give
some shortcuts for finding the inverse of and the eigenvectors of $2\times 2$ matrices.
%
Let
\[
   A = \begin{bmatrix}
              a & b \\ c & d
       \end{bmatrix}.
\]

\noindent
\textbf{Inverse:}
You can check through multiplying that the inverse is
\[
   A^{-1} = \frac{1}{\det A}\begin{bmatrix}
                               d & -b \\ -c & a
                            \end{bmatrix}.
\]
So to find the inverse of a $2x2$ matrix,
\emph{interchange the diagonal elements}, \emph{change the sign of the off-diagonal elements}, and
\emph{divide by the determinant}.

\begin{xexample}
\[
  A = \begin{bmatrix}
          1 & 7 \\ -3 & 4
      \end{bmatrix}
  \quad\quad\quad
  A^{-1} = \frac{1}{25}\begin{bmatrix}
                          4 & -7 \\ 3 & 1
                       \end{bmatrix}
\]
\end{xexample}

\noindent
{\bf Eigenvalues and eigenvectors:}
To find the eigenvalues of $A$, we solve
$\det(A-\lambda I)=0$ for $\lambda$.
We have
\[
\begin{split}
   \det(A-\lambda I) & = (a-\lambda)(d-\lambda)-bc \\
                     & = \lambda^2-(a+d)\lambda + (ad-bc) \\
		     & = \lambda^2 - \textrm{Tr(A)}\lambda + \det(A).
\end{split}
\]
The eigenvalues are found by using the quadratic
formula, resulting in two solution $\lambda_1$ and $\lambda_2$.

Let's find the eigenvectors
for the eigenvalues $\lambda_1$ and $\lambda_2$.
The eigenvector associated with $\lambda_1$ is a nontrivial
solution $\BV_1$ to
\begin{equation}
    (A-\lambda_1 I)\BV_1 = \BZero.
\label{eqn:eigvec}
\end{equation}
But $(A-\lambda_1 I)$ was chosen to be singular, so the second row must
be a multiple of the first.  Using only the first row of 
\eqref{eqn:eigvec} we find 
\begin{equation}
  (a-\lambda_1)v_1 + b v_2 = 0,
\label{eqn:eigveceqn}
\end{equation}
where $\BV = [v_1,v_2]^{\textsf{T}}$.
An easy solution to \eqref{eqn:eigveceqn}
is $v_1=-b$ and $v_2 = (a-\lambda_1)$ so that
\[
   \BV_1 = \begin{bmatrix} -b \\ a-\lambda_1 \end{bmatrix}
\]
If both $(a-\lambda_1)$ and $b$ are zero, we can use the
second row to find an eigenvector:
\[
   \BV_1 = \begin{bmatrix} d-\lambda_1 \\ -c \end{bmatrix}.
\]
So, once we have an eigenvalue
of a $2\times 2$ matrix, it is very easy to find
a corresponding eigenvector.
This works even when the eigenvalue is complex.
It will give a correct complex eigenvector.

\begin{xexample}
\[
   A = \begin{bmatrix} 1 & 2 \\ 3 & -4 \end{bmatrix}
\]
The characteristic polynomial is
\[
   \lambda^2 - (1+(-4))\lambda + ((1)(-4)-(2)(3)) = \lambda^2 + 3\lambda - 10,
\]
so we find
\[
  \lambda = \frac{-3\pm\sqrt{9-4(-10)}}{2} = -5, 2.
\]
Let $\lambda_1 = -5$ and $\lambda_2 = 2$.
Now we'll find an eigenvector for each eigenvalue.

\medskip
\noindent
\underline{$\lambda_1 = -5$:}
\[
   A-\lambda_1 I = \begin{bmatrix}
                   6 & 2 \\ 3 & 1
                   \end{bmatrix}
\]
As expected, we see that the second row
is a multiple of the first. Using the shortcut discussed
above, we can immediately find one eigenvector to be
\[
   \BV_1 = \begin{bmatrix} -2 \\ 6 \end{bmatrix}
\]

\medskip
\noindent
\underline{$\lambda_2 = 2$:}
\[
  A - \lambda_2 I = \begin{bmatrix}
                      -1 & 2 \\ 3 & -6
                    \end{bmatrix}
\]
In this case, an eigenvector is
\[
  \BV_2 = \begin{bmatrix} -2 \\ -1 \end{bmatrix}
\]
Of course, since any nonzero multiple of an eigenvector
is also an eigenvector, we could also choose
\[
   \BV_1 = \begin{bmatrix} -1 \\ 3 \end{bmatrix}
  \quad \mbox{ and } \quad
  \BV_2 = \begin{bmatrix} 2 \\ 1 \end{bmatrix}
\]
\end{xexample}

\begin{xexample}
\[
   A = \begin{bmatrix} -1 & -3 \\ 4 & 3 \end{bmatrix}
\]
The characteristic polynomial is
\[
  \lambda^2 - 2\lambda + 9,
\]
and the eigenvalues are
\[
  \lambda = \frac{2\pm \sqrt{4-36}}{2} = 1\pm 2\sqrt{-2}
    = 1 \pm 2 \sqrt{2} \, i
\]
Let $\lambda_1 = 1 + 2\sqrt{2}\, i$, and $\lambda_2 = \lambda_1^{*}$.
We'll find an eigenvector associated with
the eigenvalue $\lambda_1$.

We have
\[
   A - \lambda_1 I = \begin{bmatrix}
                        -1-(1+2\sqrt{2}\,i) & -3 \\
			4 & 3-(1+2\sqrt{2}\,i)
                     \end{bmatrix}
		   = \begin{bmatrix}
		        -2-2\sqrt{2}\, i & -3 \\
			4 & 2-2\sqrt{2}\,i
		     \end{bmatrix}
\]
By using the shortcut discussed above, we can
immediately write down the eigenvector
\[
  \BV_1 = \begin{bmatrix} 3 \\ -2-2\sqrt{2}\, i \end{bmatrix}
\]
%(If we were solving a system of differential equations, we would
%then want to express $\BV_1$ as
%\[
%   \BV_1 = \begin{bmatrix} 3 \\ -2 \end{bmatrix}
%           + i \begin{bmatrix} 0 \\ -2\sqrt{2} \end{bmatrix}
%\]
%so $\BA = \begin{bmatrix} 3 \\ -2\end{bmatrix}$
%and $\BB = \begin{bmatrix} 0 \\ -2\sqrt{2} \end{bmatrix}$.)
\end{xexample}
