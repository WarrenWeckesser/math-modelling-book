
\chapter{Markov Chains}

\section{Introduction}
A (finite) Markov chain\index{Markov chain} is a process with a finite
number of states (or outcomes, or events) in which
the probability of being in a particular state
at step $n+1$ depends only on the state occupied
at step $n$.

Let $S = \{S_1,S_2,\ldots,S_r\}$ be the possible states.
Let
\begin{equation}
   \BX(n) = \begin{bmatrix}
               x_1(n) \\ x_2(n) \\ \vdots \\ x_r(n)
           \end{bmatrix}
\end{equation}
be the vector  of probabilities of each state
at step $n$. That is, the $i$th entry of $\BX(n)$
is the probability that the process is in
state $S_i$ at step $n$.
For such a probability vector, $x_1(n)+x_2(n)+\cdots+x_r(n) = 1$.

Let
\begin{equation}
  a_{ij} = \textrm{Prob( State $n+1$ is $S_i$ $|$ State $n$ is $S_j$)},
\end{equation}
and let
\begin{equation}
   A = \begin{bmatrix}
           a_{11} & a_{12} & \cdots & a_{1r} \\
	   a_{21} & a_{22} & \cdots & a_{2r} \\
	   \vdots  &  & \ddots \\
	   a_{r1} &  & & a_{rr}
       \end{bmatrix}
\end{equation}
That is, $a_{ij}$ is the (conditional) probability
of being in state $S_i$ at step $n+1$ given that the
process was in state $S_j$ at step $n$.
$A$ is called the \emph{transition matrix}.
Note that $A$ is a stochastic matrix:
the sum of the entries in each column is 1.
$A$ contains all the conditional probabilities
of the Markov chain.
It can be useful to label the rows and columns
of $A$ with the states, as in this example
with three states:

\medskip
\centerline{%
\begin{minipage}{3in}
\begin{tabbing}
  MMMMM \= MMMMMMMMMMMMMMM \kill
    \> \hspace{1.5cm} State $n$ \\
    \> \hspace{1cm} $\overbrace{S_1 \quad\; S_2 \quad \; S_3}$ \\
    State $n+1$ \> $\;\left\{\begin{matrix} S_1 \T\B \\ S_2 \B \\ S_3 \B \end{matrix}
                          \begin{bmatrix}
                              a_{11} & a_{12} & a_{13} \T\B \\
			      a_{21} & a_{22} & a_{23} \B \\
			      a_{31} & a_{32} & a_{33} \B
                          \end{bmatrix}\right.$
\end{tabbing}
\end{minipage}
}

\medskip
The fundamental property of a Markov chain is that 
\begin{equation}
   \BX(n+1) = A\BX(n).
\end{equation}
Given an initial probability vector $\BX(0)$,
we can determine the probability vector at any
step $n$ by computing the iterates of a linear map.

The information contained in the transition matrix
can also be represented in a \emph{transition diagram}.
This is the graph $\Gamma(A)$ associated with 
the transition matrix $A$.
If $a_{ij} > 0$, the graph has an edge from state
$j$ to state $i$, and we label the edge with the
value $a_{ij}$.
Examples are given
in the following discussions.

We will consider two special cases
of Markov chains: regular Markov chains and absorbing
Markov chains.
Generalizations of Markov chains, including
continuous time Markov processes and infinite
dimensional Markov processes, are widely studied,
but we will not discuss them in these notes.
%
\bigskip
\begin{exercises}
\begin{exercise}
Draw the transition diagram for each of the following
matrices.
\begin{enumerate}
\item[(a)] $\ds \begin{bmatrix} 0 & 1/4 \\ 1 & 3/4 \end{bmatrix}$
\vspace{0.2cm}
\item[(b)] $\ds \begin{bmatrix}
                0.5 & 0.5 & 0.3 \\
		0.5 & 0.25 & 0.7 \\
		 0  & 0.25 & 0
           \end{bmatrix}$
\end{enumerate}
\end{exercise}
\end{exercises}
\newpage
%
\section{Regular Markov Chains}

%\noindent
\begin{definition} A Markov chain is a \emph{regular}
Markov chain if the transition
matrix is primitive.
(Recall that a matrix $A$ is primitive if there is
an integer $k>0$ such that all entries in $A^k$
are positive.)
\end{definition}
\medskip
Suppose a Markov chain with transition matrix $A$ is regular,
so that $A^k > 0$ for some $k$.
Then no matter what the
initial state, in $k$ steps there is a positive
probability that the process is in \emph{any} of the states.

\noindent
\textbf{Essential facts about regular Markov chains.}
\begin{enumerate}
\item
$A^n\rightarrow W$ as $n\rightarrow\infty$, where
$W$ is a constant matrix and all the columns of
$W$ are the same.
\item There is a unique probability vector
$\BW$ such that $A\BW = \BW$.

Note:
\begin{enumerate}
\item $\BW$ is a fixed point of the linear map $\BX(n+1) = A\BX(n)$.
\item $\BW$ is an eigenvector associated with the eigenvalue $\lambda=1$.
(The claim implies that the transition matrix $A$ of a regular
Markov chain \emph{must} have the eigenvalue $\lambda=1$.
Then $\BW$ is the eigenvector whose entries add up to $1$.)
\item The matrix $W$ is $W = \begin{bmatrix} \BW & \BW & \cdots & \BW\end{bmatrix}$.
\end{enumerate}
\item $A^n\BX(0) \rightarrow \BW$ as $n\rightarrow\infty$ for \emph{any}
initial probability vector $\BX(0)$.
Thus $\BW$ gives the long-term probability distribution of the
states of the Markov chain.
\end{enumerate}

\begin{xexample}
\textbf{Sunny or Cloudy?}
A meteorologist studying the weather in a region
decides to classify each day as simply \emph{sunny}
or \emph{cloudy}.  After analyzing several years of weather records,
he finds:
\begin{itemize}
\item the day after a sunny day is
sunny $80$\% of the time, and cloudy $20$\% of the time; and
\item the day after a cloudy day is
sunny $60$\% of the time, and cloudy $40$\% of the time.
\end{itemize}
We can setup up a Markov chain to model this process.
There are just two states: $S_1=$ \emph{sunny}, and $S_2 = $ \emph{cloudy}.
The transition diagram is

\medskip
\centerline{%
\includegraphics[width=4in]{fig/SunnyCloudyStateDiagram.eps}
}

\medskip
\noindent
and the transition matrix is
\begin{equation}
  A = \begin{bmatrix}
          0.8 & 0.6 \\
	  0.2 & 0.4 
      \end{bmatrix}.
\end{equation}
We see that all entries of $A$ are positive, so the Markov
chain is regular.
%  (The conditions of the 
% definition are satisfied when $k=1$.)

To find the long-term probabilities of sunny and cloudy days,
we must find the eigenvector of $A$ associated with the
eigenvalue $\lambda=1$. We know from Linear Algebra that
if $\BV$ is an eigenvector, then so is $c\BV$ for any
constant $c\ne0$.  The probability vector $\BW$ is the
eigenvector that is also a probability vector. That is,
the sum of the entries of the vector $\BW$ must be $1$.

We solve
\begin{equation}
\begin{split}
   A\BW & = \BW \\
   (A-I)\BW & = \BZero
\end{split}
\end{equation}
Now
\begin{equation}
  A-I = \begin{bmatrix}
            -0.2 &  0.6 \\
	     0.2 & -0.6
        \end{bmatrix}
\end{equation}
If you have recently studied Linear Algebra, you could probably
write the answer down with no further work, but we will
show the details here.
We form the augmented matrix and use Gaussian elimination:
\begin{equation}
\begin{bmatrix}
   -0.2 &  0.6 & \vdots & 0 \\
    0.2 & -0.6 & \vdots & 0
\end{bmatrix}
\rightarrow
\begin{bmatrix}
   1 & -3 & \vdots & 0 \\
   0 &  0 & \vdots & 0
\end{bmatrix}
\end{equation}
which tells us $w_1 = 3w_2$, or $w_1=3s$, $w_2 = s$, where 
$s$ is arbitrary, or
\begin{equation}
   \BW = s\begin{bmatrix} 3 \\ 1 \end{bmatrix}
\end{equation}
The vector $\BW$ must be a probability vector, so $w_1+w_2=1$.
This implies $4s=1$ or $s=1/4$.
Thus
\begin{equation}
   \BW = \begin{bmatrix} 3/4 \\ 1/4 \end{bmatrix}.
\end{equation}
This vector tells us that in the long run,
the probability is $3/4$ that the process will be in state 1,
and $1/4$ that the process will be in state 2.
In other words, in the long run
$75$\% of the days are sunny and $25$\% of the days
are cloudy.
\end{xexample}
%
\begin{xexample}
Here are a few examples of determining whether or not a
Markov chain is regular.
\begin{enumerate}
\item Suppose the transition matrix is
\begin{equation}
  A = \begin{bmatrix} 1/3 & 0 \\ 2/3 & 1 \end{bmatrix}.
\end{equation}
We find
\begin{equation}
 A^2 = \begin{bmatrix} (1/3)^2 & 0 \\ (2/3)(1+1/3) & 1 \end{bmatrix},
 \quad
 A^3 = \begin{bmatrix} (1/3)^3 & 0 \\ (2/3)(1+1/3+(1/3)^2) & 1 \end{bmatrix},
\end{equation}
and, in general,
\begin{equation}
 A^n = \begin{bmatrix} (1/3)^n & 0 \\ (2/3)(1+1/3+\cdots+(1/3)^{n-1}) & 1 \end{bmatrix}.
\end{equation}
The upper right entry in $A^n$ is $0$ for all $n$, so the Markov chain
is \emph{not} regular.
\item Here's a simple example that is not regular.
\begin{equation}
   A = \begin{bmatrix} 0 & 1 \\ 1 & 0 \end{bmatrix}
\end{equation}
Then
\begin{equation}
   A^2 = I, \quad A^3 = A, \quad \textrm{etc.}
\end{equation}
Since $A^n=I$ if $n$ is even and $A^n=A$ if $n$ is odd, $A$
always has two entries that are zero.  Therefore the Markov
chain is not regular.
\item
Let
\begin{equation}
  A = \begin{bmatrix}
           1/5 & 1/5 & 2/5 \\
	   0   & 2/5 & 3/5 \\
	   4/5 & 2/5 & 0
      \end{bmatrix}
\end{equation}
The transition matrix has two entries that are zero, but
\begin{equation}
  A^2 = \begin{bmatrix}
           9/25 & 7/25 & 5/25 \\
	   12/25 & 10/25 & 6/25 \\
	   4/25 & 8/25 & 14/25
        \end{bmatrix}.
\end{equation}
Since all the entries of $A^2$ are positive, the Markov chain
is regular.
\end{enumerate}
\end{xexample}
%
%
\begin{exercises}
\begin{exercise}
\label{ex:markov1}
Let
\begin{equation}
  A = \begin{bmatrix}
           19/20 & 1/10 & 1/10 \\
	   1/20 & 0   & 0   \\
	   0    & 9/10 & 9/10
      \end{bmatrix}
\end{equation}
be the transition matrix of a Markov chain.
\begin{enumerate}
\item[(a)]
Draw the transition diagram that corresponds to this transition matrix.
\item[(b)]
Show that this Markov chain is regular.
\item[(c)]
Find the long-term probability distribution for the state of the Markov
chain.
\end{enumerate}
\end{exercise}

\begin{exercise}
\label{ex:markov2}
Consider the following transition diagram:

\medskip
\centerline{%
\includegraphics[width=3in]{fig/trans_diag.eps}
}
\begin{enumerate}
\item[(a)] Find the transition matrix,
and show that the Markov chain is regular.
\item[(b)]  Find the long-term probability distribution of the states
$A$, $B$, and $C$.
\end{enumerate}
\end{exercise}
\end{exercises}
%
%
\newpage
%
\section{Absorbing Markov Chains}

We consider another important class of Markov chains.
\begin{definition}
A state $S_k$ of a Markov chain is called
an \emph{absorbing state} if, once the Markov chains enters
the state, it remains there forever.  In other words,
the probability of leaving the state is zero.
This means $a_{kk} = 1$, and $a_{jk} = 0$ for $j\ne k$.
\end{definition}

\begin{definition}
A Markov chain is called an \emph{absorbing chain}
if
\begin{enumerate}
\item[(i)] it has at least one absorbing state; and
\item[(ii)] for every state in the chain, the probability
of reaching an absorbing state in a finite number
of steps is nonzero.
\end{enumerate}
\end{definition}
An essential observation for an absorbing Markov chain
is that it will eventually enter an absorbing state.
(This is a consequence of the fact that if
a random event has a probability $p>0$ of occurring,
then the probability that it does not occur is $1-p$,
and the probability that it does not occur in $n$
trials is $(1-p)^n$.  As $n\rightarrow\infty$, the
probability that the event does not occur
goes to zero\footnote{%
\emph{``Infinity converts the possible into the inevitable.''}
 -- Norman Cousins 
}.)
The non-absorbing states in an absorbing Markov chain
are called \emph{transient states}.

\medskip
Suppose an absorbing Markov chain
has $k$ absorbing states and $\ell$ transient
states.
If, in our set of states, we list the absorbing
states first, we see that the
transition matrix has the form

\medskip
\centerline{%
\begin{minipage}{3in}
\begin{tabbing}
  MMMMM \= MMMMMMMMMMMMMMM \kill
    \> \hspace{0.8cm}Absorbing States \hspace{0.6cm}Transient States \\
    \> \hspace{1cm} $\overbrace{S_1 \;\; S_2 \;\;\; \cdots\;\; S_k}$
       \hspace{0.2cm} $\overbrace{\;\;S_{k+1} \quad \; \cdots \quad\;\; S_{k+\ell}\;\;}$ \\
    \> $\begin{matrix}
             S_1 \B \\
	     S_2 \T \\
	     \vdots \B \\
	     S_{k}  \\
	     S_{k+1}  \\
	     \vdots \\
	     S_{k+\ell}
	\end{matrix}
    \begin{bmatrix}
         1 & 0 & \cdots & 0 & p_{1,k+1} & \cdots & p_{1,k+\ell} \T \\
	 0 & 1 &        & \vdots  &   \vdots  &        & \vdots      \\
	 \vdots & & \ddots & 0\\
	 0      & \cdots & 0 & 1       & p_{k,k+1} & \cdots & p_{k,k+\ell} \\
	 0      & \cdots & \cdots & 0 & p_{k+1,k+1} & \cdots & p_{k+1,k+\ell} \\
	 \vdots & & & \vdots & \vdots & & \vdots \\
	 0  & \cdots & \cdots & 0 & p_{k+\ell,k+1} & \cdots & p_{k+\ell,k+\ell}
    \end{bmatrix}$
\end{tabbing}
\end{minipage}
}

% \begin{equation}
%     \begin{bmatrix}
%          1 & 0 & \cdots & 0 & p_{1,k+1} & \cdots & p_{1,k+\ell} \\
% 	 0 & 1 &        & \vdots  &   \vdots  &        & \vdots      \\
% 	 \vdots & & \ddots & 0\\
% 	 0      & \cdots & 0 & 1       & p_{k,k+1} & \cdots & p_{k,k+\ell} \\
% 	 0      & \cdots & \cdots & 0 & p_{k+1,k+1} & \cdots & p_{k+1,k+\ell} \\
% 	 \vdots & & & \vdots & \vdots & & \vdots \\
% 	 0  & \cdots & \cdots & 0 & p_{k+\ell,k+1} & \cdots & p_{k+\ell,k+\ell}
%     \end{bmatrix}.
% \end{equation}

\medskip
\noindent
That is, we may partition $A$ as
\begin{equation}
  A = \begin{bmatrix}
             I & R \\
	     \textbf{0} & Q \\
      \end{bmatrix}
\end{equation}
where $I$ is $k\times k$, $R$ is $k \times \ell$,
\textbf{0} is $\ell\times k$ and $Q$ is $\ell\times \ell$.
$R$ gives the probabilities of transitions from
transient states to absorbing states, while $Q$
gives the probabilities of transitions from
transient states to transient states.

Consider the powers of $A$:
\begin{equation}
  A^2 = \begin{bmatrix}
             I & R(I+Q) \\
	     \textbf{0} & Q^2 \\
        \end{bmatrix},
	\quad
  A^3 = \begin{bmatrix}
             I & R(I+Q+Q^2) \\
	     \textbf{0} & Q^3 \\
        \end{bmatrix},
\end{equation}
and, in general,
\begin{equation}
  A^n = \begin{bmatrix}
             I & R(I+Q+Q^2+\cdots+Q^{n-1}) \\
	     \textbf{0} & Q^n \\
        \end{bmatrix}
     = \begin{bmatrix}
             I & R\sum_{i=0}^{n-1}Q^i \\
	     \textbf{0} & Q^n \\
        \end{bmatrix},
\end{equation}

\noindent
Now I claim that\footnote{%
There is a slight abuse of notation in the formula given.
I use the symbol \textbf{0} to mean ``a matrix of zeros
of the appropriate size''. The two \textbf{0}'s in the
formula are not necessarily the same size.
The \textbf{0} in the lower left is $\ell \times k$,
while the \textbf{0} in the lower right is $\ell\times\ell$.}
\begin{equation}
  \lim_{n\rightarrow\infty} A^n
   = \begin{bmatrix}
             I & R(I-Q)^{-1} \\
	     \textbf{0} & \textbf{0}
     \end{bmatrix}
\end{equation}
That is, we have
\begin{enumerate}
\item $Q^n\rightarrow\textbf{0}$ as $n \rightarrow\infty$, and
\item $\ds \sum_{i=0}^{\infty} Q^i = (I-Q)^{-1}$.
\end{enumerate}
The first claim, $Q^n\rightarrow\textbf{0}$, means that in the
long run, the probability is $0$ that the process will be in 
a transient state. In other words, the probability is $1$
that the process will eventually enter an absorbing state.
We can derive the second claim as follows.
Let
\begin{equation}
   U = \sum_{i=0}^{\infty}Q^i = I + Q + Q^2 + \cdots 
\end{equation}
Then
\begin{equation}
   QU = Q\sum_{i=0}^{\infty}Q^i = Q + Q^2 + Q^3 + \cdots
      = (I + Q + Q^2 + Q^3+\cdots) - I
      = U-I.
\end{equation}
Then $QU=U-I$ implies
\begin{equation}
\begin{split}
   U-UQ & = I \\
   U(I-Q) & = I \\
   U & = (I-Q)^{-1},
\end{split}
\end{equation}
which is the second claim.
% (The claims can be rigorously justified, but for our purposes,
% the above arguments will suffice.)

The matrix $R(I-Q)^{-1}$ has the following meaning.
The column $i$ of $R(I-Q)^{-1}$ gives the probabilities of
ending up in each of the absorbing states, given that the
process started in the $i^{\textrm{th}}$ transient state.

There is more information that we can glean from
$(I-Q)^{-1}$.
For convenience, call the transient states
$T_1$, $T_2$, \ldots, $T_{\ell}$.
(So $T_j = S_{k+j}$.)
Let $V(T_i,T_j)$ be the expected number 
of times that the process is in state $T_i$, given
that it started in $T_j$.
($V$ stands for the number of ``\textbf{v}isits''.)
Also recall that $Q$ gives the probabilities
of transitions from transient states to transient states,
so
\begin{equation}
  q_{ij} = \textrm{Prob( State $n+1$ is $T_i$ $|$ State $n$ is $T_j$)} 
\end{equation}
I claim that $V(T_i,T_j)$ satisfies the following equation:
\begin{equation}
  V(T_i,T_j) = e_{ij} + q_{i1}V(T_1,T_j) + q_{i2} V(T_2,T_j)
                 + \cdots + q_{i\ell}V(T_{\ell},T_j)
\label{eqn:Vij}
\end{equation}
where
\begin{equation}
   e_{ij} = \begin{cases}
                  1 & \textrm{if} \;i=j \\
		  0 & \textrm{otherwise}
            \end{cases}
\end{equation}
Why? Consider just the term $q_{i1} V(T_1,T_j)$.
Given that the process started in $T_j$,
$V(T_1,T_j)$ gives the expected number of times that the process
will be in $T_1$.
The number $q_{i1}$ gives the probability of making
a transition from $T_1$ to $T_i$.
Therefore, the product $q_{i1}V(T_1,T_j)$
gives the \emph{expected number of transitions from} $T_1$ \emph{to} $T_i$,
given that the process started in $T_j$.
Similarly, $q_{i2}V(T_2,T_j)$ gives the expected number of transitions
from $T_2$ to $T_i$, and so on.
Therefore the total number of expected transition to $T_i$
is $q_{i1}V(T_1,T_j)+q_{i2}V(T_2,T_j)+\cdots+q_{i\ell}V(T_{\ell},T_j)$.

The expected number of transitions into a state is the same
as the expected number of times that the process is in a state, except in
one case.  Counting the \emph{transitions} misses the state
in which the process started, since there is no ``transition''
into the initial state.  This is why the term $e_{ij}$
is included in \eqref{eqn:Vij}.  When we consider
$V(T_i,T_i)$, we have to add 1 to the expected number of transitions
into $T_i$ to get the correct expected number of times that the
process was in $T_i$.

Equation \eqref{eqn:Vij} is actually a set of $\ell^2$ equations,
one for each possible $(i,j)$.  In fact, it is just one component
of a matrix equation.
Let
\begin{equation}
   N = \begin{bmatrix}
            V(T_1,T_1) & V(T_1,T_2) & \cdots & V(T_1,T_{\ell}) \\
	    V(T_2,T_1) & V(T_2,T_2) \\
	       \vdots  &  &   \ddots \\
	    V(T_{\ell},T_1) & & & V(T_{\ell},T_{\ell})
       \end{bmatrix}
\end{equation}
Then equation \eqref{eqn:Vij} is the $(i,j)$ entry in the
matrix equation
\begin{equation}
   N = I + QN.
\label{eqn:determinesN}
\end{equation}
(You should check this yourself!)
Solving \eqref{eqn:determinesN} gives
\begin{equation}
\begin{split}
  N-QN & = I \\
  (I-Q)N & = I \\
       N & = (I-Q)^{-1}
\end{split}
\end{equation}
Thus the $(i,j)$ entry of $(I-Q)^{-1}$ gives the expected number
of times that the process is in the $i^{\textrm{th}}$ transient
state, given that it started in the $j^{\textrm{th}}$
transient state.
It follows that the sum of the $j^{\textrm{th}}$ column of $N$
gives the expected number of times that the process will be in 
some transient state, given that the process started
in the $j^{\textrm{th}}$ transient state.

\begin{xexample}
\textbf{The Coin and Die Game.}
In this game there are two players, \emph{Coin}
and \emph{Die}. \emph{Coin} has a coin, and \emph{Die} has a
single six-sided die. The rules are:

\begin{itemize}
\item
When it is \emph{Coin}'s turn, he or she flips the coin.
If the coin turns up \textbf{heads}, \emph{Coin} wins the game.
If the coin turns up \textbf{tails}, it is \emph{Die}'s turn.

\item
When it is \emph{Die}'s turn, he or she rolls the die.
If the die turns up \textbf{1}, \emph{Die} wins.
If the die turns up \textbf{6}, it is \emph{Coin}'s turn.
Otherwise, \emph{Die} rolls again.
\end{itemize}

\noindent
When it is \emph{Coin}'s turn, the probability
is $1/2$ that \emph{Coin} will win and $1/2$ that
it will become \emph{Die}'s turn.
When it is \emph{Die}'s turn, the probabilities are
\begin{itemize}
\item $1/6$ that \emph{Die} will roll a \textbf{1} and win,
\item $1/6$ that \emph{Die} will roll a \textbf{6}
      and it will become \emph{Coin}'s turn, and
\item $2/3$ that \emph{Die} will
roll a \textbf{2}, \textbf{3}, \textbf{4}, or \textbf{5} and
have another turn.
\end{itemize}

\noindent
To describe this process as a Markov chain, we define
four \emph{states} of the process:
\begin{itemize}
\item \emph{State 1}: \emph{Coin} has won the game.
\item \emph{State 2}: \emph{Die} has won the game.
\item \emph{State 3}: It is \emph{Coin}'s turn.
\item \emph{State 4}: It is \emph{Die}'s turn.
\end{itemize}
We represent the possible outcomes in
the following transition diagram:

\medskip
\centerline{%
\includegraphics[width=4in]{fig/CoinAndDieStateDiagram.eps}
}

\medskip
\noindent
This is an absorbing Markov chain.  The absorbing
states are State 1 and State 2, in which one of the players
has won the game, and the transient states
are State 3 and State 4.

The transition matrix is
\begin{equation}
 A = \begin{bmatrix}
         1 & 0 & 1/2 &  0  \\
	 0 & 1 &  0  & 1/6 \\
	 0 & 0 &  0  & 1/6 \\
	 0 & 0 & 1/2 & 2/3 
     \end{bmatrix}
   = \begin{bmatrix}
         1 & 0 & \vdots & 1/2 &  0  \\
	 0 & 1 & \vdots &  0  & 1/6 \\
	 \hdotsfor{5} \\
	 0 & 0 & \vdots &  0  & 1/6 \\
	 0 & 0 & \vdots & 1/2 & 2/3 
     \end{bmatrix}
   = \begin{bmatrix}
         I & R \\
	 \textbf{0} & Q 
     \end{bmatrix}
\end{equation}
where
\begin{equation}
  R = \begin{bmatrix}
            1/2 & 0 \\
	     0  & 1/6
      \end{bmatrix}
  \quad\textrm{and}\quad
  Q = \begin{bmatrix}
            0 & 1/6 \\
	    1/2 & 2/3
      \end{bmatrix}.
\end{equation}
We find
\begin{equation}
  I-Q = \begin{bmatrix}
            1  & -1/6 \\
	  -1/2 & 1/3 
        \end{bmatrix},
\end{equation}
so
\begin{equation}
  N = (I-Q)^{-1} = \begin{bmatrix}
                 4/3 & 2/3 \\
		 2   &  4
               \end{bmatrix},
\label{eqn:CDN}
\end{equation}
and
\begin{equation}
  R(I-Q)^{-1} = \begin{bmatrix}
                   2/3 & 1/3 \\
		   1/3 & 2/3
                \end{bmatrix}
\end{equation}
Recall that the first column of $R(I-Q)^{-1}$
gives the probabilities of entering State 1 or
State 2 if the process starts in State 3.
``Starts in State 3'' means \emph{Coin} goes first,
and ``State 1'' means \emph{Coin} wins,
so this matrix tells us that if \emph{Coin} goes
first, the probability that \emph{Coin} will win
is $2/3$, and the probability that
\emph{Die} will win is $1/3$.
Similarly, if \emph{Die} goes first,
the probability that \emph{Coin} will win
is $1/3$, and the probability that
\emph{Die} will win is $2/3$.

From \eqref{eqn:CDN}, we can also conclude the following:
\begin{enumerate}
\item If \emph{Coin} goes first, then the expected number of
turns for \emph{Coin} is $4/3$, and the expected number of
turns for \emph{Die} is $2$.  Thus the expected total number
of turns is $10/3\approx 3.33$.
\item If \emph{Die} goes first, then the expected number of
turns for \emph{Coin} is $2/3$, and the expected number of
turns for \emph{Die} is $4$.  Thus the expected total number
of turns is $14/3\approx 4.67$.
\end{enumerate}

The following table gives the results of an
experiment with the Coin and Die Game along with
the predictions of the theory.
A total of 220 games were played in which
\emph{Coin} went first.
\emph{Coin} won 138 times, and the total number of
turns was 821, for an average of 3.73 turns per game.

\medskip
\centerline{
\fbox{%
\begin{tabular}{lrr}
\emph{Quantity}  & \emph{Predicted} & \emph{Experiment} \\
Percentage Won by Coin & 66.7     &   62.7      \\
Average Number of Turns per Game & 3.33  & 3.73
\end{tabular} 
}}

\medskip
It appears that in this experiment, \emph{Die} won more
often than predicted by the theory.
Presumably if the games was played the game a lot more, the
experimental results would approach the predicted results.
\end{xexample}

\begin{exercises}


%
\begin{exercise}
\label{ex:markov3}
We consider a modification of the ``Coin and Die'' game.
The rules for \emph{Coin} are the same:
\begin{itemize}
\item If the coin turns up \textbf{heads}, \emph{Coin} wins.
\item If the coin turns up \textbf{tails}, it is \emph{Die}'s turn.
\end{itemize}
The new rules for \emph{Die} are:
\begin{itemize}
\item If the die turns up \textbf{1 or 2}, \emph{Die} wins.
\item If the die turns up \textbf{3}, \emph{Die} rolls again.
\item If the die turns up \textbf{4, 5, or 6}, it is \emph{Coin}'s turn.
\end{itemize}
\begin{enumerate}
\item[(a)] If \emph{Coin} goes first, what is the probability that
\emph{Coin} wins, and what is the expected number of turns?
\item[(b)] If \emph{Die} goes first, what is the probability that
\emph{Coin} wins, and what is the expected number of turns?
\end{enumerate}
\end{exercise}
\end{exercises}
