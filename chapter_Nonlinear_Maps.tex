
\chapter{Nonlinear Maps}
\section{One-dimensional Maps}
The general one-dimensional map has the form
\begin{equation}
   x(n+1) = f(x(n)), \quad \textrm{given $x(0)$}.
\label{eqn:onedmap}
\end{equation}
We'll develop several tools for studying such maps.


\subsection*{``Cobwebbing'': The Graphical Iteration Procedure}
There is a simple graphical procedure for generating the
iterations of a one-dimensional map.
As an example, we consider the logistic map
\begin{equation}
   x(n+1) = rx(n)(1-x(n))
\label{eqn:logisticmap}
\end{equation}
where $r > 0$ is a constant.
Table~\ref{tbl:data} shows the first few iterates in the
sequence that results when
$r=2.8$ and $x_0=0.1$.
\begin{table}
\centerline{%
\begin{tabular}{|c|l|c|l|}
\hline
 $n$  &  $x(n)$ & $n$ & $x(n)$ \\ 
\hline
  0  &  0.1    & 5 & 0.6771 \\ 
\hline
  1 &  0.2520  & 6 & 0.6122 \\ 
\hline
  2 &  0.5278  & 7 & 0.6648 \\ 
\hline
  3 &  0.6978  & 8 & 0.6240 \\ 
\hline
  4 &  0.5904  & 9 & 0.6570 \\ 
\hline
\end{tabular}
\vspace{0.25cm}
}
\caption{Iteration of the logistic map~\eqref{eqn:logisticmap}, with
$r=2.8$ and $x_0=0.1$.}
\label{tbl:data}
\end{table}
%
Figure~\ref{fig:logistic_ts} shows the plot of $x(n)$ as a function
of $n$.
%
\begin{figure}
\centerline{\includegraphics{matlab_map1d/logisticmap_timeseries.eps}}
\caption{The first several iterates of the logistic map~\eqref{eqn:logisticmap}
with $r=2.8$ and $x_0=0.1$. This is a plot of the data shown
in Table~\ref{tbl:data}.}
\label{fig:logistic_ts}
\end{figure}
%

The graphical technique for finding the iterations of this map
begins by plotting the graph of the $f(x) = rx(1-x)$, as in
Figure~\ref{fig:logisticmap_example} (where $r=2.8$).
Given $x(0)$, we draw a line from the $x=x(0)$ on the $x$
axis up the the graph of $x(0)$ to obtain $x(1)$.
To obtain $x(2)$, we need to go to $x=x(1)$ on the $x$ axis.
This can be done by drawing a line horizontally from
$(x(0),x(1))$ to the line $y=x$.
See the upper left plot of Figure~\ref{fig:cobwebsequence}.
To find $x(2)$, we again draw a line vertically
at $x=x(1)$ to the graph, and from the graph we draw a 
line horizontally to $y=x$.
See the upper right plot of Figure~\ref{fig:cobwebsequence}.
We continue this process to generate further iterations,
as in the lower left and right plots of
Figure~\ref{fig:cobwebsequence}.
In these plots, I've included vertical lines from the $x$ axis
up to the graph, but we don't really need these lines.
In practice, the procedure is: 
draw a line vertically
to the graph, then 
draw a line horizontally to $y=x$,  and then repeat
the process.
An example is shown in Figure~\ref{fig:cobwebfinal}.

\begin{figure}
\centerline{\includegraphics[width=3in]{matlab_map1d/logisticmap_example.eps}}
\caption{A graph of the logistic map $rx(1-x)$ for $r=2.8$.}
\label{fig:logisticmap_example}
\end{figure}
%
\begin{figure}
\centerline{%
\includegraphics[width=2.65in]{matlab_map1d/logisticmap_cobweb1.eps}
\includegraphics[width=2.65in]{matlab_map1d/logisticmap_cobweb2.eps}
}
\centerline{%
\includegraphics[width=2.65in]{matlab_map1d/logisticmap_cobweb3.eps}
\includegraphics[width=2.65in]{matlab_map1d/logisticmap_cobweb4.eps}
}
\caption{A sequence of plots that illustrate
``cobwebbing''. The initial point is $x(0)=0.1$.}
\label{fig:cobwebsequence}
\end{figure}
%
\begin{figure}
\centerline{%
\includegraphics[width=3.25in]{matlab_map1d/logisticmap_cobwebfinal.eps}
}
\caption{When cobwebbing, we usually don't bother dropping
lines down to the horizontal axis. After the first vertical
line is drawn from $(x(0),x(0))$ to the graph, we draw
a line horizontally to the line
$y=x$, and then another line vertically to graph, and repeat.
This cobweb plot shows
the same data listed in Table~\ref{tbl:data} and plotted in
Figure~\ref{fig:logistic_ts}.}
\label{fig:cobwebfinal} 
\end{figure}
%


%
\subsection*{Fixed Points, Stability, Linearization}
A \emph{fixed point}\index{fixed point} of the map \eqref{eqn:onedmap}
is a point $x^*$ where $f(x^*)=x^*$.
If $x(0)=x^*$, then $x(n)=x^*$ for all $n>0$, so the fixed points
of $f$ are the constant solutions to \eqref{eqn:onedmap}.

In the logistic map shown in Figure~\ref{fig:logisticmap_example},
we see there are two
points where $f(x)=x$.  These are the points
where the graph of $f$ crosses the line $y=x$.
For the example shown in Figure~\ref{fig:logisticmap_example},
the graph is $f(x) = 2.8x(1-x)$, so to find the fixed points,
we must solve
\begin{equation}
  2.8x(1-x) = x \implies -2.8x^2 + 1.8x = 0
  \implies x = 0 \;\textrm{or}\; x = \frac{1.8}{2.8}
\end{equation}
So the two fixed points are $x=0$ and $x=\frac{1.8}{2.8}\approx 0.643$.

\begin{definition}
The fixed point $x^*$ is a \emph{sink} or \emph{attractor}
if there is a neighborhood $N$ of $x^*$ such that
$x(n)\rightarrow x^*$ for all $x(0)$ in $N$.
We also say $x^*$ is \emph{asymptotically stable}.
\end{definition}
\begin{definition}
The fixed point $x^*$ is a \emph{source} or \emph{repellor}
if there is a neighborhood $N$ of $x^*$ such that if $x(0)$
is in $N$, then $x(n)$ eventually leaves $N$.
\end{definition}
\begin{definition}
The fixed point $x^*$ is \emph{unstable} if for every
neighborhood $N$ of $x^*$, there are points arbitrarily close
to $x^*$ whose iterates leave $N$.
\end{definition}

\vspace{0.2cm}
Note that a source is unstable, but an unstable fixed point
is not necessarily a source.  See, for example,
Figure~\ref{fig:noconclusion1}.

\noindent
\textbf{Behavior near a fixed point: the linearization.}
Let $x^*$ be a fixed point of \eqref{eqn:onedmap}.
If $x$ is close to $x^*$,
we can approximate $f(x)$ with the tangent line at $x^*$:
\begin{equation}
   f(x) \approx f(x^*) + f'(x^*)(x-x^*).
\end{equation}
Let $u(n) = x(n)-x^*$ (i.e. $x(n) = x^* + u(n)$).
Then, by replacing $x(n)$ with $x^*+u(n)$
in \eqref{eqn:onedmap}, we obtain
\begin{equation}
  x^* + u(n+1) = f(x^*+u(n)) \approx f(x^*)+f'(x^*)u(n).
\end{equation}
Since $f(x^*)=x^*$, we can cancel $x^*$ on the left
and $f(x^*)$ on the right.
We are left with the
\emph{linearization of the map at } $x^*$:
\begin{equation}
  u(n+1) = f'(x^*)u(n)
\end{equation}
This is a linear map.
We have already seen how the solution to the linear
map depends on $a=f'(x^*)$; see Figures~\ref{fig:linearcases_pos}
and \ref{fig:linearcases_neg}.
However, the linearization is just an approximation
to the actual map~\eqref{eqn:onedmap}.
The following theorem tells us when the linear approximation
is ``good enough'' to classify the stability of the
fixed point $x^*$ of \eqref{eqn:onedmap}.

\begin{theorem}
(i) If $|f'(x^*)| < 1$, then $x^*$ is a sink.
(ii) If $|f'(x^*)| > 1$, then $x^*$ is a source.
\end{theorem}

\begin{xexample}
Consider again the logistic map \eqref{eqn:logisticmap}
with $r=2.8$.  The graph is shown in
Figure~\ref{fig:logisticmap_example}, and earlier we found the
fixed points to be $x^*_1=0$ and $x^*_2=\frac{1.8}{2.8}$.
We find
\begin{equation}
  f'(x) = 2.8(1-2x).
\end{equation}
We use Theorem 1 to determine the stability of the fixed points.
\begin{itemize}
\item
At $x^*_1$, we have $f'(0) = 2.8 > 1$, so by Theorem 1,
$x^*_1$ is a source.
\item
At $x^*_2$, we have $f'(1.8/2.8) = -0.8$,
so by Theorem 1,
$x^*_2$ is a sink.
\end{itemize}
\end{xexample}

\bigskip
If $|f'(x^*)|=1$, we can not make any conclusions about the
stability of the fixed point.  Examples demonstrating why this
is the case are shown in Figures~\ref{fig:noconclusion2},
\ref{fig:noconclusion3}
and \ref{fig:noconclusion1}.
%
\begin{figure}
\centerline{%
\includegraphics[width=2.25in]{matlab_map1d/noconclusion2.eps}%
}
\caption{An example that shows why we can not determine
the stability of a fixed point based on the linearization
when
$|f'(x^*)|=1$.  The fixed point is $x^*=0$, and $f'(x^*)=1$.
In this case, $x^*$ is a source.}
\label{fig:noconclusion2}
\end{figure}
%
\begin{figure}
\centerline{%
\includegraphics[width=2.25in]{matlab_map1d/noconclusion3.eps}%
}
\caption{An example that shows why we can not determine
the stability of a fixed point based on the linearization
when
$|f'(x^*)|=1$.  The fixed point is $x^*=0$, and $f'(x^*)=1$.
In this case, $x^*$ is a sink.}
\label{fig:noconclusion3}
\end{figure}
%
\begin{figure}
\centerline{%
\includegraphics[width=2.25in]{matlab_map1d/noconclusion1.eps}%
}
\caption{Another example that shows why we can not determine
the stability of a fixed point based on the linearization
when
$|f'(x^*)|=1$.  In this case, $x^*=1/2$ is a fixed point,
and $f'(x^*)=1$.
The iterations of points close to but less than $x^*$
converge to $x^*$, but iterations of points that are
greater than $x^*$ diverge from $x^*$.
In this case, $x^*$ is unstable (but it is not a source).
}
\label{fig:noconclusion1}
\end{figure}
%
%
\newpage
%
\section{A Bit of Chaos}
We give a brief demonstration of \emph{chaos}.\index{chaos}
A thorough analysis of this phenomenon would lead us too far
astray, but it is important to know that chaos is a possible
(and, in fact, common) behavior of even very simple
discrete maps.
As an example, we consider
\begin{equation}
   x(n+1) = 4x(n)(1-x(n)).
\end{equation}
This is the logistic map with the parameter set to 4.

\emph{Describe solutions; philosophical implications: solutions
do not converge to stable fixed points or periodic solutions.}
%
%
\newpage
%
\section{Two-dimensional Maps}
We now consider two-dimensional maps, or ``maps of the plane''.
Such a map has the form
\begin{equation}
\begin{split}
    x(n+1) & = f(x(n),y(n)) \\
    y(n+1) & = g(x(n),y(n))
\end{split}
\label{eqn:map}
\end{equation}
%
The fixed points of a two-dimensional map satisfy the equations
\begin{equation}
\begin{split}
   x & = f(x,y), \\
   y & = g(x,y).
\end{split}
\label{eqn:twodimfixedpoints}
\end{equation}
The set of fixed points of a linear map is
the nullspace (or kernel) of the matrix $A$; if
$A$ is nonsingular, only the origin is a fixed point.
If $A$ singular, then every point in the nullspace
is a fixed point;  a linear map can not have an
isolated fixed point other than the origin.

Unlike a linear map, a nonlinear map may have more than
one isolated fixed point.  The following example
has (typically) two isolated fixed points.
%
\begin{xexample}
Consider the map
\begin{equation}
\begin{split}
  x(n+1) & = (1- x(n))x(n) + b y(n) \\
  y(n+1) & = \frac{y(n)}{2} + x(n)
\end{split}
\label{eqn:linearizationexample}
\end{equation}
where $b$ is a constant.
This is a map of the form shown in~\eqref{eqn:map}, with
\begin{equation}
   f(x,y) = (1-x)x+by \quad \textrm{and} \quad g(x,y) = \frac{y}{2}+x.
\end{equation}
To find the fixed points, we must solve
\begin{equation}
   (1 - x)x + by = x, \quad \textrm{and} \quad \frac{y}{2}+x = y.
\end{equation}
From the second equation we have $y = 2x$; putting this into the first equation
leads to
\begin{equation}
   x^2-2bx = 0
\end{equation}
and the solutions are
\begin{equation}
  x = 0 \quad \textrm{or} \quad x= 2b.
\end{equation}
Thus the fixed points are
$(0,0)$ and $(2b,4b)$.

The plot in Figure~\ref{fig:nonlinearexample}
shows the first 20 or so iterations for four
initial conditions $(0.4,0)$, $(0.78,0)$,
$(0.79,0)$ and $(1,0)$ when $b=-1/4$.
\begin{figure}
\centerline{%
\includegraphics[width=5.5in]{scilab/nonlinearexample.eps}
}
\caption{Trajectories of the map
\eqref{eqn:linearizationexample} with
$b=-1/4$ and with the initial conditions
$(0.4,0)$, $(0.78,0)$,
$(0.79,0)$ and $(1,0)$.
The fixed points $(0,0)$ and $(-1/2,-1)$ are indicated
with circles.
Note that the two trajectories that begin at
$(0.78,0)$ and $(0.79,0)$ remain close to each other
until they approach the vicinity of the
fixed point at $(-1/2,-1)$.
}
\label{fig:nonlinearexample}
\end{figure}
\end{xexample}
%
\newpage
%
\section{Linearization and Stability of Fixed Points}
%
In this section we discuss the \emph{linearization}\index{linearization}
of a map at a fixed point. 
We will focus on two-dimensional systems, but the
techniques used here also work in $n$ dimensions.
While there are a few differences, the steps that
we follow here are fundamentally the same as those
in Section~\ref{sec:DELinearization}.

A two-dimensional map has the form
shown in \eqref{eqn:map}.
The constant solutions to this system are called the fixed points
of the map.
They satisfy the equations \eqref{eqn:twodimfixedpoints}.

Replacing $f(x,y)$ with its tangent plane
approximation at $(x^*,y^*)$ converts the first equation in~\eqref{eqn:map}
to
\begin{equation}
  x(n+1) = f(x^*,y^*) + f_x(x^*,y^*)(x(n)-x^*) + f_y(x^*,y^*)(y(n)-y^*).
\label{eqn:mapapprox1}
\end{equation}
We define
\begin{equation}
    u(n) = x(n) - x^* \quad\textrm{and}\quad
    v(n) = y(n) - y^*.
\end{equation}
  Then $x(n+1) = u(n+1) + x^*$, and~\eqref{eqn:mapapprox1}
becomes
\begin{equation}
   u(n+1) + x^* = f(x^*,y^*) + f_x(x^*,y^*)u(n) + f_y(x^*,y^*)v(n)
\end{equation}
At a fixed point, $f(x^*,y^*) = x^*$, so we can cancel the $x^*$ on the
left with the $f(x^*,y^*)$ on the right.  Applying the same arguments
to the equation for $y(n+1)$ results in the system
\begin{equation}
\begin{split}
   u(n+1) & = f_x(x^*,y^*)u(n) + f_y(x^*,y^*)v(n) \\
   v(n+1) & = g_x(x^*,y^*)u(n) + g_y(x^*,y^*)v(n) \\
\end{split}
\end{equation}
This is the linearization\index{linearization}
of~\eqref{eqn:map} at $(x^*,y^*)$.
By defining the vector $\BU(n) = \begin{bmatrix} u(n) \\ v(n) \end{bmatrix}$,
we can write the system in matrix form as
\begin{equation}
  \BU(n+1) = J\BU(n),
\label{eqn:linearizedmap}
\end{equation}
where
\begin{equation}
   J = \begin{bmatrix}
             f_x(x^*,y^*) & f_y(x^*,y^*) \\
	     g_x(x^*,y^*) & g_y(x^*,y^*)
       \end{bmatrix}
\label{eqn:mapjac}
\end{equation}
is the Jacobian matrix\index{Jacobian matrix}.
(Note that this is the same matrix used in the linearization
at an equilibrium point of a system of differential equations.)

\noindent
\textbf{What does the linearization tell us about the original system?}
We have the following result:
\emph{If neither eigenvalue has magnitude equal to $1$,
then the behavior of the system \eqref{eqn:map}
near $(x^*,y^*)$ is qualitatively the same as the behavior of the
linear approximation \eqref{eqn:linearizedmap}.}
The classification of the fixed point of the nonlinear map
is the same as the classification of the origin in the linearization.

In a one-dimensional map $x(n+1) = f(x(n))$, with a fixed
point $x^*$, the 
Jacobian ``matrix'' is simply $f'(x^*)$.  We saw examples
in Section~\ref{sec:OneDimMaps} that showed why we could not determine
the stability of a fixed point based on just the linearization
in the case $|f'(x^*)|=1$.  The above results are a generalization
of that phenomena to higher dimensions.


\begin{xexample}
In this rather long example, we determine
how the stability of a fixed point depends on a parameter.
We consider the example given in
equation \eqref{eqn:linearizationexample}.
We will determine the behavior near $(0,0)$.

The Jacobian matrix 
at a fixed point $(x^*,y^*)$ is
\begin{equation}
   J = \begin{bmatrix}
          f_x(x^*,y^*) & f_y(x^*,y^*) \\
	  g_x(x^*,y^*) & g_y(x^*,y^*) 
       \end{bmatrix}
     = \begin{bmatrix}
          1-2x^* & b \\
	   1    & \frac{1}{2}
       \end{bmatrix}
\end{equation}
At the fixed point $(0,0)$, we find
\begin{equation}
  J = \begin{bmatrix}
          1 & b \\
	  1 & \frac{1}{2}
      \end{bmatrix}
\end{equation}
The eigenvalues of the Jacobian are
\begin{equation}
  \lambda_1 = \frac{3}{4} - \frac{1}{2} \sqrt{\frac{1}{4}+4b}, \quad
  \lambda_2 = \frac{3}{4} + \frac{1}{2} \sqrt{\frac{1}{4}+4b}.
\label{eqn:exeigvals}
\end{equation}
The first thing to determine is whether the eigenvalues are
complex or real.  The eigenvalues are complex if
\begin{equation}
  \frac{1}{4} + 4b < 0 \implies b < -\frac{1}{16}.
\end{equation}
So we have complex eigenvalues if $b < -\frac{1}{16}$ and real
eigenvalues if $b \ge -\frac{1}{16}$.
We treat each case separately.

When $b < -\frac{1}{16}$, the eigenvalues are
\begin{equation}
   \lambda = \frac{3}{4} \pm \frac{i}{2}\sqrt{-\frac{1}{4}-4b}
\end{equation}
To classify the fixed point and determine its stability,
we must determine whether the magnitude of the eigenvalues
are greater than or less than one.  To do this, we'll find the
value(s) of $b$, if any, where $|\lambda| = 1$.
(Note that this is equivalent to $|\lambda|^2 = 1$.)
We have
\begin{equation}
\begin{split}
  1 & = |\lambda|^2 \\
    & = \left(\frac{3}{4}\right)^2 + \left(\frac{1}{2}\sqrt{-\frac{1}{4}-4b} \right)^2 \\
    & = \frac{1}{2} - b
\end{split}
\end{equation}
which gives $b=-\frac{1}{2}$.
So we have the following:
\begin{itemize}
\item
If $b < -\frac{1}{2}$, then $|\lambda| > 1$, and $(0,0)$ is a spiral source.
\item
If $-\frac{1}{2} < b < -\frac{1}{16}$, then $|\lambda| < 1$, and $(0,0)$
is a spiral sink.
\end{itemize}
Next we consider $b > -\frac{1}{16}$, where the eigenvalues are real.
From~\eqref{eqn:exeigvals} we observe that $\lambda_1 < \frac{3}{4}$,
and $\lambda_2 > \frac{3}{4}$, so we only need to determine
if $\lambda_1 < -1$ or  $\lambda_2 > 1$.
First, consider
\begin{equation}
\begin{split}
   \lambda_1 & = -1 \\
   \frac{3}{4} - \frac{1}{2} \sqrt{\frac{1}{4}+4b} & = -1 \\
   \frac{1}{4}+4b & = \frac{49}{4} \\
   b & = 3
\end{split}
\end{equation}
So $\lambda_1 < -1$ if $b > 3$.

Next consider
\begin{equation}
\begin{split}
   \lambda_2 & = 1 \\
   \frac{3}{4} + \frac{1}{2} \sqrt{\frac{1}{4}+4b} & = 1 \\
   \frac{1}{4}+4b & = \frac{1}{4} \\
   b & = 0
\end{split}
\end{equation}
So $\lambda_2 < 1$ if $-\frac{1}{16} < b < 0$.
Thus, we have
\begin{itemize}
\item
If $-\frac{1}{16} < b < 0$, we have $-1 < \lambda_1 < 1$ and $\frac{3}{4} < \lambda_2 < 1$, so
$(0,0)$ is a sink.
\item
If $0 < b < 3$, then $-1 < \lambda_1 < 1$ but $\lambda_2 > 1$, so 
$(0,0)$ is a saddle.
\item
If $b > 3$, then $\lambda_1 < -1$ and $\lambda_2 > 1$, so
$(0,0)$ is a source.
\end{itemize}
The five bulleted statements above give all the cases for the
fixed point $(0,0)$.
Figure~\ref{fig:LinearizationFPExamples} shows trajectories
near the fixed point $(0,0)$ for several values of $b$.
\begin{figure}
\centerline{%
\includegraphics[width=2.5in]{matlab/LinearizationFPExample_n1.25.eps}
\includegraphics[width=2.5in]{matlab/LinearizationFPExample_n0.25.eps}
}
\centerline{%
\includegraphics[width=2.5in]{matlab/LinearizationFPExample_p0.25.eps}
\includegraphics[width=2.5in]{matlab/LinearizationFPExample_p3.50.eps}
}
\caption{Trajectories near $(0,0)$ of system
\eqref{eqn:linearizationexample}. Upper left: $b=-1.25$, spiral source.
Upper right: $b=-0.25$, spiral sink.
Lower left: $b=0.25$, saddle.
Lower right: $b=3.5$, source.}
\label{fig:LinearizationFPExamples}
\end{figure}
\end{xexample}

\begin{exercises}
\begin{exercise}
Consider the nonlinear map
\begin{equation}
\begin{split}
  x(n+1) & = x(n) + \frac{y(n)}{2} \\
  y(n+1) & = bx(n) + \frac{(y(n)-1)^3}{12}
\end{split}
\end{equation}
\begin{enumerate}
\item[(a)]
Show that this map has a single fixed point
(the location of which depends on $b$).
\item[(b)]
Determine how the classification and stability
of the fixed point depends on $b$.
\end{enumerate}
\end{exercise}
\end{exercises}

\newpage
% 
% \section{The Saddle-Node Bifurcation for Maps}
% 
% \section{The Hopf Bifurcation for Maps}
% 
% \section{The Period-Doubling Bifurcation for Maps}

\section{Marital Relationships: Gottman and Murray's Model}
The model discussed in this section is based on the
work of Gottman, Murray, \emph{et al}\cite{GM}.
\section{Exercises}
%
%